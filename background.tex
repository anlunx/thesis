
% !TEX root = thesis.tex

\chapter{Background}

In this chapter we introduces designs of existing restrictive effect systems and denotational effect systems and discusses their shortcomings.

\label{chapter-background}
\section{A Modular Design of Restrictive Effect Systems}

\label{sec:wyvern-effects-basics}


Restrictive Systems were originally proposed by \citet{lucassen87} to track reads and writes to memory, and then \citet{lucassen88} extended this effect system to support polymorphism.  Effects have since been used for a wide variety of purposes, including exceptions in Java~\cite{kiniry06} and asynchronous event handling~\cite{bracevac18}. \citet{turbak08} previously proposed effects as a mechanism for reasoning about security, which is the main application that we discuss.

This section describes the effect system designed by \citet{melicher20}, which introduces various effect system features such as effect members, effect abstraction, and path-dependent effects. The paper shows that the effect system lays a solid foundation for effect systems that can scale up and can deal with complexities of real-world code.



\subsection{Running Example}
\begin{figure}[t]
\begin{lstlisting}
resource type Logger
   effect ReadLog
   effect UpdateLog
   def readLog(): {this.ReadLog} String
   def updateLog(newEntry: String): {this.UpdateLog} Unit

module def logger(f: File): Logger
	effect ReadLog = {f.Read}
	effect UpdateLog = {f.Append}
	def readLog(): {this.ReadLog} String = f.read()
	def updateLog(newEntry: String): {this.UpdateLog} Unit = f.append(newEntry)
\end{lstlisting}
\caption{A type and a module implementing the logging facility in the text-editor application.}
\label{f-logger}
\end{figure}

Consider the code in Fig.~\ref{f-logger} that shows a type and a module implementing the logging facility of the text editor application.  In the given implementation of the \li{Logger} type, the \li{logger} module accesses the log file.\footnote{The keyword \li{resource} in the type definition indicates that the implementations of this type may have state and may access system resources; this is orthogonal to effect checking.} All modules of type \li{Logger} must have two methods: the \li{readLog} method that returns the content of the log file and the \li{updateLog} method that appends new entries to the log file. In addition, the \li{Logger} type declares two \textit{abstract} effects, \li{ReadLog} and \li{UpdateLog}, that are produced by the corresponding methods. These effects are abstract because they are not given a definition in the \li{Logger} type, and so it is up to the module implementing the \li{Logger} type to define what they mean. The effect names are user-defined, allowing the choice of meaningful names.

\begin{figure}[t]
\begin{lstlisting}
resource type File
   effect Read
   effect Write
   effect Append
   ...
   def read(): {this.Read} String
   def write(s: String): {this.Write} Unit
   def append(s: String): {this.Append} Unit
   ...
\end{lstlisting}
\caption{The type of the file resource.}
\label{f-file}
\end{figure}

The \li{logger} module implements the \li{Logger} type. To access the file system, an object of type \li{File} (shown in Fig.~\ref{f-file}) is passed into \li{logger} as a parameter. The \li{logger} module's effect declarations are those of the \li{Logger} type, except now they are \textit{concrete}, i.e., they have specific definitions. The \li{ReadLog} effect of the \li{logger} module is defined to be the \li{Read} effect of the \li{File} object, and accordingly, the \li{readLog} method, which produces the \li{ReadLog} effect, calls \li{f}'s \li{read} method. Similarly, the \li{UpdateLog} effect of the \li{logger} module is defined to be \li{f.Append}, and accordingly, the \li{updateLog} method, which produces the \li{UpdateLog} effect, calls \li{f}'s \li{append} method. In general, effects in a module or object definition must always be concrete, whereas effects in a type definition may be either abstract or concrete.


\subsection{Path-dependent Effects}

Effects are members of objects,\footnote{Modules are an important special case of objects} so we refer to them with the form \li{variable.EffectName}, where \li{variable} is an immutable reference to the object defining the effect and \li{EffectName} is the name of the effect. For example, in the definition of the \li{ReadLog} effect of the \li{logger} module, \li{f} is the variable referring to a specific file and \li{Read} is the effect that the \li{read} method of \li{f} produces. This conveniently ties together the resource and the effects produced on it (which represent the operations performed on it), helping a software architect or a security analyst to reason about how resources are used by any particular module and its methods. For example, when analyzing the effects produced by \li{logger}'s \li{readLog} method, a security analyst can quickly deduce that calling that method affects the file resource and, specifically, the file is read, simply by looking at the \li{Logger} type and \li{logger}'s effect definitions but not at the method's code. Furthermore, these properties can be automatically checked with an idiom of use: In addition to directly looking at the effect annotation of the method of the logger module, the security analyst may write client code that specify the effect that the logger module is allowed to have. If the logger module accesses system resources outside of the specified effect set, then the compiler would automatically reject the program.

Because an effect includes a reference to an object instance, our effect system can distinguish reads and writes on different file instances. If the developer does not want this level of precision, it is still possible to declare effects at the module level (i.e., as members of a \li{fileSystem} module object instance), and to share the same \li{Read} and \li{Write} effects (for example) across all files in \li{fileSystem}.

The basic mechanisms of path-dependence are borrowed from Scala and have been shown to scale well in practice. These mechanisms come from the Dependent Object Types (DOT) calculus \cite{amin14}, a type theory of Scala and related languages (including Wyvern). In our system, effects, instead of types are declared as members of objects.

\subsection{Effect Abstraction}
\label{sec:effect-abstraction}

An important and novel feature of our effect system design is the support for \textit{effect abstraction}. Effect abstraction is the ability to define higher-level effects in terms of lower-level effects and potentially to hide that definition from clients of an abstraction. In the logging example above, through the use of abstraction, we ``lifted'' low-level resources such as the file system (i.e., the \li{Read} and \li{Append} effects of the file) into higher-level resources such as a logging facility (i.e., the \li{ReadLog} and \li{UpdateLog} effect of the logger) and enabled application code to reason in terms of effects on those higher-level resources when appropriate.

Effect abstraction has several concrete benefits. First, it can be used to distinguish different uses of a low-level effect. For example, \li{system.FFI} describes any access to system resources via calls through the foreign function interface (FFI), but modules that define file and network I/O can represent these calls as different effects, which enables higher-level modules to reason about file and network access separately. Second, multiple low-level effects can be aggregated into a single high-level effect to reduce effect specification overhead. For instance, the \li{db.Query} effect might include both \li{file.Read} and \li{network.Access} effects. Third, by keeping an effect abstract, we can hide its implementation from clients, which facilitates software evolution: code defining a high-level effect in terms of lower-level ones can be rewritten (or replaced) to use a different set of lower-level effects without affecting clients (more on this in Section~\ref{sec-information-hiding}).


\subsection{Effect Aggregation}
\label{sec:effect-aggregation}

Wyvern's effect-system design allows reducing the effect-annotation overhead by aggregating several effects into one. For example, if, to update the log file, the \li{logger} module needed to first read the file and then write it back, the \li{UpdateLog} effect would consist of two effects: a file read and a file write. In other effect systems, this change may make effects more verbose since all the methods that call the \li{updateLog} method would need to be annotated with the two effects. However, effect aggregation allows us to define the \li{UpdateLog} effect to be the two effects and then use \li{UpdateLog} to annotate the \li{updateLog} method and all methods that call it:

\begin{minipage}{\linewidth}
\begin{lstlisting}[xleftmargin=-5pt, numbers=none]
module def logger(f: File): Logger
effect UpdateLog = {f.Read, f.Write}
def updateLog(newEntry: String): {this.UpdateLog} Unit
...
\end{lstlisting}
\end{minipage}
This way we need to use only one effect, \li{UpdateLog}, instead of two, in method effect annotations, thus reducing the effect-annotation overhead. Because more code may add more effects, larger software systems might experience a snowballing of effects, when method annotations have numerous effects in them.


\subsection{Controlling FFI Effects}
Wyvern programs access system resources via calls to other programming languages, such as Java and Python, i.e., through a foreign function interface (FFI). To monitor and control the effects caused by FFI calls, we enforce that all functions from other programming languages, when called within Wyvern, are annotated with the \li{system.FFI} effect. 

As was mentioned in Section~\ref{sec:effect-abstraction}, the \li{system.FFI} effect is an effect that describes function calls though an FFI. Since every function call though FFI has this effect, the access to system resources via FFI is guaranteed to be monitored. \li{system.FFI} is the lowest-level effect in the effect system which can be used to build other higher-level effects. The programmer can lift \li{system.FFI} to higher-level effects and reason about those higher-level effects instead.


For example, Wyvern’s import mechanism works by loading an object in a static field of a Java class, and the following code imports a field of a Java class that helps to implement file IO:\\
\begin{minipage}{\linewidth}
\begin{lstlisting}[xleftmargin=-5pt, numbers=none]
  import java:wyvern.stdlib.support.FileIO.file
\end{lstlisting}
\end{minipage}

The file object is itself of type FileIO. And FileIO has this method, among others: \\
\begin{minipage}{\linewidth}
\begin{lstlisting}[xleftmargin=3pt, numbers=none]
public void writeStringIntoFile(String content, String filename) throws IOException { ... }
\end{lstlisting}
\end{minipage}

In Wyvern, there is a type wyvern.stdlib.support.FileIO as well as an object file (of that type) that gets added to the scope as a result of the import above. The type has the following member, corresponding to the method above: \\
\begin{minipage}{\linewidth}
\begin{lstlisting}[xleftmargin=-5pt, numbers=none]
  def writeStringIntoFile(content:String, filename:String): { system.FFI } Unit 
\end{lstlisting}
\end{minipage}

Here, the system.FFI effect was added to the signature because this is a function that was imported via the FFI. The Wyvern file library that uses the \li{writeStringIntoFile} function abstracts this \li{system.FFI} effect into a library-specific \li{FileIO.Write} effect.


\subsection{The Limitation of  Abstract Effects}

In the current effect system, effects can be declared as members of objects. There are two possible declaration type for an effect member: An effect can be declared abstractly inside a type: In this example, effect Read is defined as a member of type Logger, the effect Read can be accessed as \texttt{v.Read} if \texttt{v} has the type \texttt{Logger}
\begin{verbatim}
    type Logger
        effect Read
\end{verbatim}
On the other hand, effect can be also defined as a effect set that contains other effects. In the following example, any expression of type \texttt{Logger} will have a effect Read that is equivalent to \texttt{file.Read}
\begin{verbatim}
    type Logger
        effect Read = { file.Read }
\end{verbatim}
The former type definition for Logger is useful when the programmer wants to hide the definition of the effect Read, while the second type allows the Read effect and \texttt{file.Read} to be used interchangeably. However, there is no way for a programmer to have these two benefits at once: the effect Read in Logger is either completely opaque, or completely equivalent to some other effect set.





\section{Algebraic Effects and Handlers}

\subsection{Overview}
  Algebraic effects and handlers~\cite{plotkin03,plotkin09} are a way of implementing certain kinds of side effects such as exceptions and mutable state in an otherwise purely functional setting.  As described above, algebraic effects fall into the ``denotational'' rather than ``descriptive'' family of work on effects. 
  
Algebraic effects introduce the notion of operations, which carry no predefined meaning. The interpretation for each operation is provided by the evaluation context. In most systems with algebraic effects and handlers, operations are tracked by an effect system, which is similar to the way effectful methods are tracked in the Wyvern programming language. Comparing to restrictive effect systems, algebraic effects subsumes multiple control-flow constructs, and can restore the purity of programs by handling effect operations.

Exceptions are an example of control-flow constructs that are subsumed by algebraic effects. Assume that an algebraic effect \li{exc} is defined with one operation \li{raise}:
\begin{verbatim}
effect exc = {
  raise : String -> a
}
\end{verbatim}
Then the operation \li{raise} can be used to implement programs that might cause exceptions. For example, we can write a division function that raise an exception when the divisor is 0.
\begin{verbatim}
def div(x : Int, y : Int) : {exc} Int
  if (y == 0)
    raise("divide by zero")
  else
    x / y
\end{verbatim}

The type of function \li{div} is $Int \times Int \rightarrow \{exc\}\ Int$. The $\{exc\}$ annotation indicates that the function might cause the $exc$ effect by invoking its operation. Because the meaning of operations is undefined, we need to use a handler to define semantics for effect operations. For this example, we define a handler that returns 0 whenever an error is rased

\begin{verbatim}
	handle 
	  div(1, 0)
	with 
	| return x -> x
	| raise s -> 0
\end{verbatim}

In the return clause, the variable $x$ is bound to the result of the computation if no operation is invoked. So in this case the whole handled expression would evaluate to $x$. In the \li{raise} clause, the handler decides to evaluate the expression to 0 when an exception is raised. 

Another useful property of algebraic effects and handlers is the ability to resume the computation, consider the following handled expression that invokes the function \li{div} twice.

\begin{verbatim}
	handle 
	  div(1, 0) + div(2, 1)
	with 
	| return x -> x
	| raise s -> resume 3
\end{verbatim}

The \li{raise} clause of the handler is \li{resume\ 3} , which tells the program to continue evaluate but use 3 as the result of the operation. So the final result for the whole expression is 5. This example shows algebraic effects are a more structured form of delimited continuations. 

\subsection{Effect Abstraction}
Similar to the restrictive effect system, algebraic effects also benefit from the ability to declare abstract effects. However, this problem is not studied as thoroughly as the effect abstraction in the restrictive setting. The abstraction for algebraic effects was originally proposed by \citet{leijen18}, but was not developed theoretically. The work by \citet{biernacki19} first studied the abstract algebraic effects formally, and proposed a formalization based on coercions. The relationship between \citet{biernacki}'s work and ours is discussed more in section \ref{related-abstract}








