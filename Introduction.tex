% !TEX root = thesis.tex

\chapter{Introduction}


Effect systems have been a subject of active research for nearly four decades, with the most notable practical example being checked exceptions in programming languages such as Java. According to \citet{filinski10}, the are two different views on modeling computational effects in programs: The denotational approach and the restrictive approach.

The denotational approach describes how effectful programs can be translated into a pure program, which can then be evaluated using the standard semantics for pure programs. Works by \citet{moggi89} shows that features from imperative computations, such as exceptions or mutable states, can be mimicked by monad in a pure program. Algebraic effects and handlers \cite{plotkin09} are the the latest development in this strand of work. Algebraic effects and handlers can express a wide range of computation effects such as nondeterminism, concurrency, state, and input/output \cite{plotkin09}. Comparing to the tradition approach that uses general monads, algebraic effects have the advantage of being freely composable. Therefore, algebraic effects are recently gaining popularity as an approach to model effects in the purely functional setting.

Alternatively, in a restrictive setting of computational effects, effects are considered to be built into the language. Rather than building up new behaviors, the effect systems aim to classify and restrict the use of existing effectful behavior in a language, such as reads and writes to memory, as well as checked exceptions. Restrictive effect systems are widely used for reasoning about security~\cite{turbak08}, memory effects~\cite{lucassen88}, and concurrency~\cite{bocchino09,bracevac18,dolan17}. 

\noindent\textbf{Abstraction: A requirement for scalable effect systems}

 Unfortunately, effect systems have not been widely adopted, other than checked exceptions in Java, a feature that is widely viewed as problematic~\cite{10.1145/1103845.1094847}.  The root of the problem is that existing effect systems do not provide adequate support for scaling to programs that are larger and have complex structure.  Any adequate solution must support \textit{effect abstraction} and \textit{effect composition}.

Abstraction is key to achieving scale in general, and a principal form of abstraction is abstract types~\cite{10.1145/44501.45065}. 
There are many existing works that achieves information hiding using abstract types, such as SML signatures and abstract type members in Scala~\cite{odersky05}. 
Typically, a module system allows each module to choose what names and entities to export, and what to keep hidden. The exported interface typically does not reveal details of the implementation of a module. By hiding the implementation details, the programmer of the module can be certain that the invariants within the module cannot be broken by the client. Analogously to type abstraction, we define \textit{effect abstraction} as the ability to define higher-level effects in terms of lower-level effects, and potentially to \textit{hide} that definition from clients of effects.  

In large-scale systems, abstraction should be \textit{composable}.  For example, a database component might abstract \li{file.Read} further, exposing it as a higher-level \li{db.Query} effect to clients.  Clients of the database should be oblivious to whether \li{db.Query} is implemented in terms of a \li{file.Read} effect or a \li{network.Access} effect (in the case that the backend is a remote database).

\noindent\textbf{Design of a restrictive effect system in Wyvern}

This thesis presents a novel and scalable effect-system design that supports bounded effect abstraction that extends the effect system presented by \citet{melicher20}.  The abstraction facility of our effect-system is inspired by type members in languages such as Scala. Just as Scala objects may define type members, in our effect calculus, any object may define one or more \textit{effect members}.  An effect member defines a new effect in terms of the lower-level effects that are used to implement it.  The set of lower-level effects may be empty in the base case or may include low-level effects that are hard-coded in the system.  Type ascription can enable information hiding by concealing the definition of an effect member from the containing object's clients. In addition to completely concealing the definition of an effect, our calculus provides bounded abstraction, which exposes upper or lower bounds of the definition of an effect, while still hiding the definition of it. 


\textit{Effect polymorphism} is a form of parametric polymorphism that allows functions or types to be implemented generically for handling computations with different effects~\cite{lucassen88}. In systems at a larger scale, there are various possible effects, and each program component may cause different effects. With effect polymorphism, we can write general code that handles objects with different effects, thereby reducing the amount of replicated code. In practice, we have found that to make effects work well with modules, it is essential to extend effect polymorphism by assigning bounds to effect parameters. We therefore introduce \textit{bounded abstract effects}, which allows programmers to define upper and lower bounds both on abstract effects and on polymorphic effect parameters.

 Just as Scala's type members can be used to encode parametric polymorphism over types, our effect members and their bounds double as a way to provide bounded effect polymorphism. We follow numerous prior Scala formalisms in including polymorphism via this encoding rather than explicitly; this keeps the formal system simpler without giving up expressive power. 


\noindent\textbf{Design of a denotational effect system with effect abstraction}  

This thesis presents a core calculus that supports algebraic effects. The calculus extends simply typed lambda calculus with algebraic effect operations and handlers and provides the ability to define abstract algebraic effects. Similar to the restrictive effect system, algebraic effects types can be defined in terms of lower-level effects. Effect abstraction in this system ensures that the client of an abstract effect type is not aware of the lower-level effects that implements the abstract effect. Consequently, the client of an abstract effect type would not be able to handle the computation that causes the abstract effect, whose operations are hidden.

Different from the restrictive effect system in Wyvern, which describes the built in effectful behavior in the language and does not affect the dynamic semantics of the program, the dynamic semantics of algebraic effects operations depend on the handler that encapsulates it during evaluation.  Therefore, the effect system needs to ensure the abstraction does not break during the evaluation of a program. This problem was originally discovered by \citet{biernacki19}, who solved the problem using the technique of coercions. In this paper, we propose the technique of agent-based reasoning, which was originally designed by \citet{grossman00}, as a solution of the problem. The benefit of this approach is that by explicitly dividing modules with hidden information into agents, the system supports syntactic proof for effect-abstraction properties

\noindent\textbf{Outline and Contributions.}  
Chapter \ref{chapter-background} introduces the background for both restrictive and denotational effect systems, and discusses the basics of the Wyvern effect system, after which we describe the main contributions of our paper:
\begin{itemize}
\item A design of a more expressive effect system for Wyvern. Specifically, ours is the first system to provide the programmer with a general form of bounded effect polymorphism and bounded effect abstraction, supporting upper and lower bounds that are other arbitrary effects.  (Section~\ref{sec-bound});
\item A precise, formal description of our effect system, and proof of its soundness.  Our formal system shows how to generalize and enrich earlier work on path-dependent effects by leveraging the type theory of DOT.(Section~\ref{sec-form});
\item A multi-agent calculus  that supports abstraction for algebraic effects, and proof of its type soundness theorems;  Our system enables a syntactic proof of the effect-abstraction property. (Section~\ref{sec-core});
\item A multi-agent calculus extended with existential effect types that demonstrates how multi-agent calculus interacts with traditional techniques of type abstraction. (Section~\ref{sec-exist});
\end{itemize}
The last chapter in the thesis discusses related work and concludes.
