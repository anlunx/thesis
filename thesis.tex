%for a more compact document, add the option openany to avoid
%starting all chapters on odd numbered pages
\documentclass[12pt]{cmuthesis}

% This is a template for a CMU thesis.  It is 18 pages without any content :-)
% The source for this is pulled from a variety of sources and people.
% Here's a partial list of people who may or may have not contributed:
%
%        bnoble   = Brian Noble
%        caruana  = Rich Caruana
%        colohan  = Chris Colohan
%        jab      = Justin Boyan
%        josullvn = Joseph O'Sullivan
%        jrs      = Jonathan Shewchuk
%        kosak    = Corey Kosak
%        mjz      = Matt Zekauskas (mattz@cs)
%        pdinda   = Peter Dinda
%        pfr      = Patrick Riley
%        dkoes = David Koes (me)

% My main contribution is putting everything into a single class files and small
% template since I prefer this to some complicated sprawling directory tree with
% makefiles.

% some useful packages
\usepackage{times}
\usepackage{fullpage}
\usepackage{graphicx}
\usepackage{amsmath}
\usepackage[numbers,sort]{natbib}
\usepackage[backref,pageanchor=true,plainpages=false, pdfpagelabels, bookmarks,bookmarksnumbered,
%pdfborder=0 0 0,  %removes outlines around hyper links in online display
]{hyperref}
\usepackage{subfigure}
\usepackage{listings}
\usepackage{setspace}

% Approximately 1" margins, more space on binding side
%\usepackage[letterpaper,twoside,vscale=.8,hscale=.75,nomarginpar]{geometry}
%for general printing (not binding)
\usepackage[letterpaper,twoside,vscale=.8,hscale=.75,nomarginpar,hmarginratio=1:1]{geometry}


\newcommand{\li}[1]{\lstinline{#1}}


% Provides a draft mark at the top of the document. 
\draftstamp{\today}{DRAFT}

\begin {document} 
\frontmatter

%initialize page style, so contents come out right (see bot) -mjz
\pagestyle{empty}

\title{ %% {\it \huge Thesis Proposal}\\
{\bf Extending Abstract Effects with Bounds and Algebraic Handlers}}
\author{Anlun Xu}
\date{November 2020}
\Year{2020}
\trnumber{}

\committee{
\ 
}

\support{}
\disclaimer{}

% copyright notice generated automatically from Year and author.
% permission added if \permission{} given.

\keywords{Effect Systems}

\maketitle

\begin{dedication}
For my dog
\end{dedication}

\pagestyle{plain} % for toc, was empty


\begin{abstract}
Effect systems have been a subject of active research for nearly four decades, with the most notable practical example being checked exceptions in programming languages such as Java. While many exception systems support abstraction, aggregation, and hierarchy (e.g., via class declaration and subclassing mechanisms), it is rare to see such expressive power in more generic effect systems. We designed an effect system around the idea of protecting system resources and incorporated our effect system into the Wyvern programming language. Similar to type members, a Wyvern object can have effect members that can abstract lower-level effects, allow for aggregation, and have both lower and upper bounds, providing for a granular effect hierarchy. We argue that Wyvern's effects capture the right balance of expressiveness and power from the programming language design perspective. We present a full formalization of our effect-system design, show that it allows reasoning about authority and attenuation.  Our approach is evaluated through a security-related case study.\end{abstract}
\begin{acknowledgments}
My advisor is cool.
\end{acknowledgments}



\tableofcontents
\listoffigures
\listoftables

\mainmatter

%% Double space document for easy review:
%\renewcommand{\baselinestretch}{1.66}\normalsize

% The other requirements Catherine has:
%
%  - avoid large margins.  She wants the thesis to use fewer pages, 
%    especially if it requires colour printing.
%
%  - The thesis should be formatted for double-sided printing.  This
%    means that all chapters, acknowledgements, table of contents, etc.
%    should start on odd numbered (right facing) pages.
%
%  - You need to use the department standard tech report title page.  I
%    have tried to ensure that the title page here conforms to this
%    standard.
%
%  - Use a nice serif font, such as Times Roman.  Sans serif looks bad.
%
% Other than that, just make it look good...
\doublespacing

\chapter{Introduction}

An effect system can be used to reason about the side effects of code, such as reads and writes to memory, exceptions, and I/O operations.  Java's checked exceptions is a simple effect system that has found widespread use, and interest is growing in effect systems for reasoning about security~\cite{turbak08}, memory effects~\cite{lucassen88}, and concurrency~\cite{bocchino09,bracevac18,dolan17}.

\noindent\textbf{Requirements for a scalable effect system} Unfortunately, effect systems have not been widely adopted, other than checked exceptions in Java, a feature that is widely viewed as problematic~\cite{10.1145/1103845.1094847}.  The root of the problem is that existing effect systems do not provide adequate support for scaling to programs that are larger and have complex structure.  Any adequate solution must support \textit{effect abstraction}, \textit{effect composition}, and \textit{path-dependent effects}. Furthermore, effects should be an inherent part of the type system, instead of being encoding of other type abstractions such as monad.

Abstraction is key to achieving scale in general, and a principal form of abstraction is abstract types~\cite{10.1145/44501.45065}, a modern form of which appears as abstract type members in Scala~\cite{odersky05}. Analogously to type abstraction, we define \textit{effect abstraction} as the ability to define higher-level effects in terms of lower-level effects, and potentially to \textit{hide} that definition from clients of an abstraction.  In order to integrate with modularity mechanisms, and by analogy to type members, we define effects using \textit{effect members} of modules or objects.  For example, a \li{file.Read} effect could abstract a lower-level \li{system.FFI} effect. Then clients of a file should be able to reason about side effects in terms of file reads and writes, not in terms of the low-level calls that are made to the foreign function interface (FFI).  In large-scale systems, abstraction should be \textit{composable}.  For example, a database component might abstract \li{file.Read} further, exposing it as a higher-level \li{db.Query} effect to clients.  Clients of the database should be oblivious to whether \li{db.Query} is implemented in terms of a \li{file.Read} effect or a \li{network.Access} effect (in the case that the backend is a remote database).

\textit{Effect polymorphism} is a form of parametric polymorphism that allows functions or types to be implemented generically for handling computations with different effects~\cite{lucassen88}. In systems at a larger scale, there are various possible effects, and each program component may cause different effects. With effect polymorphism, we can write general code that handles objects with different effects, thereby reducing the amount of replicated code. In practice, we have found that to make effects work well with modules, it is essential to extend effect polymorphism by assigning bounds to effect parameters. We therefore introduce \textit{bounded abstract effects}, which allows programmers to define upper and lower bounds both on abstract effects and on polymorphic effect parameters.


We also leverage \textit{path-dependent effects}, i.e., effects whose definitions depend on an object. This adds expressiveness; for example, if we have two \li{File} objects, \li{x} and \li{y}, we can distinguish effects on one file from effects on the other: the effects \li{x.Read} and \li{y.Read} are distinct.  Path-dependent effects are particularly important in the context of modules, where two different modules may implement the same abstract effect in different ways.  For example, it may be important to distinguish \li{db1.Query} from \li{db2.Query} if \li{db1} is an interface to a database stored in the local file system whereas \li{db2} is a database accessed over the network.

Effects should be an inherent part of the type system, instead of being encoded as a type abstraction. The reason is twofold: although effect checking may be implemented in terms of monad, the safety of the effect system is compromised. The effectful code can bypass the effect-checking process if the programmer use the effectful code in a context which is not encapsulated by a monad. The other reason is that monad could be unintuitive to use for programmers outside of the functional programming community, while a type system enhanced by effect types doesn't require prior knowledge of  monad and is straightforward to use.

\noindent\textbf{Design of the effect system in Wyvern}
This paper presents a novel and scalable effect-system design that supports effect abstraction and composition. The abstraction facility of our effect-system is inspired by type members in languages such as Scala. Just as Scala objects may define type members, in our effect calculus, any object may define one or more \textit{effect members}.  An effect member defines a new effect in terms of the lower-level effects that are used to implement it.  The set of lower-level effects may be empty in the base case or may include low-level effects that are hard-coded in the system.  Type ascription can enable information hiding by concealing the definition of an effect member from the containing object's clients. In addition to completely concealing the definition of an effect, our calculus provides bounded abstraction, which exposes upper or lower bounds of the definition of an effect, while still hiding the definition of it. 





 Just as Scala's type members can be used to encode parametric polymorphism over types, our effect members double as a way to provide effect polymorphism. Bounded effect polymorphism is also provided in our system, because abstract effect members can be bounded by upper or lower bounds. We follow numerous prior Scala formalisms in including polymorphism via this encoding rather than explicitly; this keeps the formal system simpler without giving up expressive power.

Finally, because effect members are defined on objects, our effects are \textit{generative}, even dynamically.  This yields great expressivity: each object created at runtime defines a new effect for each effect member in that object so that, for example, we can separately track effects on different \li{File} objects, statically distinguishing the effects on one object from the effects on another.

%Aside from the properties that makes our effect-system usable at scale, we are also interested in capturing the sensitive system resources as an application of the effect system. We enforce the methods that call through FFI have the \li{system.FFI} effect. So the low-level system resources is guaranteed to be tracked in our effect-system. Moreover, the design of effect abstraction in our system helps programmers to build higher-level effects on top of the low-level effects, so the access to system resources is controlled when building larger software.

\noindent\textbf{Evaluation and Security Applications.}  A promising area of application for effects is software security.  For example, in the setting of mobile code, \cite{turbak08} proposed that effects could be used to ensure that any untrusted code we download can only access the system resources it needs to do its tasks, thus following the principle of least privilege~\cite{denning76}.  We are not aware of prior work that explores this idea in depth.

In order to evaluate our design for effect abstraction, we have incorporated it into an effect system that tracks the use of system resources such as the file system, network, and keyboard.  Our effect system is intended to help developers reason about which source code modules use these resources.  Through the use of abstraction, we can ``lift'' low-level resources such as the file system into higher-level resources such as a logging facility or a database and enable application code to reason in terms of effects on those higher-level resources when appropriate.  In fact, even the use of resources such as the file system is scaffolded as an abstraction on top of a primitive \li{system.FFI} effect that our system attaches to uses of the language's foreign function interface.  A set of illustrative examples demonstrates the benefits of abstraction for effect aggregation, as well as for information hiding and software evolution.  Finally, we show how our effect system allows us to reason about the \textit{authority}~\cite{miller06} of code, i.e., what effects a component can have, as well as the \textit{attenuation} of that authority.

Our effect system is implemented in the context of Wyvern, a programming language designed for highly productive development of secure software systems.  In this paper, we give several concrete examples of how our effect-system design can be used in software production, all of which are functional Wyvern code that runs in the Wyvern regression test suite.

\noindent\textbf{Outline and Contributions.}  The next section introduces a running example, after which we describe the main contributions of our paper:
\begin{itemize}
\item The design of a novel effect system fulfilling the requirements above. Our system is the first to bring together effect abstraction and composition with the effect member construct.  Ours is also the first system to provide the programmer with a general form of bounded effect polymorphism and bounded effect abstraction, supporting upper and lower bounds that are other arbitrary effects.  (Section~\ref{sec:wyvern-effects-basics});
\item The application of our effect system to a number of forms of security reasoning, illustrating its expressiveness and making the benefits described above concrete (Section~\ref{sec:patterns}); 
\item A precise, formal description of our effect system, and proof of its soundness.  Our formal system shows how to generalize and enrich earlier work on path-dependent effects by leveraging the type theory of DOT (Section~\ref{sec:formalization});
\item A formalization of authority using effects, and of authority attenuation (Section~\ref{authority});
\item A feasibility demonstration, via the implementation of our approach in the Wyvern programming language (Section~\ref{sec:case-study}).
\end{itemize}
The last sections in the paper discuss related work and conclude.

\chapter{Conclusion}

%\appendix
%% !TEX root = thesis.tex
\pagebreak


\appendix
\chapter{Transitivity of Subtyping}
\section{Lemmas}
\begin{lemma}
If $\Gamma, x : \tau \vdash \varepsilon_1 <: \varepsilon_2$, and $\Gamma \vdash \tau' <: \tau$, then $\Gamma, x : \tau' \vdash \varepsilon_1 <: \varepsilon_2$.
\begin{proof}
The proof is by structural induction on the rule to derive $\Gamma, x : \tau \vdash \varepsilon_1 <: \varepsilon_2$. \begin{enumerate}
\item Subeffect-Subset\\
Since the premise doesn't rely on the context. This case is trivially true.
\item Subeffect-Upperbound\\
If the type of $n$ is not changed, then we can apply the same rule to to derive $\Gamma, x : \tau' \vdash \varepsilon_1 \cup \{n.g\}<: \varepsilon_2$. If the type of $n$ is replaced by $\tau'$, then we have $\keyw{effect}\ g \leqslant$ $\varepsilon' \in \sigma$, where $\Gamma, n: \tau' \vdash \varepsilon' <: \varepsilon$. By IH, we have $\Gamma, n : \tau' \vdash [n/y]\varepsilon \cup \varepsilon_1 <: \varepsilon_2$. By transitivity of subeffecting, we have $\Gamma, n : \tau' \vdash [n/y]\varepsilon' \cup \varepsilon_1 <: \varepsilon_2$. Then we can apply Subeffect-Upperbound again to derive $\Gamma, x : \tau' \vdash \varepsilon_1 \cup \{n.g\}<: \varepsilon_2$.
\item Subeffect-Lowerbound\\ This case is similar to Subeffect-Upperbound
\item Subeffect-Def-1\\ Since the declaration type $\keyw{effect} g = \{\varepsilon\}$ is not changed, the result follows directly by induction hypothesis.
\item Subeffect-Def-2\\ Since the declaration type $\keyw{effect} g = \{\varepsilon\}$ is not changed, the result follows directly by induction hypothesis.
\end{enumerate}
\end{proof}
\label{lemma-context-effect}
\end{lemma}

\begin{lemma}
If $\Gamma, x:\tau \vdash \tau_1 <: \tau_2$, and $\Gamma \vdash \tau' <: \tau$, then $\Gamma, x:\tau' \vdash \tau_1 <: \tau_2$\\
If $\Gamma, x:\tau \vdash \sigma_1 <: \sigma_2$, and $\Gamma \vdash \tau' <: \tau$, then $\Gamma, x:\tau' \vdash \sigma_1 <: \sigma_1$

\begin{proof}
We induct on the number of S-Alg used to derive the typing judgment in the premise of the statement.
\begin{itemize}
\item[BC] S-Alg is not used, so we have $\Gamma, x:\tau \vdash \sigma_1 <: \sigma_2$ derived by S-Refl2 or one of the S-Effect rules. The proof is trivial if we apply lemma \ref{lemma-context-effect}.
\item[IS1] Assume we used S-Alg n times to derive $\Gamma, x:\tau \vdash \{ y \Rightarrow \sigma_i^{i\in1\dots m}\} <: \Gamma \vdash \{y \Rightarrow {\sigma'}_i^{i\in1\dots n}\}$. Then for each subtyping judgments in the premise of S-Alg, we can apply induction hypothesis to derive $\Gamma,~x:\tau',~y : \{ y \Rightarrow {\sigma}_i^{i \in 1..m} \} \vdash \sigma_{p(i)}<: \sigma_i'$. Then by applying S-Alg, we have $\Gamma, x:\tau' \vdash \{ y \Rightarrow \sigma_i^{i\in1\dots m}\} <: \Gamma \vdash \{y \Rightarrow {\sigma'}_i^{i\in1\dots n}\}$
\item[IS2] Assume we used S-Alg n times to derive $\Gamma, y:\tau \vdash \keyw{def} m(x : \tau_1) : \{ \varepsilon_1 \}~\tau_2 <: \keyw{def} m(x : \tau_1') : \{ \varepsilon_2 \}~\tau_2'$, by inversion on S-Def, we have $\Gamma, y:\tau \vdash \tau_1' <: \tau_1$, $\Gamma, y:\tau \vdash \tau_2 <: \tau_2'$, and $\Gamma, y:\tau, x:\tau_1 \vdash \varepsilon_1 <: \varepsilon_2$. Then by induction hypothesis and lemma \ref{lemma-context-effect}, we have $\Gamma, y:\tau' \vdash \tau_1' <: \tau_1$, $\Gamma, y:\tau' \vdash \tau_2 <: \tau_2'$, and $\Gamma, y:\tau', x:\tau_1 \vdash \varepsilon_1 <: \varepsilon_2$. Then we use S-Def to derive $  \Gamma, y:\tau' \vdash \keyw{def} m(x : \tau_1) : \{ \varepsilon_1 \}~\tau_2 <: \keyw{def} m(x : \tau_1') : \{ \varepsilon_2 \}~\tau_2'$
\end{itemize}
\end{proof}
\label{lemma-context-type}
\end{lemma}



\section{Proof of Theorem \ref{theorem-transitivity}}
If $\Gamma \vdash \tau_1 <: \tau_2$ and $\Gamma \vdash \tau_2 <: \tau_3$, then $\Gamma \vdash \tau_1 <: \tau_3$. \\
If $\Gamma \vdash \sigma_1 <: \sigma_2$ and $\Gamma \vdash \sigma_2 <: \sigma_3$, then $\Gamma \vdash \sigma_1 <: \sigma_3$. 
\begin{proof}
We induct on the the number of S-Alg used to derive the two judgments in the premise of the first statement: $\Gamma \vdash \tau_1 <: \tau_2$ and $\Gamma \vdash \tau_2 <: \tau_3$, or the two judgments in the premise of the second statement: $\Gamma \vdash \sigma_1 <: \sigma_2$ and $\Gamma \vdash \sigma_2 <: \sigma_3$.
\begin{itemize}
\item[BC] The S-Alg is not used, so we have $\Gamma \vdash \sigma_1 <: \sigma_2$ and $\Gamma \vdash \sigma_2 <: \sigma_3$ by S-Refl2 or one of S-Effect.  By lemma \ref{lemma-trans-effect} transitivity of subeffecting, it is easy to see $\Gamma \vdash \sigma_1 <: \sigma_3$

\item[IS1] Assume we used S-Alg $n$ times to derive $\Gamma \vdash \{x \Rightarrow \sigma_i^{i\in1...m}\} <:  \{x \Rightarrow {\sigma'}_i^{i\in1...n}\} $ and \mbox{$\Gamma \vdash \{x \Rightarrow {\sigma'}_i^{i\in1...n}\} <:  \{x \Rightarrow {\sigma''}_i^{i\in1...k}\}$}. By inversion of S-Alg, there is an injection $p: \{1..n\} \mapsto  \{1..m\}$ such that $\forall i \in 1..n,\ \Gamma, x: \{x \Rightarrow {\sigma}_i^{i\in 1..m} \} \vdash \sigma_{p(i)} <: \sigma'_i$. There is another injection $q: \{1..k\} \mapsto \{1..n\}$ such that $\forall i \in 1..k,\ \Gamma, x: \{x \Rightarrow {\sigma'}_i^{i\in 1..n} \} \vdash \sigma'_{q(i)} <: \sigma''_i$. So for each $i \in 1..k$ we have two judgments 
\begin{align*}
\Gamma, x: \{x \Rightarrow {\sigma}_i^{i\in 1..m} \} &\vdash \sigma_{p(q(i))} <: \sigma'_{q(i)}\\
\Gamma, x: \{x \Rightarrow {\sigma'}_i^{i\in 1..n} \} &\vdash \sigma'_{q(i)} <: \sigma''_i
\end{align*}
By lemma \ref{lemma-context-type}, we can write the second judgment as $\Gamma, x: \{x \Rightarrow {\sigma}_i^{i\in 1..m} \} \vdash \sigma'_{q(i)} <: \sigma''_{i}$. By IH, for all $i \in 1..k$, $ \Gamma, x: \{x \Rightarrow {\sigma''}_i^{i\in 1..k} \} \vdash \sigma_{p(q(i))} <: \sigma''_i$. Since the  function $p \circ q$ is a bijection from $\{1..k\} \mapsto \{1..n\}$, we can use the rule S-Alg again to derive $\Gamma \vdash \{x \Rightarrow \sigma_i^{i\in1...m}\} <:  \{x \Rightarrow {\sigma''}_i^{i\in1...k}\} $ 
\item[IS2] Assume we used S-Alg $n$ times to derive 
$\Gamma \vdash \keyw{def} m(x : \tau_1) : \{ \varepsilon_1 \}~\tau_1' <: \keyw{def} m(x : \tau_2) : \{ \varepsilon_2 \}~\tau_2'$
and
$\Gamma \vdash \keyw{def} m(x : \tau_2) : \{ \varepsilon_2 \}~\tau_2' <: \keyw{def} m(x : \tau_3) : \{ \varepsilon_3 \}~\tau_3'$. By inverse on S-Def, we have $\Gamma \vdash \tau_2 <: \tau_1$, $\Gamma \vdash \tau_3 <: \tau_2$, $\Gamma \vdash \tau_1' <: \tau_2'$, and $\Gamma \vdash \tau_2' <: \tau_3'$. By IH, we have $\Gamma \vdash \tau_1' <: \tau_3'$ and $\Gamma \vdash \tau_3 <: \tau_1$. We have $\Gamma \vdash \varepsilon_1 <: \varepsilon_3$ by transitivity of subeffects. Hence we can use S-Def again to derive $\Gamma \vdash \keyw{def} m(x : \tau_1) : \{ \varepsilon_1 \}~\tau_1' <: \keyw{def} m(x : \tau_3) : \{ \varepsilon_3 \}~\tau_3'$. 
\item[IS3] By transitivity of subeffecting, other cases for $\Gamma \vdash \sigma_1 <: \sigma_3$ are trivial. 

\end{itemize}
\end{proof}


\chapter{Proofs of the Type Soundness Theorems for Bounded Abstract Effects}
\label{app-effects-type-soundness}

\section{Lemmas}

%\begin{lemma}[Permutation]
%If \mbox{$\Gamma~|~\varnothing \vdash e : \{ \varepsilon \}~\tau$} and $\Delta$ is a permutation of $\Gamma$, then\linebreak
%\mbox{$\Delta~|~\varnothing \vdash e : \{ \varepsilon \}~\tau$}, and the latter derivation has the same depth as the former.
%\end{lemma}

\begin{proof}
Straightforward induction on typing derivations.
\end{proof}


\begin{lemma}[Weakening]
If $\Gamma~|~\varnothing \vdash e : \{ \varepsilon \}~\tau$ and $x \not\in dom(\Gamma)$, then $\Gamma,~x : \tau'~|~\varnothing \vdash e : \{ \varepsilon \}~\tau$, and the latter derivation has the same depth as the former.
\end{lemma}

\begin{proof}
Straightforward induction on typing derivations.
\end{proof}


\sloppy
\begin{lemma}[Reverse of \textsc{Subeffecting-Lowerbound}] \label{lemma-reverse1}
If $\Gamma \vdash \varepsilon_1 <: \varepsilon_2 \cup \{x.g\}$
, $\Gamma \vdash x : \{y \Rightarrow \sigma\}$, and
$\keyw{effect}\ g \leqslant \varepsilon \in \sigma$
then 
$\Gamma \vdash \varepsilon_1 <: \varepsilon_2 \cup [x/y]\varepsilon$
\end{lemma}
\begin{proof}
We prove this by induction on $size(\varepsilon_1 \cup \varepsilon_2 \cup \{x.g\})$, which is defined in Fig. \ref{f-size}
\begin{enumerate}
    \item[BC] If $size(\varepsilon_1 \cup \varepsilon_2 \cup \{x.g\}) = 0$. Then $x.g$ can not have a definition. This case is vacuously true.
    \item[IS] We case on the rule used to derive  $\Gamma \vdash \varepsilon_1  <: \varepsilon_2 \cup \{x.g\}$:
    \begin{enumerate}
        \item  $\Gamma \vdash \varepsilon_1  <: \varepsilon_2 \cup \{x.g\}$ is derived by Subeffect-Subset: If $x.g \not\in \varepsilon_1$, then we can use Subeffect-Subset to show 
        $\Gamma \vdash \varepsilon_1  <: \varepsilon_2 \cup [x/y]\varepsilon$
        If $x.g \in \varepsilon_1$. Then $\varepsilon_1 = \varepsilon_1' \cup \{x.g\}$, where $\varepsilon_1' \subseteq \varepsilon_2$. So we can use Subeffect-Def-1 to show
        $\Gamma \vdash \varepsilon_1' \cup \{x.g\}  <: \varepsilon_2 \cup [x/y]\varepsilon$
        \item $\Gamma \vdash \varepsilon_1  <: \varepsilon_2 \cup \{x.g\}$ is derived by Subeffect-Upperbound:\\
        Then we have 
        $\varepsilon_1 = \varepsilon_1' \cup \{z.h\}$,
        $\Gamma \vdash z : \{y' \Rightarrow\sigma\}$, 
        $\keyw{effect}\ h = \{ \varepsilon'\} \in \sigma$,
        and 
        \mbox{$\Gamma \vdash \varepsilon_1' \cup [z/y']\varepsilon' <: \varepsilon_2 \cup \{x.g\}$}
        By IH, we have 
        $\Gamma \vdash \varepsilon_1' \cup [z/y']\varepsilon' <: \varepsilon_2 \cup [x/y]\varepsilon$
        Using Subeffect-Upperbound, we have 
        $\Gamma \vdash \varepsilon_1' \cup \{z.h\} <: \varepsilon_2 \cup [x/y]\varepsilon$
        \item $\Gamma \vdash \varepsilon_1  <: \varepsilon_2 \cup \{x.g\}$ is derived by Subeffect-Def-1:\\
        If Subeffect-Def-1 uses the effect ${x.g}$, then we immediately have
        $\Gamma \vdash \varepsilon_1 <: \varepsilon_2 \cup [x/y]\varepsilon$
        Otherwise, if Subeffect-Def-1 doesn't use $x.g$, then we have 
        \mbox{$\varepsilon_2 = \varepsilon_2' \cup \{z.h\}$},
        \mbox{$\Gamma \vdash z : \{y' \Rightarrow\sigma\}$},
        \mbox{$\keyw{effect}\ h = \{ \varepsilon'\} \in \sigma$},
        and 
        \mbox{$\Gamma \vdash \varepsilon_1  <: \varepsilon_2' \cup [z/y']\varepsilon' \cup \{x.y\}$}.
        By IH, we have
        \mbox{$\Gamma \vdash \varepsilon_1  <: \varepsilon_2' \cup [z/y']\varepsilon' \cup [x/y]\varepsilon$}.
        Using Subeffect-Def-1, we have
        \mbox{$\Gamma \vdash \varepsilon_1<: \varepsilon_2 \cup [x/y]\varepsilon$}
        \item \mbox{$\Gamma \vdash \varepsilon_1  <: \varepsilon_2 \cup \{x.g\}$} is derived by Subeffect-Def-2:\\ This case is similar to (b)
    \end{enumerate}
\end{enumerate}
\end{proof}



\begin{lemma}[Reverse of \textsc{Subeffecting-Def-2}] \label{lemma-reverse2}
If $\Gamma \vdash \varepsilon_1 \cup \{x.g\} <: \varepsilon_2$
, $\Gamma \vdash x : \{y \Rightarrow \sigma\}$, and
$\keyw{effect}\ g = \{\varepsilon\} \in \sigma$
then 
$\Gamma \vdash \varepsilon_1 \cup [x/y]\varepsilon <: \varepsilon_2$
\end{lemma}

\begin{proof}
We prove this by induction on $size(\varepsilon_1 \cup \varepsilon_2 \cup \{x.g\})$, which is defined in Fig. \ref{f-size}
\begin{enumerate}
    \item[BC] If $size(\varepsilon_1 \cup \varepsilon_2 \cup \{x.g\}) = 0$. Then $x.g$ can not have a definition. This case is vacuously true.
    \item[IS] We case on the rule used to derive  $\Gamma \vdash \varepsilon_1 \cup \{x.g\} <: \varepsilon_2$:
    \begin{enumerate}
        \item  $\Gamma \vdash \varepsilon_1 \cup \{x.g\} <: \varepsilon_2$ is derived by Subeffect-Subset:\\
        Then $x.g \in \varepsilon_2$. So we can use Subeffect-Def-1 to derive $\Gamma \vdash \varepsilon_1 \cup [x/y]\varepsilon <: \varepsilon_2$
        \item $\Gamma \vdash \varepsilon_1 \cup \{x.g\} <: \varepsilon_2$ is derived by Subeffect-Upperbound:\\
        If the Subeffect-Upperbound rule uses the effect $x.g$, then we by the premise of Subeffect-Upperbound, we have
        $\Gamma \vdash \varepsilon_1 \cup [x/y]\varepsilon <: \varepsilon_2$
        If the Subeffect-Upperbound rule does not use the effect $x.g$, then we have 
        $\varepsilon_1 = \varepsilon_1' \cup \{z.h\}$,
        $\Gamma \vdash z : \{y' \Rightarrow\sigma\}$,
        $\keyw{effect}\ h \leqslant \varepsilon' \in \sigma$,
        and 
        $\Gamma \vdash \varepsilon_1' \cup [z/y']\varepsilon' \cup \{x.g\} <: \varepsilon_2$
        By IH, we have
        $\Gamma \vdash \varepsilon_1' \cup [z/y']\varepsilon' \cup [x/y]\varepsilon <: \varepsilon_2$.
        Using Subeffect-Upperbound, we derive
        $\Gamma \vdash \varepsilon_1 \cup [x/y]\varepsilon <: \varepsilon_2$.
        \item $\Gamma \vdash \varepsilon_1 \cup \{x.g\} <: \varepsilon_2$ is derived by Subeffect-Def-1:\\
        Then we have 
        $\varepsilon_2 = \varepsilon_2' \cup \{z.h\}$,
        $\Gamma \vdash z : \{y' \Rightarrow\sigma\}$,
        $\keyw{effect}\ h = \{ \varepsilon'\} \in \sigma$,
        and 
        \mbox{$\Gamma \vdash \varepsilon_1 \cup \{x.g\} <: \varepsilon_2' \cup [z/y']\varepsilon'$}.
        By IH, we have
        \mbox{$\Gamma \vdash \varepsilon_1 \cup [x/y]\varepsilon <: \varepsilon_2' \cup [z/y']\varepsilon'$}.
        Using Subeffect-Def-1, we have
        $\Gamma \vdash \varepsilon_1 \cup [x/y]\varepsilon <: \varepsilon_2 \cup \{z.h\}$.
        \item $\Gamma \vdash \varepsilon_1 \cup \{x.g\} <: \varepsilon_2$ is derived by Subeffect-Def-2:\\ This case is similar to (b)
    \end{enumerate}
\end{enumerate}
\end{proof}





\begin{lemma}[Transitivity in subeffecting]
If $\Gamma \vdash \varepsilon_1 <: \varepsilon_2$ and $\Gamma \vdash \varepsilon_2 <: \varepsilon_3$, then $\Gamma \vdash \varepsilon_1 <: \varepsilon_3$.
\label{lemma-trans-effect}
\end{lemma}
\begin{proof}
We prove this using structural induction on $size(\Gamma, \varepsilon_1 \cup \varepsilon_2 \cup \varepsilon_3)$, which is defined in Fig. \ref{f-size}

\begin{enumerate}
\item[BC] Let $size(\Gamma, \varepsilon_1 \cup \varepsilon_2 \cup \varepsilon_3) = 0$. The judgments $\Gamma \vdash \varepsilon_1 <: \varepsilon_2$ and $\Gamma \vdash \varepsilon_2 <: \varepsilon_3$ are derived from Subeffect-Subset. So we have transitivity immediately.
\item[IS] Let $N \geq 0$, assume $\forall \varepsilon_1, \varepsilon_2, \varepsilon_3$ with $size(\Gamma, \varepsilon_1 \cup \varepsilon_2 \cup \varepsilon_3) \leq N$, if $\varepsilon_1 <: \varepsilon_2$ and $\varepsilon_2 <: \varepsilon_3$, then $\varepsilon_1 <: \varepsilon_3$. Let $\Gamma \vdash \varepsilon_1 <: \varepsilon_2$ and $\Gamma \vdash \varepsilon_2 <: \varepsilon_3$ and $size(\Gamma, \varepsilon_1 \cup \varepsilon_2 \cup \varepsilon_3) = N+1$. Want to show $\varepsilon_1 <: \varepsilon_3$. We case on the rules used to derive $\Gamma \vdash \varepsilon_1 <: \varepsilon_2$ and $\Gamma \vdash \varepsilon_2 <: \varepsilon_3$
\begin{enumerate}
    \item $\Gamma \vdash \varepsilon_1 <: \varepsilon_2$ by Subeffect-Subset
    \begin{enumerate}
        \item $\Gamma \vdash \varepsilon_2 <: \varepsilon_3$ by Subeffect-Subset. \\
        Transitivity in this case is trivial.
        \item $\Gamma \vdash \varepsilon_2 <: \varepsilon_3$ by Subeffect-Upperbound. \\
        Let $\varepsilon_2 = \varepsilon_2' \cup \{x.g\}$. By Subeffect-Upperbound, we have $\Gamma \vdash x : \{y \Rightarrow \sigma\}$ $\keyw{effect}\ g \leqslant \varepsilon \in \sigma$ and $\varepsilon_2'\cup [x/y]\varepsilon <: \varepsilon_3$ There are two cases:
        \begin{enumerate}
            \item  If $\{x.g\} \not\in \varepsilon_1$, then $\varepsilon_1 \subseteq \varepsilon_2'$. Therefore $\Gamma \vdash \varepsilon_1 <: \varepsilon_2'\cup [x/y]\varepsilon$. By induction hypothesis, we have $\Gamma \vdash \varepsilon_1 <: \varepsilon_3$. 
            \item  If $\{x.g\} \in \varepsilon_1$, then $\varepsilon_1 = \varepsilon_1'\cup \{x.g\}$, and $\varepsilon_1' \subseteq \varepsilon_2'$. So we have $\Gamma \vdash \varepsilon_1'\cup [x/y]\varepsilon <: \varepsilon_2'\cup [x/y]\varepsilon$ by Subeffect-Subset. By IH, we have $\varepsilon_1'\cup[x/y]\varepsilon <: \varepsilon_3$. Then we use Subeffect-Upperbound to derive $\varepsilon_1' \cup \{x.g\} <: \varepsilon_3$
        \end{enumerate}       
        \item 
         $\Gamma \vdash \varepsilon_2 <: \varepsilon_3$ by Subeffect-Def-1.\\
        Let $\varepsilon_3 = \varepsilon_3' \cup \{x.g\}$. We have
        $\Gamma \vdash x : \{y \Rightarrow \sigma\}$,
        $\keyw{effect}\ g = \{\varepsilon\}$,
        and
        $\Gamma \vdash \varepsilon_2 <: \varepsilon_3' \cup [x/y]\varepsilon$.
        By IH, we have $\Gamma \vdash \varepsilon_1 <: \varepsilon_3' \cup [x/y]\varepsilon$
        Then we can use Subeffect-Def-1 again to derive $\Gamma \vdash \varepsilon_1 <: \varepsilon_3$
        

        \item $\Gamma \vdash \varepsilon_2 <: \varepsilon_3$ by Subeffect-Def-2.\\
        The proof is identical to ii.
    \end{enumerate}
    \item 
    $\Gamma \vdash \varepsilon_1 <: \varepsilon_2$ by Subeffect-Upperbound. \\
    So we have $\varepsilon_1 = \varepsilon_1' \cup \{x.g\}$.
    $\Gamma \vdash x : \{y \Rightarrow \sigma\}$,
    $\keyw{effect}\ g = \{\varepsilon\}$,
    and
    $\Gamma \vdash \varepsilon_1' \cup [x/y]\varepsilon <: \varepsilon_2$.
    Using IH, we have 
     $\Gamma \vdash \varepsilon_1' \cup [x/y]\varepsilon <: \varepsilon_3$.
     Using Suveffect-Upperbound again, we have 
     $\Gamma \vdash \varepsilon_1 <: \varepsilon_3$.

    \item $\Gamma \vdash \varepsilon_1 <: \varepsilon_2$ by Subeffect-Def-1. \\
    Therefore we let $\varepsilon_2 = \varepsilon_2' \cup \{x.g\}$, $\Gamma\vdash x:\{y \Rightarrow \sigma\}$, and $effect\ g = \{\varepsilon\} \in \sigma$. By premise of Subeffect-Def-1, we have $\Gamma \vdash \varepsilon_1 <: [x/y]\varepsilon \cup \varepsilon_2'$. Since $\Gamma \vdash \varepsilon_2 <: \varepsilon_3$, we have \mbox{$\Gamma \vdash \varepsilon_2' \cup \{x.g\} <: \varepsilon_3$}. 
    \begin{enumerate}
        \item $\Gamma \vdash \varepsilon_2' \cup \{x.g\} <: \varepsilon_3$ by Subeffect-Subset\\
        Then we have $\varepsilon_3 = \varepsilon_3' \cup \{x.g\}$, and $\varepsilon_2' \subseteq \varepsilon_3'$. Therefore we have $\varepsilon_2' \cup [x/y]\varepsilon \subseteq \varepsilon_3' \cup [x/y] \varepsilon$. Therefore, $\Gamma \vdash\varepsilon_2' \cup [x/y]\varepsilon  <: \varepsilon_3' \cup [x/y] \varepsilon $. By IH, we have $\Gamma \vdash \varepsilon_1 <: \varepsilon_3' \cup [x/y] \varepsilon$. By Subeffect-Def-1 ,we have $\Gamma\vdash \varepsilon_1 <: \varepsilon_3' \cup \{x.g\} = \varepsilon_3$
        \item $\Gamma \vdash \varepsilon_2 <: \varepsilon_3$ by Subeffect-Upperbound\\
      Since the effect $\{x.g\}$ is used by Subeffect-Def-1, it is not used by the rule Subeffect-Upperbound. Let $\varepsilon_2 = \varepsilon_2'' \cup \{x.g\} \cup \{z.h\}$. By Subeffect-Def-1, we have \mbox{$\Gamma \vdash \varepsilon_1 <: \varepsilon_2'' \cup [x/y]\varepsilon \cup \{z.h\}$}. By Subeffect-Upperbound, we have 
      $\Gamma \vdash z:\{y' \Rightarrow \sigma'\}$,
      $\keyw{effect}\ h \leqslant \varepsilon' \in \sigma'$,
      and $\Gamma \vdash \varepsilon_2'' \cup \{x.g\} \cup [z/y']\varepsilon' <: \varepsilon_3$.
      By Lemma \ref{lemma-reverse1} and $\Gamma \vdash \varepsilon_1 <: \varepsilon_2'' \cup [x/y]\varepsilon \cup \{z.h\}$ , we have 
      $\Gamma \vdash \varepsilon_1 <: \varepsilon_2''\cup [x.y]\varepsilon \cup [z/y']\varepsilon'$.
      By Lemma \ref{lemma-reverse2} and $\Gamma \vdash \varepsilon_2'' \cup \{x.g\} \cup [z/y']\varepsilon' <: \varepsilon_3$, we have 
      $\Gamma \vdash \varepsilon_2'' \cup [x/y]\varepsilon \cup [z/y']\varepsilon' <: \varepsilon_3$.
      Therefore, we use IH to derive $\Gamma \vdash \varepsilon_1 <: \varepsilon_3$.
      
        
        \item $\Gamma \vdash \varepsilon_2 <: \varepsilon_3$ by Subeffect-Def-1\\
        Therefore, let $\varepsilon_3 = \varepsilon_3' \cup \{z.h\}$, $\Gamma \vdash z:\{y \Rightarrow \sigma'\}$, and $effect\ h = \{\varepsilon'\} \in \sigma'$. And we have $\Gamma \vdash \varepsilon_2 <: \varepsilon_3' \cup \{z.h\}$. By premise of Subeffect-Def-1, we have $\Gamma \vdash \varepsilon_2 <: [z/y]\varepsilon'\cup\varepsilon_3'$. By IH, we have $\Gamma \vdash \varepsilon_1 <: [z/y]\varepsilon'\cup\varepsilon_3'$. Using Subeffect-Def-1, we derive that $\Gamma \vdash \varepsilon_1 <: \varepsilon_3$. 
        \item $\Gamma \vdash \varepsilon_2 <: \varepsilon_3$ by Subeffect-Def-2\\
        This case is identical to c (ii) 
    \end{enumerate}
    \item $\Gamma \vdash \varepsilon_1 <: \varepsilon_2$ by Subeffect-Def-2\\
    This case is identical to (b)
    \item $\Gamma \vdash \varepsilon_1 <: \varepsilon_2$ by Subeffect-Lowerbound\\
    This case is identical to (c)
\end{enumerate}

\end{enumerate}
\end{proof}



\begin{lemma}[Substitution in types]
If \mbox{$\Gamma,~z : \tau \vdash \tau_1 <:  \tau_2$} and \mbox{$\Gamma~|~\Sigma \vdash l : \{ \}~[l/z]\tau$}, then\linebreak
\mbox{$\Gamma \vdash [l/z]\tau_1 <: [l/z]\tau_2$}. Furthermore, if \mbox{$\Gamma,~z : \tau \vdash \sigma_1 <: \sigma_2$} and \mbox{$\Gamma~|~\Sigma \vdash l : \{ \}~[l/z]\tau$}, then\linebreak
\mbox{$\Gamma \vdash [l/z]\sigma_1 <: [l/z]\sigma_2$}.
\label{lemma-substitution-types}
\end{lemma}

\begin{proof} The proof is by simultaneous induction on a derivation of \mbox{$\Gamma,~z : \tau \vdash \tau_1 <:  \tau_2$} and \linebreak
    \mbox{$\Gamma,~z : \tau \vdash \sigma_1 <: \sigma_2$}. For a given derivation, we proceed by cases on the final typing rule used in the derivation:\\

\noindent\underline{\textit{Case \textsc{S-Refl1}:}} \mbox{$\tau_1 = \tau_2$}, and the desired result is immediate.\\

\noindent\underline{\textit{Case \textsc{S-Trans}:}} By inversion on \textsc{S-Trans}, we get \mbox{$\Gamma,~z : \tau \vdash \tau_1 <: \tau_2$} and \mbox{$\Gamma,~z : \tau \vdash \tau_2 <: \tau_3$}. By the induction hypothesis, \mbox{$\Gamma \vdash [l/z]\tau_1 <: [l/z]\tau_2$} and \mbox{$\Gamma \vdash [l/z]\tau_2 <: [l/z]\tau_3$}. Then, by\linebreak
\mbox{\textsc{S-Trans}}, \mbox{$\Gamma \vdash [l/z]\tau_1 <: [l/z]\tau_3$}.\\

\noindent\underline{\textit{Case \textsc{S-Perm}:}} \mbox{$\tau_1 = \{ x \Rightarrow \sigma_i^{i \in 1..n} \}$} and \mbox{$\tau_2 = \{ x \Rightarrow \sigma_i'^{i \in 1..n} \}$}. Substitution preserves the permutation relations, and thus, \mbox{$[l/z]\{ x \Rightarrow \sigma_i^{i \in 1..n} \}$} is a permutation of \mbox{$[l/z]\{ x \Rightarrow \sigma_i'^{i \in 1..n} \}$}. Then, by \textsc{S-Perm}, \mbox{$\Gamma \vdash [l/z]\{ x \Rightarrow \sigma_i^{i \in 1..n} \} <: [l/z]\{ x \Rightarrow \sigma_i'^{i \in 1..n} \}$}.\\

\noindent\underline{\textit{Case \textsc{S-Width}:}} \mbox{$\tau_1 = \{ x \Rightarrow \sigma_i^{i \in 1..n + k} \}$} and \mbox{$\tau_2 = \{ x \Rightarrow \sigma_i^{i \in 1..n} \}$}, and the desired result is immediate.\\

\noindent\underline{\textit{Case \textsc{S-Depth}:}} \mbox{$\tau_1 = \{ x \Rightarrow \sigma_i^{i \in 1..n} \}$} and \mbox{$\tau_2 = \{ x \Rightarrow {\sigma'}_i^{i \in 1..n} \}$}.  By inversion on \textsc{S-Depth}, \linebreak we get \mbox{$\forall i,~\Gamma,~x : \{ x \Rightarrow {\sigma}_i^{i \in 1..n} \},~z : \tau \vdash \sigma_i <: \sigma_i'$}. By the induction hypothesis, \linebreak
\mbox{$\forall i,~\Gamma,~x : \{ x \Rightarrow {\sigma}_i^{i \in 1..n} \} \vdash [l/z]\sigma_i <: [l/z]\sigma_i'$}. Then, by \textsc{S-Depth}, \linebreak
\mbox{$\Gamma \vdash [l/z]\{ x \Rightarrow \sigma_i^{i \in 1..n} \} <: [l/z]\{ x \Rightarrow \sigma_i'^{i \in 1..n} \}$}.\\


\noindent\underline{\textit{Case \textsc{S-Refl2}:}} \mbox{$\sigma_1 = \sigma_2$}, and the desired result is immediate.\\

\sloppy
\noindent\underline{\textit{Case \textsc{S-Def}:}} \mbox{$\sigma_1 = \keyw{def} m(x : \tau_1) : \{ \varepsilon_1 \}~\tau_2$} and \mbox{$\sigma_2 = \keyw{def} m(x : \tau_1') : \{ \varepsilon_2 \}~\tau_2'$}. By inversion on \textsc{S-Def}, we get \mbox{$\Gamma,~z : \tau \vdash \tau_1' <: \tau_1$}, \mbox{$\Gamma,~z : \tau \vdash \tau_2 <: \tau_2'$}, \mbox{$\Gamma, z : \tau \vdash \varepsilon_1 <: \varepsilon_2$}. By the induction hypothesis, \mbox{$\Gamma \vdash [l/z]\tau_1' <: [l/z]\tau_1$} and \mbox{$\Gamma \vdash [l/z]\tau_2 <: [l/z]\tau_2'$}. By lemma \ref{lemma-sub-effects}, \mbox{$\Gamma \vdash [l/z]\varepsilon_1 <: [l/z]\varepsilon_2$}. Then, by \textsc{S-Def}, \mbox{$\Gamma \vdash [l/z](\keyw{def} m(x : \tau_1) : \{ \varepsilon_1 \}~\tau_2) <: [l/z](\keyw{def} m(x : \tau_1') : \{ \varepsilon_2 \}~\tau_2')$}.\\

\noindent\underline{\textit{Case \textsc{S-Effect}:}} \mbox{$\sigma_1 = \keyw{effect} g = \{ \varepsilon \}$} and \mbox{$\sigma_2 = \keyw{effect} g$}, and the desired result is immediate.\\

\noindent Thus, substituting terms in types preserves the subtyping relationship.
\end{proof}


\begin{lemma}[Substitution in expressions and effects]
\label{lemma-sub-effects}
If \mbox{$\Gamma,~z : \tau'~|~\Sigma \vdash e :  \{ \varepsilon \}~\tau$} and \mbox{$\Gamma~|~\Sigma \vdash l : \{ \}~[l/z]\tau'$}, then \mbox{$\Gamma~|~\Sigma \vdash [l/z]e :  \{ [l/z]\varepsilon \}~[l/z]\tau$}. \\[3ex] And if $\Gamma, z : \tau' \mid \Sigma \vdash \varepsilon_1 <: \varepsilon_2$ and $\Gamma \mid \Sigma \vdash l : \{\} [l/z]\tau$, then $\Gamma \mid \Sigma \vdash [l/z] \varepsilon_1 <: [l/z] \varepsilon_2$.\\[3ex]
And if \mbox{$\Gamma,~z : \tau'~|~\Sigma \vdash d : \sigma$} and
\mbox{$\Gamma~|~\Sigma \vdash l : \{ \}~[l/z]\tau'$}, then $\Gamma~|~\Sigma \vdash [l/z]d : [l/z]\sigma$. \\[3ex]
Furthermore, if $\Gamma, z : \tau' \mid \Sigma \vdash \varepsilon\ {wf}$,  then 
$\Gamma \mid \Sigma \vdash [l/z]\varepsilon\ {wf}$
\end{lemma}

\begin{proof} The proof is by simultaneous induction on a derivation of $\Gamma,~z : \tau'~|~\Sigma \vdash e : \{ \varepsilon \}~\tau$, $\Gamma,~z : \tau'~|~\Sigma \vdash d : \sigma$, $\Gamma, z :\tau' \mid \Sigma \vdash \varepsilon_1 <: \varepsilon_1$, and $\Gamma, z : \tau' \mid \Sigma \vdash \varepsilon wf$. For a given derivation, we proceed by cases on the final typing rule used in the derivation:\\

\noindent\underline{\textit{Case \textsc{T-Var}:}} $e = x$, and by inversion on \textsc{T-Var}, we get \mbox{$x : \tau \in (\Gamma,~z : \tau')$}. There are two subcases to consider, depending on whether $x$ is $z$ or another variable. If $x = z$, then $[l/z]x = l$ and $\tau = \tau'$. The required result is then $\Gamma~|~\Sigma \vdash l : \{ \}~[l/z]\tau'$, which is among the assumptions of the lemma. Otherwise, $[l/z]x = x$, and the desired result is immediate.\\

\noindent\underline{\textit{Case \textsc{T-New}:}} \mbox{$e = \keywadj{new}(x \Rightarrow \overline{d})$}, and by inversion on \textsc{T-New}, we get\linebreak
\mbox{$\forall i,~d_i \in \overline{d},~\sigma_i \in \overline{\sigma},~\Gamma,~x : \{ x \Rightarrow \overline{\sigma} \},~z : \tau'~|~\Sigma \vdash d_i : \sigma_i$}. By the induction hypothesis,
\mbox{$\forall i,~d_i \in \overline{d},~\sigma_i \in \overline{\sigma},~\Gamma,~x : \{ x \Rightarrow \overline{\sigma} \}~|~\Sigma \vdash [l/z]d_i : [l/z]\sigma_i$}. Then, by \textsc{T-New},
\mbox{$\Gamma~|~\Sigma \vdash \keywadj{new}(x \Rightarrow [l/z]\overline{d}) : \{ \}~\{ x \Rightarrow [l/z]\overline{\sigma} \}$}, i.e.,
\mbox{$\Gamma~|~\Sigma \vdash [l/z](\keywadj{new}(x \Rightarrow \overline{d})) : \{ \}~[l/z]\{ x \Rightarrow \overline{\sigma} \}$}.\\

\noindent\underline{\textit{Case \textsc{T-Method}:}} \mbox{$e = e_1.m(e_2)$}, and by inversion on \textsc{T-Method}, we get
\mbox{$\Gamma,~z : \tau'~|~\Sigma \vdash e_1 : \{ \varepsilon_1 \}~\{ x \Rightarrow \overline{\sigma} \}$}; \mbox{$\keyw{def}~ m(y : \tau_2) : \{ \varepsilon_3 \}~\tau_1 \in \overline{\sigma}$};
\mbox{$\Gamma,~z : \tau'~|~\Sigma \vdash [e_1/x][e_2/y]\varepsilon_3~\mathit{wf}$}; and $\Gamma,~z : \tau'~|~\Sigma \vdash e_2 : \{ \varepsilon_2 \}~[e_1/x]\tau_2$. By the induction hypothesis, \mbox{$\Gamma~|~\Sigma \vdash [l/z]e_1 : \{ [l/z]\varepsilon_1 \}~[l/z]\{ x \Rightarrow \overline{\sigma} \}$},
\mbox{$\keyw{def}~ m(y : [l/z]\tau_2) : \{ [l/z]\varepsilon_3 \}~[l/z]\tau_1 \in [l/z]\overline{\sigma}$}, \mbox{$\Gamma~|~\Sigma \vdash [l/z]([e_1/x][e_2/y]\varepsilon_3)~\mathit{wf}$}, and
\mbox{$\Gamma~|~\Sigma \vdash [l/z]e_2 : \{ [l/z]\varepsilon_2 \}~[l/z][e_1/x]\tau_2$}. Then, by \mbox{\textsc{T-Method}},
\mbox{$\Gamma~|~\Sigma \vdash [l/z]e_1.m([l/z]e_2) : \{ [l/z]\varepsilon_1 \cup [l/z]\varepsilon_2 \cup [l/z]([e_1/x][e_2/y]\varepsilon_3) \}~[l/z]([e_1/x][e_2/y]\tau_1)$},
i.e., \mbox{$\Gamma~|~\Sigma \vdash [l/z](e_1.m(e_2)) : \{ [l/z](\varepsilon_1 \cup \varepsilon_2 \cup [e_1/x][e_2/y]\varepsilon_3) \}~[l/z]([e_1/x][e_2/y]\tau_1)$}.\\

\noindent\underline{\textit{Case \textsc{T-Field}:}} $e = e_1.f$, and by inversion on \textsc{T-Field}, we get $\Gamma,~z : \tau'~|~\Sigma \vdash e_1 : \{ \varepsilon \}~\{ x \Rightarrow \overline{\sigma} \}$ and \mbox{$\keyw{var}~ f : \tau \in \overline{\sigma}$}. By the induction hypothesis, $\Gamma~|~\Sigma \vdash [l/z]e_1 : \{ [l/z]\varepsilon \}~[l/z]\{ x \Rightarrow \overline{\sigma} \}$ and \mbox{$\keyw{var}~ f : [l/z]\tau \in [l/z]\overline{\sigma}$}. Then, by \textsc{T-Field}, $\Gamma~|~\Sigma \vdash ([l/z]e_1).f : \{ [l/z]\varepsilon \}~[l/z]\tau$, i.e., $\Gamma~|~\Sigma \vdash [l/z](e_1.f) : \{ [l/z]\varepsilon \}~[l/z]\tau$.\\

\noindent\underline{\textit{Case \textsc{T-Assign}:}} \mbox{$e = (e_1.f = e_2)$}, and by inversion on \textsc{T-Assign}, we get
\mbox{$\Gamma,~z : \tau'~|~\Sigma \vdash e_1 : \{ \varepsilon_1 \}~\{ x \Rightarrow \overline{\sigma} \}$}; $\keyw{var}~ f : \tau \in \overline{\sigma}$; and \mbox{$\Gamma,~z : \tau'~|~\Sigma \vdash e_2 : \{ \varepsilon_2 \}~\tau$}. By the induction hypothesis,  \mbox{$\Gamma~|~\Sigma \vdash [l/z]e_1 : \{ [l/z]\varepsilon_1 \}~[l/z]\{ x \Rightarrow \overline{\sigma} \}$}; \mbox{$\keyw{var}~ f : [l/z]\tau \in [l/z]\overline{\sigma}$};  and \mbox{$\Gamma~|~\Sigma \vdash [l/z]e_2 : \{ [l/z]\varepsilon_2 \}~[l/z]\tau$}. Then, by \textsc{T-Assign}, 
\mbox{$\Gamma~|~\Sigma \vdash [l/z]e_1.f = [l/z]e_2 : \{ [l/z]\varepsilon_1 \cup [l/z]\varepsilon_2\}~[l/z]\tau$}, i.e.,
\mbox{$\Gamma~|~\Sigma \vdash [l/z](e_1.f=e_2) : \{ [l/z](\varepsilon_1 \cup \varepsilon_2) \}~[l/z]\tau$}.\\

\noindent\underline{\textit{Case \textsc{T-Loc}:}} $e = l$, $[l/z]l = l$, and the desired result is immediate.\\

\noindent\underline{\textit{Case \textsc{T-Sub}:}} $e = e_1$, and by inversion on T-Sub, we get $\Gamma, z:\tau' \mid \Sigma \vdash e_1:\{\varepsilon_1\} \tau_1$, $\Gamma, z:\tau' \mid \Sigma \vdash \tau_1 <: \tau_2$ and $\Gamma, z:\tau' \mid \Sigma \vdash \varepsilon_1 <: \varepsilon_2$.
    By induction hypothesis, we have
    \mbox{$\Gamma \mid \Sigma \vdash [l/z]e_1:\{[l/z]\varepsilon_1\} [l/z]\tau_1$}, \mbox{$\Gamma \mid \Sigma \vdash [l/z]\tau_1 <: [l/z]\tau_2$}, and \mbox{$\Gamma \mid \Sigma \vdash [l/z]\varepsilon_1 <: [l/z]\varepsilon_2$}.
    Then, by T-sub,
    \mbox{$\Gamma \mid \Sigma \vdash [l/z]e_1 : \{[l/z]\varepsilon_2\}[l/z]\tau_2$}

\noindent\underline{\textit{Case \textsc{DT-Def}:}}  By inversion, we have 
  $ \Gamma, z:\tau ,~x : \tau_1 \mid \Sigma \vdash e : \{ \varepsilon' \}~\tau_2$,
  $\Gamma,z:\tau ,~x : \tau_1 \mid \Sigma \vdash \varepsilon~\mathit{wf} $,
  $\Gamma, z:\tau \mid \Sigma \vdash \varepsilon' <: \varepsilon $,
  By IH, we have
\mbox{$ \Gamma ,~x : [l/z]\tau_1 \mid \Sigma \vdash [l/z]e : \{ [l/z]\varepsilon' \}~[l/z]\tau_2$},
  \mbox{$\Gamma, ~x : [l/z]\tau_1 \mid \Sigma \vdash [l/z]\varepsilon~\mathit{wf} $},
  \mbox{$\Gamma \mid \Sigma \vdash [l/z]\varepsilon' <: [l/z]\varepsilon $}.
  By DT-Def, we have
  \mbox{$\Gamma \mid \Sigma \vdash \keyw{def} m(x : [l/z]\tau_1) : \{ [l/z]\varepsilon \}~[l/z]\tau_2 = [l/z]e~:~\keyw{def} m(x : [l/z]\tau_1) : \{ [l/z]\varepsilon \}~[l/z]\tau_2$}

\noindent\underline{\textit{Case \textsc{DT-Var}:}} $d = \keyw{var} f : \tau = n$, and by definition of $n$, there are two subcases:

\underline{\textit{Subcase $n$ is $x$:}} In this case, \mbox{$d = \keyw{var} f : \tau = x$}, and by inversion on \textsc{DT-Var}, we get\linebreak
\mbox{$\Gamma,~z : \tau'~|~\Sigma \vdash x : \{ \}~\tau$}. There are two subcases to consider, depending on whether $x$ is $z$ or another variable. If $x = z$, then by the induction hypothesis, \mbox{$\Gamma~|~\Sigma \vdash [l/z]x : \{ \}~[l/z]\tau$}, which yields \mbox{$\Gamma~|~\Sigma \vdash l : \{ \}~[l/z]\tau$} and \mbox{$\tau = \tau'$}, and thus,
\mbox{$\Gamma~|~\Sigma \vdash \keyw{var} f : [l/z]\tau = l~:~\keyw{var} f : [l/z]\tau$}, i.e.,\linebreak
\mbox{$\Gamma~|~\Sigma \vdash [l/z](\keyw{var} f : \tau = l)~:~[l/z](\keyw{var} f : \tau)$}, as required. If \mbox{$x \not = z$}, then
\mbox{$\Gamma~|~\Sigma \vdash [l/z]x : \{ \}~[l/z]\tau$} yields \mbox{$\Gamma~|~\Sigma \vdash x : \{ \}~[l/z]\tau$}, and thus,
\mbox{$\Gamma~|~\Sigma \vdash \keyw{var} f : [l/z]\tau = x~:~\keyw{var} f : [l/z]\tau$}, i.e.,\linebreak
\mbox{$\Gamma~|~\Sigma \vdash [l/z](\keyw{var} f : \tau = x)~:~[l/z](\keyw{var} f : \tau)$}, as required.

\underline{\textit{Subcase $n$ is $l$:}} In this case, \mbox{$d = \keyw{var} f : \tau = l$}, i.e., the field is resolved to a location $l$. This is not affected by the substitution, and the desired result is immediate.\\



\noindent\underline{\textit{Case \textsc{DT-Effect}:}}
By IH, we have
 $\Gamma \mid \Sigma \vdash [l/z]\varepsilon\ wf$.
 We use DT-Effect to derive
 \mbox{$\Gamma \mid \Sigma \vdash \keyw{effect} \ g = \{[l/z]\varepsilon\} :\keyw{effect}\ g = \{[l/z]\varepsilon\}$}


\noindent\underline{\textit{Case \textsc{Subeffect-Subset}:}}
   By inversion, we have $\varepsilon_1 \subseteq \varepsilon_2$. So $[l/z] \varepsilon_1 \subseteq [l/z] \varepsilon_2$. By Subeffect-Subset, we have 
    $\Gamma \mid \Sigma \vdash [l/z]\varepsilon_1 <: [l/z]\varepsilon_2$.
    
    
\noindent\underline{\textit{Case \textsc{Subeffect-Upperbound}:}}
    By inversion, we have 
    \mbox{$\varepsilon_1 = \varepsilon_1' \cup \{x.g\}$},
    \mbox{$\Gamma, z:\tau \mid \Sigma \vdash x: \{y \Rightarrow \sigma \}$},
    \mbox{$effect\ g \leqslant \{\varepsilon\} \in \sigma $}
    and 
    \mbox{$\Gamma, z:\tau \mid \Sigma \vdash \varepsilon_1' \cup [x/y]\varepsilon <: \varepsilon_2$}.
    By IH, we have 
    \mbox{$\Gamma \mid\Sigma \vdash [l/z]\varepsilon_1' \cup [l/z][x/y]\varepsilon <: [l/z]\varepsilon_2$}.
    Since $y$ is a free variable, we select $y$ such that $x \neq y$ and $y \neq z$. We case on if $z = x$:
    \begin{enumerate}
    \item
    If $z \neq x$, then we can swap the order of the substitutions on $\varepsilon$
    \mbox{$\Gamma \mid\Sigma \vdash [l/z]\varepsilon_1' \cup [x/y][l/z]\varepsilon <: [l/z]\varepsilon_2$}.
    Using substitution lemma for typing on \mbox{$\Gamma, z:\tau \mid \Sigma \vdash x: \{y \Rightarrow \sigma \}$}, we have 
    \mbox{$\Gamma \mid \Sigma \vdash x : \{y \Rightarrow [l/z]\sigma \}$},
    \mbox{$\keyw{effect}\ g \leqslant [l/z]\varepsilon \in [l/z]\sigma$}.
    Using Subeffect-Upperbound, we have 
    \mbox{$\Gamma \mid \Sigma \vdash [l/z]\varepsilon_1' \cup \{x.g\} <: [l/z]\varepsilon_2$},
    Which is equivalent to 
    \mbox{$\Gamma \mid \Sigma \vdash [l/z] \varepsilon_1 <: [l/z] \varepsilon_2$}.
     \item If $z = x$
     Then we have
     \mbox{$\Gamma \mid \Sigma \vdash [l/z]\varepsilon_1'\cup [l/x,y]\varepsilon <:[l/z]\varepsilon_2$},
     Which is equivalent to
     \mbox{$\Gamma \mid \Sigma \vdash [l/z]\varepsilon_1'\cup [l/y][l/z]\varepsilon <:[l/z]\varepsilon_2$}
     We case on the derivation of \mbox{$\Gamma, z:\tau \mid \Sigma \vdash z : \{y \Rightarrow \sigma\}$}. 
     \begin{enumerate}
         \item (T-Var)\\[3ex]
         \infer
    {\Gamma, z:\tau \mid \Sigma \vdash z : \tau}
    {z :\tau \in \Gamma, z : \tau}\\[3ex]
    So $\tau = \{y \Rightarrow \sigma\}$. By our assumption, we have 
    $\Gamma \mid \Sigma \vdash l : \{y \Rightarrow [l/z]\sigma\}$.
    Since \mbox{$\keyw{effect}\ g\leqslant \varepsilon \in \sigma$}, we have
    $\keyw{effect}\ g \leqslant [l/z]\varepsilon \in [l/z]\sigma$.
    Therefore, we can use Subeffect-Upperbound on $\{l.g\}$ to derive 
    $\Gamma \mid \Sigma \vdash [l/z]\varepsilon_1' \cup \{l.g\} <: [l/z]\varepsilon_2$,
    Which is equivalent to 
    $\Gamma \mid \Sigma \vdash [l/z]\varepsilon_1 <: [l/z]\varepsilon_2$
    \item (T-Sub)\\[3ex]
    \infer
    {\Gamma, z : \tau \mid \Sigma \vdash z : \{y \Rightarrow \sigma\}}
    {\Gamma, z : \tau\mid \Sigma \vdash z : \tau_1 \quad \Gamma, z:\tau\mid\Sigma \vdash \tau_1 <: \{y \Rightarrow \sigma\}}\\[3ex]
    Notice that we introduced a new type $\tau_1$ that $z$ can be ascribed to. The judgment \mbox{$\Gamma, z : \tau\mid \Sigma \vdash z : \tau_1$} can be derived by T-Sub, which introduce a new type $\tau_2$ such that $\Gamma, z:\tau \mid \Sigma \vdash \tau_2 <: \tau_1$, or T-Var, which shows $\tau_1 = \tau$. Therefore if we follow the derivation tree, we get a chain relation
    $\Gamma, z:\tau\mid\Sigma \vdash \tau_1 <: \{y \Rightarrow \sigma\}$,
    $\Gamma, z:\tau\mid\Sigma \vdash \tau_2 <: \tau_1$,
    $\dots$,
    $\Gamma, z:\tau\mid\Sigma \vdash \tau <: \tau_n$.
    We can apply IH on these judgments, so we have a chain
    $\Gamma\mid\Sigma \vdash [l/z]\tau_1 <: \{y \Rightarrow [l/z]\sigma\}$,
    $\Gamma\mid\Sigma \vdash [l/z]\tau_2 <: [l/z]\tau_1$
    $\dots$,
    $\Gamma\mid\Sigma \vdash [l/z]\tau <: [l/z]\tau_n$.
    By transitivity of subtyping, we have
    $\Gamma \mid \Sigma \vdash [l/z]\tau <: \{y \Rightarrow [l/z]\sigma\}$
    So we have 
    $\Gamma \mid \Sigma \vdash l : \{y \Rightarrow [l/z]\sigma\}$
    The rest of the proof is similar to case (a).
     \end{enumerate}
     \end{enumerate}

\noindent\underline{\textit{Case \textsc{Subeffect-Def-1}:}}
    By inversion, we have 
    $\varepsilon_2 = \varepsilon_2' \cup \{x.g\}$,
    $\Gamma, z:\tau \mid \Sigma \vdash x: \{y \Rightarrow \sigma \}$,
    \mbox{$\keyw{effect}\ g = \{\varepsilon\} \in \sigma $},
    and 
    $\Gamma, z:\tau \mid \Sigma \vdash \varepsilon_1 <: \varepsilon_2' \cup [x/y]\varepsilon$.
    By IH, we have 
    \mbox{$\Gamma \mid \Sigma \vdash [l/z]\varepsilon_1 <: [l/z]\varepsilon_2' \cup [l/z][x/y]\varepsilon$}.
    Since y is a free variable, we can select y such that $y \neq x$ and $y\neq z$. We case on if $x = z$:
    \begin{enumerate}
    \item If $z \neq x$, then
    $\Gamma \mid \Sigma \vdash [l/z]\varepsilon_1 <: [l/z]\varepsilon_2' \cup [x/y][l/z]\varepsilon$
    By substitution lemma for typing, we have 
    $\Gamma \mid \Sigma \vdash x : \{y \Rightarrow [l/z]\sigma \}$,
    $\keyw{effect}\ g = [l/z]\varepsilon \in [l/z]\sigma$.
    Using Subeffect-Def-1, we have
    $\Gamma \mid \Sigma \vdash [l/z]\varepsilon_1 <: [l/z]\varepsilon_2' \cup \{x.g\}$,
    which is equivalent to 
    $\Gamma \mid \Sigma \vdash [l/z]\varepsilon_1 <: [l/z]\varepsilon_2$
    \item If $z = x$
     Then we have
     $\Gamma \mid \Sigma \vdash [l/z]\varepsilon_1 <:[l/z]\varepsilon_2' \cup [l/x, y]\varepsilon$,
     which is equivalent to
     $\Gamma \mid \Sigma \vdash [l/z]\varepsilon_1  <:[l/z]\varepsilon_2' \cup [l/y][l/z]\varepsilon$
     
      We case on the derivation of $\Gamma, z:\tau \mid \Sigma \vdash z : \{y \Rightarrow \sigma\}$. \begin{enumerate}
         \item (T-Var)\\[3ex]
         \infer
    {\Gamma, z:\tau \mid \Sigma \vdash z : \tau}
    {z :\tau \in \Gamma, z : \tau}\\[3ex]
    So $\tau = \{y \Rightarrow \sigma\}$. By our assumption, we have 
    $\Gamma \mid \Sigma \vdash l : \{y \Rightarrow [l/z]\sigma\}$.
    Since \mbox{$\keyw{effect}\ g = \{\varepsilon\} \in \sigma$}, we have
    \mbox{$\keyw{effect}\ g = \{[l/z]\varepsilon\} \in [l/z]\sigma$}.
    Therefore, we can use Subeffect-Def-1 on $\{l.g\}$ to derive 
    $\Gamma \mid \Sigma \vdash [l/z]\varepsilon_1 <: [l/z]\varepsilon_2' \cup \{l.g\}$,
    Which is equivalent to 
    $\Gamma \mid \Sigma \vdash [l/z]\varepsilon_1 <: [l/z]\varepsilon_2$
    \item (T-Sub)\\[3ex]
    \infer
    {\Gamma, z : \tau \mid \Sigma \vdash z : \{y \Rightarrow \sigma\}}
    {\Gamma, z : \tau\mid \Sigma \vdash z : \tau_1 \quad \Gamma, z:\tau\mid\Sigma \vdash \tau_1 <: \{y \Rightarrow \sigma\}}\\[3ex]
    Notice that we introduced a new type $\tau_1$ that $z$ can be ascribed to. The judgment $\Gamma, z : \tau\mid \Sigma \vdash z : \tau_1$ can be derived by T-Sub, which introduce a new type $\tau_2$ such that $\Gamma, z:\tau \mid \Sigma \vdash \tau_2 <: \tau_1$, or T-Var, which shows $\tau_1 = \tau$. Therefore if we follow the derivation tree, we get a chain relation
    $\Gamma, z:\tau\mid\Sigma \vdash \tau_1 <: \{y \Rightarrow \sigma\}$,
    $\Gamma, z:\tau\mid\Sigma \vdash \tau_2 <: \tau_1$,
    $\dots$,
    $\Gamma, z:\tau\mid\Sigma \vdash \tau <: \tau_n$.
    We can apply IH on these judgments, so we have a chain
    $\Gamma\mid\Sigma \vdash [l/z]\tau_1 <: \{y \Rightarrow [l/z]\sigma\}$,
    $\Gamma\mid\Sigma \vdash [l/z]\tau_2 <: [l/z]\tau_1$,
    $\dots$,
    $\Gamma\mid\Sigma \vdash [l/z]\tau <: [l/z]\tau_n$.
    By transitivity of subtyping, we have
    $\Gamma \mid \Sigma \vdash [l/z]\tau <: \{y \Rightarrow [l/z]\sigma\}$.
    So we have 
    $\Gamma \mid \Sigma \vdash l : \{y \Rightarrow [l/z]\sigma\}$.
    The rest of the proof is similar to case (a).
     \end{enumerate}
    \end{enumerate}

\noindent\underline{\textit{Case \textsc{Subeffect-Def-2}:}}
This case is identical to \underline{\textit{Case \textsc{Subeffect-Upperbound}}}


\noindent\underline{\textit{Case \textsc{Subeffect-Lowerbound}:}}
This case is identical to \underline{\textit{Case \textsc{Subeffect-Def-1}}}


\noindent\underline{\textit{Case \textsc{WF-Effect}:}}Let $n_i.g_j \in \varepsilon$ be arbitrary. By inversion, we have
$\Gamma, z : \tau \mid \Sigma \vdash n_i : \{\} \{y_i \Rightarrow \overline{\sigma_i}\}$.
and the effect declaration of $g_j$ is in $\overline{\sigma_i}$. 
By IH, we have 
$\Gamma \mid \Sigma \vdash [l/z]n_i : \{\} \{y_i \Rightarrow [l/z]\overline{\sigma_i}\}$
and the effect declaration of $g_j$ is in $\overline{\sigma_i}$. So we have $[l/z]\varepsilon\ wf$ by WF-Effect. \\[3ex]


\noindent Thus, substituting terms in a well-typed expression preserves the typing.
\end{proof}


\section{Proof of Theorem \ref{theorem-preservation} (Preservation)}
\label{app-preservation}


If \mbox{$\Gamma~|~\Sigma \vdash e : \{ \varepsilon \}~\tau$}, \mbox{$\mu : \Sigma$}, and \mbox{$\langle e~|~\mu \rangle \longrightarrow \langle e'~|~\mu' \rangle$}, then \mbox{$\exists \Sigma' \supseteq \Sigma$}, \mbox{$\mu' : \Sigma'$}, $\exists \varepsilon'$, such that $\Gamma \vdash \varepsilon' <: \varepsilon$, and \mbox{$\Gamma~|~\Sigma' \vdash e' : \{ \varepsilon' \}~\tau$}.


\begin{proof} The proof is by induction on a derivation of \mbox{$\Gamma~|~\Sigma \vdash e : \{ \varepsilon \}~\tau$}. At each step of the induction, we assume that the desired property holds for all subderivations and proceed by case analysis on the final rule in the derivation. Since we assumed \mbox{$\langle e~|~\mu \rangle \longrightarrow \langle e'~|~\mu' \rangle$} and there are no evaluation rules corresponding to variables or locations, the cases when $e$ is a variable \mbox{(\textsc{T-Var})} or a location (\textsc{T-Loc}) cannot arise. For the other cases, we argue as follows:
\\

\noindent\underline{\textit{Case \textsc{T-New}:}}
\mbox{$e = \keywadj{new}(x \Rightarrow \overline{d})$}, and by inversion on \textsc{T-New}, we get\linebreak
\mbox{$\forall i,~d_i \in \overline{d},~\sigma_i \in \overline{\sigma},~\Gamma,~x : \{ x \Rightarrow \overline{\sigma} \}~|~\Sigma \vdash d_i : \sigma_i$}. The store changes from $\mu$ to\linebreak
\mbox{$\mu' = \mu,~l \mapsto \{ x \Rightarrow \overline{d} \}$}, i.e., the new store is the old store augmented with a new mapping for the location $l$, which was not in the old store ($l \not \in \mathit{dom}(\mu)$). From the premise of the theorem, we know that $\mu : \Sigma$, and by the induction hypothesis, all expressions of $\Gamma$ are properly allocated in $\Sigma$. Then, by \textsc{T-Store}, we have $\mu,~l \mapsto \{ x \Rightarrow \overline{d} \}~:~\Sigma,~l : \{ x \Rightarrow \overline{\sigma} \}$, which implies that $\Sigma' = \Sigma,~l : \{ x \Rightarrow \overline{\sigma} \}$. Finally, by \textsc{T-Loc}, $\Gamma~|~\Sigma \vdash l : \{\}~\{ x \Rightarrow \overline{\sigma} \}$, and $\varepsilon' = \varnothing = \varepsilon$. Thus, the right-hand side is well typed.
\\

\noindent\underline{\textit{Case \textsc{T-Method}:}}
$e = e_1.m(e_2)$, and by the definition of the evaluation relation, there are two subcases:

\underline{\textit{Subcase \textsc{E-Congruence}:}} In this case, either $\langle e_1~|~\mu \rangle \longrightarrow \langle e_1'~|~\mu' \rangle$ or $e_1$ is a value and \mbox{$\langle e_2~|~\mu \rangle \longrightarrow \langle e_2'~|~\mu' \rangle$}. Then, the result follows from the induction hypothesis and \mbox{\textsc{T-Method}}.

\underline{\textit{Subcase \textsc{E-Method}:}} In this case, both $e_1$ and $e_2$ are values, namely, locations $l_1$ and $l_2$ respectively. Then, by inversion on \textsc{T-Method}, we get that \mbox{$\Gamma~|~\Sigma \vdash e_1 : \{ \varepsilon_1 \}~\{ x \Rightarrow \overline{\sigma} \}$},\linebreak
\mbox{$\keyw{def}~ m(y : \tau_2) : \{ \varepsilon_3 \}~\tau_1 \in \overline{\sigma}$}, $\Gamma~|~\Sigma \vdash [e_1/x][e_2/y]\varepsilon_3~\mathit{wf}$, $\Gamma~|~\Sigma \vdash e_2 : \{ \varepsilon_2 \}~[e_1/x]\tau_2$, and\linebreak
$\varepsilon = \varepsilon_1 \cup \varepsilon_2 \cup [e_1/x][e_2/y]\varepsilon_3$. The store $\mu$ does not change, and since \mbox{\textsc{T-Store}} has been applied throughout, the store is well typed, and thus,\linebreak
\mbox{$\Gamma~|~\Sigma \vdash \keyw{def} m(x : \tau_1) : \{ \varepsilon \}~\tau_2 = e~:~\keyw{def} m(x : \tau_1) : \{ \varepsilon \}~\tau_2$}. Then, by inversion on \mbox{\textsc{DT-Def}}, we know that \mbox{$\Gamma,~x : \tau_1~|~\Sigma \vdash e : \{ \varepsilon' \}~\tau_2$} and\linebreak
$\Gamma, x : \tau_1 \mid \Sigma \vdash \varepsilon' <: \varepsilon$. Finally, by the subsumption lemma, substituting locations for variables in $e$ preserves its type, and therefore, the right-hand side is well typed.
\\

\noindent\underline{\textit{Case \textsc{T-Field}:}}
$e = e_1.f$, and by the definition of the evaluation relation, there are two subcases:

\underline{\textit{Subcase \textsc{E-Congruence}:}} In this case, $\langle e_1~|~\mu \rangle \longrightarrow \langle e_1'~|~\mu' \rangle$, and the result follows from the induction hypothesis and \textsc{T-Field}.

\underline{\textit{Subcase \textsc{E-Field}:}} In this case, $e_1$ is a value, i.e., a location $l$. Then, by inversion on \mbox{\textsc{T-Field}}, we have $\Gamma~|~\Sigma \vdash l : \{ \varepsilon \}~\{ x \Rightarrow \overline{\sigma} \}$, where $\varepsilon = \varnothing$, and $\keyw{var}~ f : \tau \in \overline{\sigma}$. The store $\mu$ does not change, and since \textsc{T-Store} has been applied throughout, the store is well typed, and thus, \mbox{$\Gamma~|~\Sigma \vdash \keyw{var}~ f : \tau = l_1~:~\keyw{var}~ f : \tau$}. Then, by inversion on \textsc{DT-Varl}, we know that\linebreak
\mbox{$\Gamma~|~\Sigma \vdash l_1 : \{ \}~\tau$} and $\varepsilon' = \varnothing = \varepsilon$, and the right-hand side is well typed.
\\

\noindent\underline{\textit{Case \textsc{T-Assign}:}}
$e = (e_1.f = e_2)$, and by the definition of the evaluation relation, there are two subcases:

\underline{\textit{Subcase \textsc{E-Congruence}:}} In this case, either $\langle e_1~|~\mu \rangle \longrightarrow \langle e_1'~|~\mu' \rangle$ or $e_1$ is a value and \mbox{$\langle e_2~|~\mu \rangle \longrightarrow \langle e_2'~|~\mu' \rangle$}. Then, the result follows from the induction hypothesis and \textsc{T-Assign}.

\underline{\textit{Subcase \textsc{E-Assign}:}} In this case, both $e_1$ and $e_2$ are values, namely locations $l_1$ and $l_2$ respectively. Then, by inversion on \textsc{T-Assign}, we get that \mbox{$\Gamma~|~\Sigma \vdash l_1 : \{ \varepsilon_1 \}~\{ x \Rightarrow \overline{\sigma} \}$}, \mbox{$\keyw{var}~ f : \tau \in \overline{\sigma}$}, \mbox{$\Gamma~|~\Sigma \vdash l_2 : \{ \varepsilon_2 \}~\tau$}, and \mbox{$\varepsilon = \varepsilon_1 = \varepsilon_2 = \varnothing$}. The store changes as follows:\linebreak
\mbox{$\mu' = [l_1 \mapsto \{ x \Rightarrow \overline{d}' \}/l_1 \mapsto \{ x \Rightarrow \overline{d} \}]\mu$}, where $\overline{d}' = [\keyw{var} f:\tau = l_2/\keyw{var} f : \tau = l]\overline{d}$. However, since \textsc{T-Store} has been applied throughout and the substituted location has the type expected by \textsc{T-Store}, the new store is well typed (as well as the old store), and thus,\linebreak
\mbox{$\Gamma~|~\Sigma \vdash \keyw{var}~ f : \tau = l_2~:~\keyw{var}~ f : \tau$}. Then, by inversion on \textsc{DT-Varl}, we know that\linebreak
\mbox{$\Gamma~|~\Sigma \vdash l_2 : \{ \}~\tau$} and $\varepsilon' = \varnothing$, and the right-hand side is well typed.
\\

\noindent\underline{\textit{Case \textsc{T-Sub}:}}
The result follows directly from the induction hypothesis.
\\

\noindent Thus, the program written in this language is always well typed.
\end{proof}



\section{Proof of Theorem \ref{theorem-progress} (Progress)}
\label{app-progress}


If $\varnothing~|~\Sigma \vdash e : \{ \varepsilon \}~\tau$ (i.e., $e$ is a closed, well-typed expression), then either
\begin{enumerate}
\item $e$ is a value (i.e., a location) or
\item $\forall \mu$ such that $\mu : \Sigma$,
   $\exists e', \mu'$ such that $\langle e~|~\mu \rangle \longrightarrow \langle e'~|~\mu' \rangle$.
\end{enumerate}

\begin{proof} The proof is by induction on the derivation of $\Gamma~|~\Sigma \vdash e : \{ \varepsilon \}~\tau$, with a case analysis on the last typing rule used. The case when $e$ is a variable (\textsc{T-Var}) cannot occur, and the case when $e$ is a location (\textsc{T-Loc}) is immediate, since in that case $e$ is a value. For the other cases, we argue as follows:
\\

\noindent\underline{\textit{Case \textsc{T-New}:}}
$e = \keywadj{new}(x \Rightarrow \overline{d})$, and by \textsc{E-New}, $e$ can make a step of evaluation if the $\keywadj{new}$ expression is closed and there is a location available that is not in the current store $\mu$. From the premise of the theorem, we know that the expression is closed, and there are infinitely many available new locations, and therefore, $e$ indeed can take a step and become a value (i.e., a location $l$). Then, the new store $\mu'$ is $\mu, l \mapsto \{ x \Rightarrow \overline{d} \}$, and all the declarations in $\overline{d}$ are mapped in the new store.
\\

\noindent\underline{\textit{Case \textsc{T-Method}:}}
\mbox{$e = e_1.m(e_2)$}, and by the induction hypothesis applied to\linebreak
\mbox{$\Gamma~|~\Sigma \vdash e_1 : \{ \varepsilon_1 \}~\{ x \Rightarrow \overline{\sigma} \}$}, either $e_1$ is a value or else it can make a step of evaluation, and, similarly, by the induction hypothesis applied to $\Gamma~|~\Sigma \vdash e_2 : \{ \varepsilon_2 \}~[e_1/x]\tau_2$, either $e_2$ is a value or else it can make a step of evaluation. Then, there are two subcases:

\underline{\textit{Subcase $\langle e_1~|~\mu \rangle \longrightarrow \langle e_1'~|~\mu' \rangle$ or $e_1$ is a value and $\langle e_2~|~\mu \rangle \longrightarrow \langle e_2'~|~\mu' \rangle$:}} If $e_1$ can take a step or if $e_1$ is a value and $e_2$ can take a step, then rule \textsc{E-Congruence} applies to $e$, and $e$ can take a step.

\underline{\textit{Subcase $e_1$ and $e_2$ are values:}} If both $e_1$ and $e_2$ are values, i.e., they are locations $l_1$ and $l_2$ respectively, then by inversion on \textsc{T-Method}, we have \mbox{$\Gamma~|~\Sigma \vdash l_1 : \{ \varepsilon_1 \}~\{ x \Rightarrow \overline{\sigma} \}$} and\linebreak
\mbox{$\keyw{def}~ m(y : \tau_2) : \{ \varepsilon_3 \}~\tau_1 \in \overline{\sigma}$}. By inversion on \textsc{T-Loc}, we know that the store contains an appropriate mapping for the location $l_1$, and since \textsc{T-Store} has been applied throughout, the store is well typed and $l_1 \mapsto \{ x \Rightarrow \overline{d} \} \in \mu$ with $\keyw{def} m(y : \tau_1) : \{ \varepsilon_3 \}~\tau_2 = e \in \overline{d}$. Therefore, the rule \textsc{E-Method} applies to $e$, $e$ can take a step, and $\mu' = \mu$.
\\

\noindent\underline{\textit{Case \textsc{T-Field}:}}
$e = e_1.f$, and by the induction hypothesis, either $e_1$ can make a step of evaluation or it is a value. Then, there are two subcases:

\underline{\textit{Subcase $\langle e_1~|~\mu \rangle \longrightarrow \langle e_1'~|~\mu' \rangle$:}} If $e_1$ can take a step, then rule \textsc{E-Congruence} applies to $e$, and $e$ can take a step.

\underline{\textit{Subcase $e_1$ is a value:}} If $e_1$ is a value, i.e., a location $l$, then by inversion on \textsc{T-Field}, we have\linebreak
\mbox{$\Gamma~|~\Sigma \vdash l : \{ \varepsilon \}~\{ x \Rightarrow \overline{\sigma} \}$} and $\keyw{var}~ f : \tau \in \overline{\sigma}$. By inversion on \textsc{T-Loc}, we know that the store contains an appropriate mapping for the location $l$, and since \textsc{T-Store} has been applied throughout, the store is well typed and $l \mapsto \{ x \Rightarrow \overline{d} \} \in \mu$ with $\keyw{var} f : \tau = l_1 \in \overline{d}$. Therefore, the rule \textsc{E-Field} applies to $e$, $e$ can take a step, and $\mu' = \mu$.
\\

\noindent\underline{\textit{Case \textsc{T-Assign}:}}
$e = (e_1.f = e_2)$, and by the induction hypothesis, either $e_1$ is a value or else it can make a step of evaluation, and likewise $e_2$. Then, there are two subcases:

\underline{\textit{Subcase $\langle e_1~|~\mu \rangle \longrightarrow \langle e_1'~|~\mu' \rangle$ or $e_1$ is a value and $\langle e_2~|~\mu \rangle \longrightarrow \langle e_2'~|~\mu' \rangle$:}} If $e_1$ can take a step or if $e_1$ is a value and $e_2$ can take a step, then rule \textsc{E-Congruence} applies to $e$, and $e$ can take a step.

\underline{\textit{Subcase $e_1$ and $e_2$ are values:}} If both $e_1$ and $e_2$ are values, i.e., they are locations $l_1$ and $l_2$ respectively, then by inversion on \textsc{T-Assign}, we have $\Gamma~|~\Sigma \vdash l_1 : \{ \varepsilon_1 \}~\{ x \Rightarrow \overline{\sigma} \}$, $\keyw{var}~ f : \tau \in \overline{\sigma}$, and $\Gamma~|~\Sigma \vdash l_2 : \{ \varepsilon_2 \}~\tau$. By inversion on \textsc{T-Loc}, we know that the store contains an appropriate mapping for the locations $l_1$ and $l_2$, and since \textsc{T-Store} has been applied throughout, the store is well typed and $l_1 \mapsto \{ x \Rightarrow \overline{d} \} \in \mu$ with $\keyw{var} f : \tau = l \in \overline{d}$. A new well-typed store can be created as follows: $\mu' = [l_1 \mapsto \{ x \Rightarrow \overline{d}' \}/l_1 \mapsto \{ x \Rightarrow \overline{d} \}]\mu$, where $\overline{d}' = [\keyw{var} f : \tau = l_2/\keyw{var} f : \tau = l]\overline{d}$. Then, the rule \textsc{E-Assign} applies to $e$, and $e$ can take a step.
\\

\noindent\underline{\textit{Case \textsc{T-Sub}:}}
The result follows directly from the induction hypothesis.
\\

\noindent Thus, the program written in this language never gets stuck.
\end{proof}


\chapter{Type Safety Theorems for Algebraic Effects and Handlers}
\label{appendix-alg}
\section{Lemmas}
\begin{lemma} (Substitution) \\
If $\Gamma, x_j : \tau' \vdash c_i : \sigma$ and $\Gamma \vdash e_j : \tau'$, then $\Gamma \vdash \{e_j/x_j\}c_i : \sigma$, and\\\
If $\Gamma, x_j : \tau' \vdash e_i : \tau $ and $\Gamma \vdash e_j : \tau'$, then $\Gamma \vdash \{e_j/x_j\}e_i : \tau$

\begin{proof}
By rule induction on $\Gamma \vdash e : \tau$ and $\Gamma \vdash c : \sigma$
\begin{enumerate}[align=left]
\item[(T-Unit)] Trivial
\item[(T-Var)] Trivial
\item[(T-Lam)] 
$$
\infer[\textsc{(T-lam)}]
  {\Gamma, x_j : \tau'  \vdash (\lambda x_i: \tau.\ c_i) : \tau \rightarrow \sigma}
  {\Gamma,x_j : \tau',  x_i:\tau \vdash c_i : \sigma} 
$$
By IH, we have $\Gamma, x_i:\tau \vdash \{e_j/x_j\}c_i : \sigma$\\
Then by (T-Lam) we have $\Gamma \vdash (\lambda x_i : \tau.\ \{e_j/x_j\}c_i) : \sigma$.\\
Which is equivalent to $\Gamma \vdash \{e_j/x_j\}(\lambda x_i : \tau.\ c_i) : \sigma$.
\item[(T-EmbedExp)] By inversion and IH
\item[(T-Ret)] Follows by induction hypothesis
\item[(T-Op)]  
$$
\infer[\textsc{(T-op)}]
  {\Gamma\vdash op(e_i; y_i.c_i) : \{\varepsilon\}\tau }
  {\Sigma(op) = \tau_A \rightarrow \tau_B & \Gamma \vdash e_i : \tau_A & \Gamma. y_i:\tau_B\vdash c_i: \{\varepsilon\}\tau & op \in \Delta_i(\varepsilon)} 
$$

By inversion we have $\Gamma, x_j : \tau' \vdash e_i : \tau_A$ and $\Gamma, x_j : \tau', y_i : \tau_B \vdash c_i : \{\varepsilon\} \tau$. \\
Since we can make $y_i$ a fresh variable, we have $\Gamma,  y_i : \tau_B, x_j:\tau' \vdash c_i : \{\varepsilon\} \tau$.\\
Then by IH we have $\Gamma \vdash \{e_j/x_j\}e_i : \tau_A$ and $\Gamma,  y_i : \tau_B \vdash \{e_j/x_j\}c_i : \{\varepsilon\} \tau$.\\
By (T-Op) we have $\Gamma \vdash op(\{e_j/x_j\}e_i; y_i.\{e_j/x_j\}c_i) : \{\varepsilon\}\tau$\\
Therefore we have $\Gamma \vdash \{e_j/x_j\}(op(e_i; y_i.c_i)) : \{\varepsilon\}\tau$

\item[(T-Seq)] 
$$
\infer[\textsc{(T-seq)}]
  {\Gamma, x_j : \tau'' \vdash \m{do} x_i \leftarrow c_i \m{in} c_i' : \{\varepsilon\}\tau' }
  {\Gamma, x_j : \tau'' \vdash c_i : \{\varepsilon\}\tau & \Gamma, x_j : \tau'', x_i:\tau \vdash c_i': \{\varepsilon\}\tau'} 
  $$
By IH, we have $\Gamma \vdash \{e_j/x_j\}c_i : \{\varepsilon\}\tau$\\
Since we can choose $x_i$ as a fresh variable, we have $\Gamma, x_i : \tau, x_j : \tau'' \vdash c_i': \{\varepsilon\}\tau'$\\
Then by IH we have  $\Gamma, x_i : \tau \vdash \{e_j/x_j\}c_i': \{\varepsilon\}\tau'$\\
Then the result follows by (T-Seq)
\item[(T-App)] Follows directly by applying IH.
\item[(T-Handle)] 
$$
\infer[\textsc{(T-handle)}]
  {\Gamma, x_j : \tau' \vdash \m{with} h_i \m{handle} c_i: \{\varepsilon'\} \tau_B}
  {\begin{gathered}
  h_i = \{\m{return} x \mapsto c^r, op^1(x;k) \mapsto c^1, \dots, op^n(x;k) \mapsto c^n \}\\
  \Gamma, x_j : \tau' , x: \tau_A \vdash c^r : \{\varepsilon'\}\tau_B \\
  \left\{ \Sigma(op^i) = \tau_i \rightarrow \tau_i'  \quad \Gamma, x_j : \tau', x:\tau_i, k:\tau_i' \rightarrow \{\varepsilon'\}\tau_B \vdash c^i: \{\varepsilon'\}\tau_B    \right\}_{1 \leq i \leq n} \\
  \Gamma, x_j : \tau' \vdash c_i : \{\varepsilon\}\tau_A \quad \varepsilon \setminus \{op^i\}_{1 \leq i \leq n} \subseteq \varepsilon'
  \end{gathered}}
  $$
    Then handling clauses bind variables $x$ and $k$ in the handling computation $c^i$, so we can make them fresh variables that do not appear in context $\Gamma$. Then we can apply IH to typing judgements in the premise.
    
\item[(T-Embed)] Follows by applying IH
\item[(T-EmbedOp)] The proof is similar to  the case for (T-Op)

\end{enumerate}
\end{proof}
\end{lemma}

\begin{lemma} 
\label{lemma-exact}
If $\Gamma \vdash c_i : \{\varepsilon\}\tau$ then $\overline{\Delta_i}(\varepsilon) = \varepsilon$
\begin{proof}
By induction on derivation of $\Gamma \vdash c : \sigma$. (T-Ret) has a premise the ensures the lemma is correct. For other rules, the result is immediate by applying IH.

\end{proof}
\end{lemma}

\begin{lemma}
\label{lemma-minus}
If $\varepsilon \leq_l \varepsilon'$, then $\varepsilon \setminus op \leq_l \varepsilon' \setminus op$
\begin{proof}
By induction on $\varepsilon \leq_l \varepsilon$. The proof is straightforward.
\end{proof}

\end{lemma}

\begin{lemma}
\label{lemma-relation}
If $op \leq_{l} \varepsilon$, then $op \leq_{l} \varepsilon \setminus op'$
\begin{proof}
By induction on the derivation of $\varepsilon \leq_l \varepsilon$. If (R-Eff1) is used, then the proof is straightforward because the subset relation on the premise still holds. If (R-Eff2) is used, by inversion on (R-Eff2), we have $op \leq_l \varepsilon'$ and $\varepsilon' \leq_{l'} \varepsilon$. By IH we have $op \leq_l \varepsilon' \setminus op'$. By lemma \ref{lemma-minus} we have $\varepsilon'  \setminus op' \leq_{l'} \varepsilon \setminus op'$. Then the result follows by (R-Eff2)
\end{proof}
\end{lemma}

\begin{lemma}
\label{lemma-relation2}
If $\tau \leq_l \tau'$ then $\tau' \leq_{rev(l)} \tau$
\begin{proof}
By induction on the type relation rules. The proof consists of simple arguments that follow directly from IH.  
\end{proof}
\end{lemma}

\section{Preservation}

\subsubsection{Proof of lemma \ref{preservation-exp} (Preservation for expressions)}  
For all agent $i$, If $\Gamma \vdash e_i : \tau$ and $e_i \mapsto e_i'$, then $\Gamma \vdash e_i' : \tau$.
\begin{proof}
By induction on derivation of $e_i \mapsto e_i'$
\begin{enumerate}[align=left]
\item[(E-Congruence)] By inversion on the typing rule for embedded expressions, we have $\Gamma \vdash e_j : \tau'$. By IH, we have $\Gamma \vdash e_j' : \tau'$. Then we use (E-Contruence) to derive $\Gamma \vdash [e_j']^\tau_j : \tau$

\item[(E-Unit)] Follows immediately from  (T-Unit)
\item[(E-Lambda)] By inversion on (T-Embed), we have $\Gamma \vdash \lambda x_j:\tau'.\ c_j : \tau' \rightarrow \sigma'$, where $\tau' \rightarrow \sigma' \leq_{ji} \tau \rightarrow \sigma$.\\
By inversion on (R-Arrow), we have $\tau' \leq_{ji} \tau$ and $\sigma' \leq_{ji} \sigma$\\
By inversion on (T-Lambda), we have $\Gamma, x_j : \tau' \vdash c_j : \sigma'$. And since $x_i$ is a fresh variable in $c_j$, we have $\Gamma, x_i : \tau, x_j : \tau' \vdash c_j : \sigma'$\\
By lemma \ref{lemma-relation2}, we have $\tau \leq_{ij} \tau'$, and therefore $\Gamma, x_i : \tau \vdash [x_i]^{\tau'}_i : \tau'$\\
Then we can use the substitution lemma to derive $\Gamma, x_i : \tau \vdash \{[x_i]^{\tau'}_i / x_j\}c_j : \sigma'$.\\
Then by (T-Embed), we have $\Gamma, x_i : \tau \vdash [  \{[x_i]^{\tau'}_i / x_j\}c_j ]^\sigma_j : \sigma$\\
Then the result follows by (T-Lambda).

\end{enumerate}
\end{proof}

\subsubsection{Proof of lemma \ref{preservation-com} (Preservation for computations)}
If $\Gamma \vdash c_i : \{\varepsilon\} \tau$ and $c_i \longrightarrow c_i'$, then $\Gamma \vdash c_i' : \{\varepsilon\}\tau$

\begin{proof} (Sketch)
By induction on the derivation that $c_i \longrightarrow c_i'$. We proceed by the cases on the last step of the derivation.

\begin{enumerate}
\item E-Ret: By inversion, $\Gamma \vdash e_i: \tau$. By preservation of expressions and IH, we have  $\Gamma \vdash e_i': \tau$. Then we can use E-Ret to derive $\Gamma  \vdash c_i' : \{\varepsilon\} \tau$
\item E-Op: Follow immediately from inversion and IH
\item E-EmbedOp1: Follow immediately from inversion and IH
\item E-EmbedOp2:  
$$\infer[\textsc{(E-EmbedOp2)}]
{[op_j]^\varepsilon_l(v_i; y_i.c_i) \longrightarrow [op_j]^{\varepsilon''}_l(v_i; y_i.c_i) }
{\overline{\Delta_i}(\varepsilon) = \varepsilon''} \quad $$
We have the typing rule as follows:
$$\infer
  {\Gamma\vdash [op]^\varepsilon_l(e_i; y_i.c_i) : \{\overline{\Delta_i}(\varepsilon')\}\tau }
  {
  \Sigma(op) = \tau_A \rightarrow \tau_B & \Gamma \vdash e_i : \tau_A & \Gamma. y_i:\tau_B\vdash c_i: \{\varepsilon'\}\tau & \overline{\Delta_i}(\varepsilon) \subseteq \overline{\Delta_i}(\varepsilon') & \Gamma \vdash op \leq_{li} \varepsilon
  }  $$
  Since $\overline{\Delta_i}(\varepsilon'') = \varepsilon''$ and $\varepsilon'' = \overline{\Delta_i}(\varepsilon)$, we have $\overline{\Delta_i}(\varepsilon'')  \subseteq \overline{\Delta_i}(\varepsilon')$. Then we can use T-EmbedOp to derive $\Gamma\vdash [op]^{\varepsilon''}_l(e_i; y_i.c_i) : \{\overline{\Delta_i}(\varepsilon')\}\tau $
\item E-EmbedOp3:
  We have the typing rule as follows:
$$\infer
  {\Gamma\vdash [op]^\varepsilon_l(e_i; y_i.c_i) : \{\overline{\Delta_i}(\varepsilon')\}\tau }
  {\Sigma(op) = \tau_A \rightarrow \tau_B & \Gamma \vdash e_i : \tau_A & \Gamma. y_i:\tau_B\vdash c_i: \{\varepsilon'\}\tau & \overline{\Delta_i}(\varepsilon) \subseteq \overline{\Delta_i}(\varepsilon') & \Gamma \vdash op \leq_{li} \varepsilon}  $$
  By E-EmbedOp3, $op \in \overline{\Delta_i}(\varepsilon)$. So $op \in \overline{\Delta_i}(\varepsilon')$. 
By inversion on the typing rule, we have $\Gamma, y_i:\tau_B\vdash c_i: \{\varepsilon'\}\tau$. By lemma \ref{lemma-exact}, we have $\Gamma, y_i:\tau_B\vdash c_i: \{\overline{\Delta_i}(\varepsilon')\}\tau$. Then we can use T-Op to derive the designed result $\Gamma \vdash op(e_i; y_i.c_i) : \{\overline{\Delta_i}(\varepsilon')\}\tau $

\item E-EmbedOp4:
We have the typing rule as follows:
$$\infer
  {\Gamma\vdash [op]^\varepsilon_l(e_i; y_i.c_i) : \{\overline{\Delta_i}(\varepsilon')\}\tau }
  {\Sigma(op) = \tau_A \rightarrow \tau_B & \Gamma \vdash e_i : \tau_A & \Gamma. y_i:\tau_B\vdash c_i: \{\varepsilon'\}\tau & \overline{\Delta_i}(\varepsilon) \subseteq \overline{\Delta_i}(\varepsilon') & \Gamma \vdash op \leq_{li} \varepsilon}  $$
By lemma \ref{lemma-relation}, we have $op \leq_{li} \varepsilon \setminus op'$. By inversion on the typing rule, we have $\Gamma, y_i:\tau_B \vdash c_i:\{\varepsilon'\}\tau$ and $\varepsilon \subseteq \overline{\Delta_i}(\varepsilon')$.  So $\varepsilon \setminus op' \subseteq \overline{\Delta_i}(\varepsilon')$. By lemma \ref{lemma-exact}, we have $\Gamma, y_i:\tau_B \vdash c_i:\{\overline{\Delta_i}(\varepsilon')\}\tau$. Then we can apply T-EmbedOp again to derive $\Gamma\vdash [op]^{\varepsilon \setminus op'}_l(e_i; y_i.c_i) : \{\overline{\Delta_i}(\varepsilon')\}\tau $
  
\item E-App1: Follows immediately by T-App
\item E-App2: Follows immediately by T-App
\item E-App3: By inversion of T-App, we $\Gamma \vdash (\lambda x_i: \tau.\ c_i) : \tau \rightarrow \sigma$, $\Gamma \vdash v_i : \tau$. By inversion of T-Lam, $\Gamma, x_i:\tau \vdash c_i : \sigma$. By substitution lemma, we have $\Gamma \vdash \{v_i/x_i\}c_i : \sigma$.

\item E-Seq1:  Follows immediately by T-Seq and IH.
\item E-Seq2: By inversion on T-Seq, we have $\Gamma \vdash \m{return} v_i : \{\varepsilon\}\tau$ and $\Gamma, x_i:  \tau \vdash c_i': \{\varepsilon\}\tau'$. By inversion on T-Ret, we have $\Gamma \vdash v_i: \tau$. Then by substitution lemma we have $\Gamma \vdash \{v_i/x\}c_i' : \{\varepsilon\}\tau'$.
\item E-Seq3: 
$$
\infer[\textsc{(E-Seq3)}]
  {\texttt{do}\ x \leftarrow op_i(v_i; y_i.c_i) \texttt{in}\ c_i' \longrightarrow op_i(v_i; y_i. \m{do} x \leftarrow c_i \m{in} c_i')}
  {} $$
  By inversion of T-Seq, we have $\Gamma \vdash op_i(v_i; y_i.c_i) : \{\varepsilon\}\tau$  and $\Gamma , x:\tau \vdash c_i' : \{\varepsilon\}\tau'$. By inversion on T-OP, we have $\Gamma, y_i: \tau_B \vdash c_i: \{\varepsilon\}\tau$ and $op \in \varepsilon$ and $\Gamma \vdash v_i : \tau_A$.  Then by T-Seq, we have $\Gamma, y_i : \tau_B \vdash \m{do} x \leftarrow c_i \m{in} c_i' : \{\varepsilon\}\tau'$. Then we can use T-Op to derive $\Gamma \vdash op_i(v_i; y_i. \m{do} x \leftarrow c_i \m{in} c_i') : \{\varepsilon\}\tau'$.

\item E-Seq4
$$
\infer[\textsc{(E-Seq4)}]
  {\texttt{do}\ x \leftarrow [op_j]^\varepsilon_l(v_i; y_i.c_i) \texttt{in}\ c_i' \longrightarrow [op_j]^\varepsilon_l(v_i; y_i. \m{do} x \leftarrow c_i \m{in} c_i')}
  {\Delta_i(\varepsilon) = \varepsilon & op \not \in \varepsilon} $$
$$
\infer[\textsc{(T-seq)}]
  {\Gamma\vdash \m{do} x_i \leftarrow c_i \m{in} c_i' : \{\varepsilon'\}\tau' }
  {\Gamma \vdash c_i : \{\varepsilon'\}\tau & \Gamma, x_i:\tau \vdash c_i': \{\varepsilon'\}\tau'}  $$
By inversion on T-Seq, we have $\Gamma \vdash [op_j]^\varepsilon_l(v_i; y_i.c_i): \{\varepsilon'\}\tau$ and $\Gamma, x:\tau \vdash c_i' : \{\varepsilon'\} \tau'$. Then by inversion on T-EmbedOp, we have $\Gamma, y_i:\tau_B \vdash c_i: \{\varepsilon'\}\tau$, $\overline{\Delta_i}(\varepsilon) \subseteq \varepsilon'$.
Then by T-Seq, we have $\Gamma, y_i:\tau_B \vdash \m{do} x \leftarrow c_i \m{in} c_i' : \{\varepsilon'\}\tau'$. Then by T-EmbedOp, we have $\Gamma \vdash [op]^\varepsilon_l(v_i; y_i.  \m{do} x \leftarrow c_i \m{in} c_i' ): \{\varepsilon'\}\tau'$

\item E-Handle1: Follows immediately by inversion on T-Handle and IH
\item E-Handle2: By T-Handle, we have $\Gamma \vdash \m{with} h_i \m{handle} \m{return} v_i :  \{\varepsilon'\}\tau_B$. By inversion on T-Handle, we have $\Gamma, x_i:\tau_A \vdash c_i' : \{\varepsilon\}\tau_B$, and $\Gamma \vdash \m{return} v_i : \{\varepsilon\} \tau_A$. By inversion on T-Ret, we have $\Gamma \vdash v_i : \tau_A$. Then by substitution lemma, we have $\Gamma \vdash \{v_i/x_i\}c_i' : \{\varepsilon'\}\tau_B$. 

\item E-Handle3
$$
\infer  {\m{with} h_i \m{handle} op(v_; y_i.c_i) \longrightarrow \{v_i/x_i\} \{(\lambda y_i: \tau_i'.
   \m{with} h_i\m{handle} c_i) / k\}c_i' }
  {op(x_i; k) \mapsto c_i' \in h_i & \Sigma(op) = \tau_i \rightarrow \tau_i'} 
$$
By T-Handle, we have $\Gamma \vdash \m{with} h_i \m{handle} op(v;y_i.c_i) :  \{\varepsilon'\}\tau_B$. By inversion on T-Handle, we have $\Gamma, x_i:\tau_i, k: \tau_i' \rightarrow \{\varepsilon'\}\tau_B \vdash c_i' : \{\varepsilon'\} \tau_B$, and $\Gamma \vdash op(v; y_i.c_i) : \{\varepsilon\} \tau_A$. By inversion on T-Op, we have $\Gamma \vdash v_i : \tau_i$ and $\Gamma, y_i: \tau_i' \vdash c_i: \{\varepsilon\}\tau_A$. By T-Handle, we have $\Gamma, y_i: \tau_i' \vdash \m{with} h_i \m{handle} c_i : \{\varepsilon'\}\tau_B$. Then by T-Lam, we have $\Gamma \vdash \lambda y_i:\tau_i'.\ \m{with} h_i \m{handle} c_i : \tau_i' \rightarrow  \{\varepsilon'\}\tau_B$. Then, by substitution lemma, we have $\Gamma \vdash \{v_i/x_i\} \{(\lambda y_i: \tau_i'.
   \m{with} h_i\m{handle} c_i) / k\}c_i' : \{\varepsilon'\}\tau_B$.
   
\item E-Handle4:
$$
\infer[\textsc{(E-Handle4)}]
  {\m{with} h_i \m{handle} [op]^\varepsilon_l(v_i, y_i.c_i) \longrightarrow [op]^\varepsilon_l(v_i; y_i. \m{with} h_i \m{handle} c_i))}
  {\Delta_i(\varepsilon) = \varepsilon & op \not \in \varepsilon} 
$$

$$
\infer[\textsc{(T-handle)}]
  {\Gamma\vdash \m{with} h_i \m{handle} c_i: \{\varepsilon'\} \tau_B}
  {\begin{gathered}
  h_i = \{\m{return} x \mapsto c^r, op^1(x;k) \mapsto c^1, \dots, op^n(x;k) \mapsto c^n \}\\
  \Gamma, x: \tau_A \vdash c^r : \{\varepsilon'\}\tau_B \quad \\
  \left\{ \Sigma(op^i) = \tau_i \rightarrow \tau_i'  \quad \Gamma, x:\tau_i, k:\tau_i' \rightarrow \{\varepsilon'\}\tau_B \vdash c^i: \{\varepsilon'\}\tau_B    \right\}_{1 \leq i \leq n} \\
  \Gamma \vdash c_i : \{\varepsilon''\}\tau_A \quad \varepsilon'' \setminus \{op^i\}_{1 \leq i \leq n} \subseteq \varepsilon'
  \end{gathered}}
$$
By T-Handle, we have $\Gamma \vdash \m{with} h_i \m{handle} [op]^\varepsilon_l(v;y_i.c_i) :  \{\varepsilon'\}\tau_B$. By inversion on T-Handle, we have $\Gamma \vdash [op]^\varepsilon_l(v;y_i.c_i): \{\varepsilon''\}\tau_A$ and $\varepsilon'' \setminus \{op^i\} \subseteq \varepsilon'$. By inversion on T-EmbedOp, we have $\Gamma \vdash v_i: \tau_i$, $\Gamma, y_i:\tau_i' \vdash c_i : \{\varepsilon''\}\tau_A$ and $\varepsilon \subseteq \varepsilon''$. Since $\varepsilon$ doesn't contain any concrete operation, we have $\varepsilon \subseteq  \varepsilon'' \setminus \{op^i\} \subseteq \varepsilon' $. Then by T-Handle, we have $\Gamma, y_i: \tau_i' \vdash \m{with} h_i \m{handle} c_i : \{\varepsilon'\}\tau_B$. Then, we use T-EmbedOp to derive $\Gamma \vdash [op]^\varepsilon_l(v_i; y_i. \m{with} h_i \m{handle} c_i) : \{\varepsilon'\}\tau_B$


\item E-Embed1: Follows immediately from Inversion and IH
\item E-Embed2: By typing rule, we have $\Gamma \vdash [\m{return} v_j]^{\{\varepsilon\}\tau}_l :\{\varepsilon\}\tau$. By inversion on the typing rule, we have $\Gamma \vdash \m{return} v_j : \{\varepsilon'\}\tau'$ such that $\{\varepsilon\}\tau' \leq_{li} \{\varepsilon\}\tau$. By inversion on R-Sigma, we have $\tau' \leq_{li} \tau$. Then by T-EmbedExp, we have $\Gamma \vdash [v_j]^\tau_l : \tau$. Then by T-Ret, we have $\Gamma \vdash \m{return}  [v_j]^\tau_l : \{\varepsilon\}\tau$. $\Gamma \vdash [\m{return} v_j]^{\{\varepsilon\}\tau}_l :\{\varepsilon\}\tau$

\item E-Embed3: 
$$
\infer[\textsc{(E-Embed3)}]
{[op(v_j; y_j.c_j)]^{\{\varepsilon\}\tau}_l \longrightarrow [op]^\varepsilon_l([v_j]^{\tau_A}_j; y_i. \{[y_i]^{\tau_B}_i / y_j\}[c_j]^{\{\varepsilon\}\tau}_l)}
{ \Sigma(op) = \tau_A \rightarrow \tau_B} 
$$
By typing rule, we have $\Gamma \vdash op(v_j; y_j.c_j) : \{\varepsilon'\}\tau'$, where $\{\varepsilon'\}\tau' \leq_{li} \{\varepsilon\}\tau$. By inversion on T-Op, we have $\Gamma \vdash v_j : \tau_A$, and $\Gamma, y_j : \tau_B \vdash c_j: \{\varepsilon'\}\tau'$. Then, by T-EmbedExp, we have $\Gamma \vdash [v_j]^{\tau_A}_j : \tau_A$. By substitution lemma, we have $\Gamma, y_i: \tau_B \vdash \{[y_i]^{\tau_B}_i/y_j\}c_j : \{\varepsilon'\}\tau'$. By T-Embed, we have $\Gamma, y_i: \tau_B \vdash \{[y_i]^{\tau_B}_i/y_j\}[c_j]^{\{\varepsilon\}\tau}_l : \{\varepsilon\}\tau$. Then we can use T-EmbedOp to derive $\Gamma \vdash [op]^\varepsilon_l([v_j]^{\tau_A}_j; y_i. \{[y_i]^{\tau_B}_i / y_j\}[c_j]^{\{\varepsilon\}\tau}_l): \{\varepsilon\}\tau$. 

\item E-Embed4:
$$
\infer[\textsc{(E-Embed4)}]
{[[op_k]^{\varepsilon'}_{l'}(v_j; y_j.c_j)]^{\{\varepsilon\}\tau}_l \longrightarrow [op_k]^\varepsilon_{l'jl}([v_j]^{\tau_A}_j; y_i. \{[y_i]^{\tau_B}_i / y_j\}[c_j]^{\{\varepsilon\}\tau}_l)}
{ \Sigma(op_k) = \tau_A \rightarrow \tau_B& \Delta_j(\varepsilon') = \varepsilon' & op \not \in \varepsilon'} 
$$
By typing rule, we have $\Gamma \vdash [op_K]^{\varepsilon'}_{l'}(v_j; y_j.c_j) : \{\varepsilon''\}\tau''$, where $\{\varepsilon''\}\tau'' \leq_{li} \{\varepsilon\}\tau$. By inversion on T-EmbedOp, we have $\Gamma \vdash v_j: \tau_A$ and $\Gamma, y_j : \tau_B \vdash c_j: \{\varepsilon''\}\tau''$. Then, by T-EmbedExp, we have $\Gamma \vdash [v_j]^{\tau_A}_j : \tau_A$.
By substitution lemma, we have $\Gamma, y_i: \tau_B \vdash \{[y_i]^{\tau_B}_i/y_j\}c_j : \{\varepsilon''\}\tau''$. By T-Embed, we have $\Gamma, y_i: \tau_B \vdash \{[y_i]^{\tau_B}_i/y_j\}[c_j]^{\{\varepsilon\}\tau}_l : \{\varepsilon\}\tau$. Then we use T-EmbedOp to derive $\Gamma \vdash [op_k]^\varepsilon_{l'jl}([v_j]^{\tau_A}_j; y_i. \{[y_i]^{\tau_B}_i / y_j\}[c_j]^{\{\varepsilon\}\tau}_l) : \{\varepsilon\}\tau$.
\end{enumerate}
\end{proof}

\section{Progress}

\subsubsection{Proof of lemma \ref{progress-exp} (Progress for expressions)}
For agent i, if $\varnothing \vdash e_i : \tau$ then either $e_i = v_i$ or $e_i \longrightarrow e_i'$.
\begin{proof}
By induction on structure of $e_i$.
\begin{enumerate}[align=left]
\item[ \underline{Case $e_i = ()$}:] $e_i$ is already a value.
\item[ \underline{Case $e_i = \lambda x : \tau. c$} :] $e_i$ is already a value.
\item[ \underline{Case $e_i = [e_j]^\tau_j$} :] By IH, either $e_j$ is a j-value, or $e_j \longrightarrow e_j'$. If $e_j \longrightarrow e_j'$, then by (E-Congruence), $[e_j]^\tau_j \longrightarrow [e_j']^\tau_j$. If $e_j$ is a value, then it is either $()$ or $\lambda x_j : \tau' . c_j$. So $e_i$ can be evaluated by (E-Unit) and (E-Lambda) correspondingly.
\end{enumerate}
\end{proof}

\subsubsection{Proof of lemma \ref{progress-com} (Progress for computation)}
If $\varnothing \vdash c_i : \{\varepsilon\} \tau$ then either
\begin{enumerate}
\item  $c_i \longrightarrow c_i'$ 
\item  $c_i = \m{return} v_i$
\item $c_i = op(v_i; y_i.c_i')$
\item $c_i = [op]^\varepsilon_l(v_i; y_i.c_i')$ and  $op \not\in \varepsilon$
 \end{enumerate}
\begin{proof}
By induction on structure of $c_i$.
\begin{enumerate}[align=left]
\item[ \underline{Case $c_i = \m{return} e_i$} :] Immediate by applying IH on $e_i$.
\item[ \underline{Case $c_i = op(e_i, y_i.c_i')$} :] Immediate by applying IH on $e_i$.
\item[ \underline{Case $c_i = [c_j]^\sigma_j$} :] By IH on $c_j$, $c_j$ can either evaluates to another computation, or be a return statement, an operation call, or an embedded operation call. Then we can apply (E-Embed) rules to evaluate $c_i$ accordingly.
 \item[ \underline{Case $c_i = [op]^\varepsilon_l(e_i, y_i.c_i')$ :}] Follows directly by applying IH on $e_i$.
  \item[ \underline{Case $c_i =e_i\ e_i'$ :}] If $e_i$ or $e_i'$ are not values, then (E-App1) or (E-App2) can be applied to $c_i$. Otherwise, (E-App3) could be applied.
  \item[ \underline{Case $c_i = \m{do} x \leftarrow c_i' \m{in} c_i''$ :}] Follows directly from applying IH on $c_i'$.  
    \item[ \underline{Case $c_i = \m{with} h_1 \m{handle} c_i'$ :}] Follows directly from applying IH on $c_i'$.  

\end{enumerate}
\end{proof}





\backmatter

%\renewcommand{\baselinestretch}{1.0}\normalsize

% By default \bibsection is \chapter*, but we really want this to show
% up in the table of contents and pdf bookmarks.
\renewcommand{\bibsection}{\chapter{\bibname}}
%\newcommand{\bibpreamble}{This text goes between the ``Bibliography''
%  header and the actual list of references}
\bibliographystyle{plainnat}
\bibliography{related} %your bib file

\end{document}
