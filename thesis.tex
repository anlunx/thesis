%for a more compact document, add the option openany to avoid
%starting all chapters on odd numbered pages
\documentclass[12pt]{cmuthesis}

% This is a template for a CMU thesis.  It is 18 pages without any content :-)
% The source for this is pulled from a variety of sources and people.
% Here's a partial list of people who may or may have not contributed:
%
%        bnoble   = Brian Noble
%        caruana  = Rich Caruana
%        colohan  = Chris Colohan
%        jab      = Justin Boyan
%        josullvn = Joseph O'Sullivan
%        jrs      = Jonathan Shewchuk
%        kosak    = Corey Kosak
%        mjz      = Matt Zekauskas (mattz@cs)
%        pdinda   = Peter Dinda
%        pfr      = Patrick Riley
%        dkoes = David Koes (me)

% My main contribution is putting everything into a single class files and small
% template since I prefer this to some complicated sprawling directory tree with
% makefiles.

% some useful packages
\usepackage{times}
\usepackage{fullpage}
\usepackage{graphicx}
\usepackage{amsmath}
\usepackage[numbers,sort]{natbib}
\usepackage[backref,pageanchor=true,plainpages=false, pdfpagelabels, bookmarks,bookmarksnumbered,
%pdfborder=0 0 0,  %removes outlines around hyper links in online display
]{hyperref}
\usepackage{subfigure}
\usepackage{listings}
\usepackage{setspace}

% Approximately 1" margins, more space on binding side
%\usepackage[letterpaper,twoside,vscale=.8,hscale=.75,nomarginpar]{geometry}
%for general printing (not binding)
\usepackage[letterpaper,twoside,vscale=.8,hscale=.75,nomarginpar,hmarginratio=1:1]{geometry}


\newcommand{\li}[1]{\lstinline{#1}}


% Provides a draft mark at the top of the document. 
\draftstamp{\today}{DRAFT}

\begin {document} 
\frontmatter

%initialize page style, so contents come out right (see bot) -mjz
\pagestyle{empty}

\title{ %% {\it \huge Thesis Proposal}\\
{\bf Extending Abstract Effects with Bounds and Algebraic Handlers}}
\author{Anlun Xu}
\date{November 2020}
\Year{2020}
\trnumber{}

\committee{
\ 
}

\support{}
\disclaimer{}

% copyright notice generated automatically from Year and author.
% permission added if \permission{} given.

\keywords{Effect Systems}

\maketitle

\begin{dedication}
For my dog
\end{dedication}

\pagestyle{plain} % for toc, was empty


\begin{abstract}
Effect systems have been a subject of active research for nearly four decades, with the most notable practical example being checked exceptions in programming languages such as Java. While many exception systems support abstraction, aggregation, and hierarchy (e.g., via class declaration and subclassing mechanisms), it is rare to see such expressive power in more generic effect systems. We designed an effect system around the idea of protecting system resources and incorporated our effect system into the Wyvern programming language. Similar to type members, a Wyvern object can have effect members that can abstract lower-level effects, allow for aggregation, and have both lower and upper bounds, providing for a granular effect hierarchy. We argue that Wyvern's effects capture the right balance of expressiveness and power from the programming language design perspective. We present a full formalization of our effect-system design, show that it allows reasoning about authority and attenuation.  Our approach is evaluated through a security-related case study.\end{abstract}
\begin{acknowledgments}
My advisor is cool.
\end{acknowledgments}



\tableofcontents
\listoffigures
\listoftables

\mainmatter

%% Double space document for easy review:
%\renewcommand{\baselinestretch}{1.66}\normalsize

% The other requirements Catherine has:
%
%  - avoid large margins.  She wants the thesis to use fewer pages, 
%    especially if it requires colour printing.
%
%  - The thesis should be formatted for double-sided printing.  This
%    means that all chapters, acknowledgements, table of contents, etc.
%    should start on odd numbered (right facing) pages.
%
%  - You need to use the department standard tech report title page.  I
%    have tried to ensure that the title page here conforms to this
%    standard.
%
%  - Use a nice serif font, such as Times Roman.  Sans serif looks bad.
%
% Other than that, just make it look good...
\doublespacing

% !TEX root = thesis.tex

\chapter{Introduction}


Effect systems have been a subject of active research for nearly four decades, with the most notable practical example being checked exceptions in programming languages such as Java. According to \citet{filinski10}, there are two different views on modeling computational effects in programs: the denotational approach and the restrictive approach.

The denotational approach describes how effectful programs can be translated into a pure program, which can then be evaluated using the standard semantics for pure programs. Works by \citet{moggi89} shows that features from imperative computations, such as exceptions or mutable states, can be mimicked by monads in a pure program. Algebraic effects and handlers \cite{plotkin09} are the the latest development in this strand of works. Algebraic effects and handlers can express a wide range of computation effects such as nondeterminism, concurrency, state, and input/output \cite{plotkin09}. Comparing to the traditional approach that uses general monads, algebraic effects have the advantage of being freely composable. Therefore, algebraic effects have recently been gaining popularity as an approach to model effects in the purely functional setting.

Alternatively, in a restrictive setting of computational effects, effects are considered to be built into the language. Rather than building up new behaviors, the effect systems aim to classify and restrict the use of existing effectful behavior in a language, such as reads and writes to memory, as well as checked exceptions. Restrictive effect systems are widely used for reasoning about security~\cite{turbak08}, memory effects~\cite{lucassen88}, and concurrency~\cite{bocchino09,bracevac18,dolan17}. \\

\noindent\textbf{Abstraction: A requirement for scalable effect systems}

 Unfortunately, effect systems have not been widely adopted, other than checked exceptions in Java, a feature that is widely viewed as problematic~\cite{10.1145/1103845.1094847}.  The root of the problem is that existing effect systems do not provide adequate support for scaling to programs that are larger and have complex structure.  Any adequate solution must support \textit{effect abstraction} and \textit{effect composition}.

Abstraction is key to achieving scale in general, and a principal form of abstraction is abstract types~\cite{10.1145/44501.45065}. 
There are many existing works that achieve information hiding using abstract types, such as SML signatures and abstract type members in Scala~\cite{odersky05}. 
Typically, a module system allows each module to choose what names and entities to export, and what to keep hidden. The exported interface typically does not reveal details of the implementation of a module. By hiding the implementation details, the programmer of the module can be certain that the invariants within the module cannot be broken by the client. Analogously to type abstraction, we define \textit{effect abstraction} as the ability to define higher-level effects in terms of lower-level effects, and potentially to \textit{hide} that definition from clients of effects.  

In large-scale systems, abstraction should be \textit{composable}.  For example, a database component might abstract \li{file.Read} further, exposing it as a higher-level \li{db.Query} effect to clients.  Clients of the database should be oblivious to whether \li{db.Query} is implemented in terms of a \li{file.Read} effect or a \li{network.Access} effect (in the case that the backend is a remote database). \\

\noindent\textbf{Design of a restrictive effect system in Wyvern}

This thesis presents a novel and scalable effect system design that supports bounded effect abstraction, extending the effect system presented by \citet{melicher20}.  The abstraction facility of our effect-system is inspired by type members in languages such as Scala. Just as Scala objects may define type members, in our effect calculus, any object may define one or more \textit{effect members}.  An effect member defines a new effect in terms of the lower-level effects that are used to implement it.  The set of lower-level effects may be empty in the base case or may include low-level effects that are hard-coded in the system.  Type ascription can enable information hiding by concealing the definition of an effect member from the containing object's clients. In addition to completely concealing the definition of an effect, our calculus provides bounded abstraction, which exposes upper or lower bounds of the definition of an effect, while still hiding the definition of it.  


\textit{Effect polymorphism} is a form of parametric polymorphism that allows functions or types to be implemented generically for handling computations with different effects~\cite{lucassen88}. In systems at a larger scale, there are various possible effects, and each program component may cause different effects. With effect polymorphism, we can write general code that handles objects with different effects, thereby reducing the amount of replicated code. In practice, we have found that to make effects work well with modules, it is essential to extend effect polymorphism by assigning bounds to effect parameters. We therefore introduce \textit{bounded abstract effects}, which allows programmers to define upper and lower bounds both on abstract effects and on polymorphic effect parameters.

 Just as Scala's type members can be used to encode parametric polymorphism over types, our effect members and their bounds double as a way to provide bounded effect polymorphism. We follow numerous prior Scala formalisms in including polymorphism via this encoding rather than explicitly; this keeps the formal system simpler without giving up expressive power. \\


\noindent\textbf{Design of a denotational effect system with effect abstraction}  

This thesis presents a core calculus that supports algebraic effects. The calculus extends the simply typed lambda calculus with algebraic effect operations and handlers and provides the ability to define abstract algebraic effects. Similar to the restrictive effect system, algebraic effects types can be defined in terms of lower-level effects. Effect abstraction in this system ensures that the client of an abstract effect type is not aware of the lower-level effects that implements the abstract effect. Consequently, the client of an abstract effect type would not be able to handle the computation that causes the abstract effect, whose operations are hidden.

Different from the restrictive effect system in Wyvern, which describes the built in effectful behavior in the language and does not affect the dynamic semantics of the program, the dynamic semantics of algebraic effects operations depend on the handler that encapsulates it during evaluation.  Therefore, the effect system needs to ensure the abstraction does not break during the evaluation of a program. This problem was originally discovered by \citet{biernacki19}, who solved the problem using the technique of coercions. In this paper, we propose the technique of agent-based reasoning, which was originally designed by \citet{grossman00}, as a solution of the problem. The benefit of this approach is that by explicitly dividing modules with hidden information into agents, the system supports syntactic proof for effect-abstraction properties \\

\noindent\textbf{Outline and Contributions.}  
Chapter \ref{chapter-background} introduces the background for both restrictive and denotational effect systems, and discusses the basics of the Wyvern effect system, after which we describe the main contributions of our paper:
\begin{itemize}
\item A design of a more expressive effect system for Wyvern. Specifically, ours is the first system to provide the programmer with a general form of bounded effect polymorphism and bounded effect abstraction, supporting upper and lower bounds that are other arbitrary effects.  (Section~\ref{sec-bound});
\item A precise, formal description of our effect system, and proof of its soundness.  Our formal system shows how to generalize and enrich earlier work on path-dependent effects by leveraging the type theory of DOT \cite{amin14}. (Section~\ref{sec-form});
\item A multi-agent calculus  that supports abstraction for algebraic effects, and proof of its type soundness theorems;  Our system enables a syntactic proof of the effect-abstraction property. (Section~\ref{sec-core});
\item A multi-agent calculus extended with existential effect types that demonstrates how multi-agent calculus interacts with traditional techniques of type abstraction. (Section~\ref{sec-exist});
\end{itemize}
The last chapter in the thesis discusses related work and concludes.

\chapter{Conclusion}

%\appendix
%% !TEX root = thesis.tex
\pagebreak


\appendix
\chapter{Proof of Theorem \ref{theorem-transitivity} (Transitivity of Subtyping)}
\section{Lemmas}
\begin{lemma}
If $\Gamma, x : \tau \vdash \varepsilon_1 <: \varepsilon_2$, and $\Gamma \vdash \tau' <: \tau$, then $\Gamma, x : \tau' \vdash \varepsilon_1 <: \varepsilon_2$.
\begin{proof}
The proof is by structural induction on the rule to derive $\Gamma, x : \tau \vdash \varepsilon_1 <: \varepsilon_2$. \begin{enumerate}
\item Subeffect-Subset\\
Since the premise doesn't rely on the context. This case is trivially true.
\item Subeffect-Upperbound\\
If the type of $n$ is not changed, then we can apply the same rule to to derive $\Gamma, x : \tau' \vdash \varepsilon_1 \cup \{n.g\}<: \varepsilon_2$. If the type of $n$ is replaced by $\tau'$, then we have $\keyw{effect}\ g \leqslant$ $\varepsilon' \in \sigma$, where $\Gamma, n: \tau' \vdash \varepsilon' <: \varepsilon$. By IH, we have $\Gamma, n : \tau' \vdash [n/y]\varepsilon \cup \varepsilon_1 <: \varepsilon_2$. By transitivity of subeffecting, we have $\Gamma, n : \tau' \vdash [n/y]\varepsilon' \cup \varepsilon_1 <: \varepsilon_2$. Then we can apply Subeffect-Upperbound again to derive $\Gamma, x : \tau' \vdash \varepsilon_1 \cup \{n.g\}<: \varepsilon_2$.
\item Subeffect-Lowerbound\\ This case is similar to Subeffect-Upperbound
\item Subeffect-Def-1\\ Since the declaration type $\keyw{effect} g = \{\varepsilon\}$ is not changed, the result follows directly by induction hypothesis.
\item Subeffect-Def-2\\ Since the declaration type $\keyw{effect} g = \{\varepsilon\}$ is not changed, the result follows directly by induction hypothesis.
\end{enumerate}
\end{proof}
\label{lemma-context-effect}
\end{lemma}

\begin{lemma}
If $\Gamma, x:\tau \vdash \tau_1 <: \tau_2$, and $\Gamma \vdash \tau' <: \tau$, then $\Gamma, x:\tau' \vdash \tau_1 <: \tau_2$\\
If $\Gamma, x:\tau \vdash \sigma_1 <: \sigma_2$, and $\Gamma \vdash \tau' <: \tau$, then $\Gamma, x:\tau' \vdash \sigma_1 <: \sigma_1$

\begin{proof}
We induct on the number of S-Alg used to derive the typing judgment in the premise of the statement.
\begin{itemize}
\item[BC] S-Alg is not used, so we have $\Gamma, x:\tau \vdash \sigma_1 <: \sigma_2$ derived by S-Refl2 or one of the S-Effect rules. The proof is trivial if we apply lemma \ref{lemma-context-effect}.
\item[IS1] Assume we used S-Alg n times to derive $\Gamma, x:\tau \vdash \{ y \Rightarrow \sigma_i^{i\in1\dots m}\} <: \Gamma \vdash \{y \Rightarrow {\sigma'}_i^{i\in1\dots n}\}$. Then for each subtyping judgments in the premise of S-Alg, we can apply induction hypothesis to derive $\Gamma,~x:\tau',~y : \{ y \Rightarrow {\sigma}_i^{i \in 1..m} \} \vdash \sigma_{p(i)}<: \sigma_i'$. Then by applying S-Alg, we have $\Gamma, x:\tau' \vdash \{ y \Rightarrow \sigma_i^{i\in1\dots m}\} <: \Gamma \vdash \{y \Rightarrow {\sigma'}_i^{i\in1\dots n}\}$
\item[IS2] Assume we used S-Alg n times to derive $\Gamma, y:\tau \vdash \keyw{def} m(x : \tau_1) : \{ \varepsilon_1 \}~\tau_2 <: \keyw{def} m(x : \tau_1') : \{ \varepsilon_2 \}~\tau_2'$, by inversion on S-Def, we have $\Gamma, y:\tau \vdash \tau_1' <: \tau_1$, $\Gamma, y:\tau \vdash \tau_2 <: \tau_2'$, and $\Gamma, y:\tau, x:\tau_1 \vdash \varepsilon_1 <: \varepsilon_2$. Then by induction hypothesis and lemma \ref{lemma-context-effect}, we have $\Gamma, y:\tau' \vdash \tau_1' <: \tau_1$, $\Gamma, y:\tau' \vdash \tau_2 <: \tau_2'$, and $\Gamma, y:\tau', x:\tau_1 \vdash \varepsilon_1 <: \varepsilon_2$. Then we use S-Def to derive $  \Gamma, y:\tau' \vdash \keyw{def} m(x : \tau_1) : \{ \varepsilon_1 \}~\tau_2 <: \keyw{def} m(x : \tau_1') : \{ \varepsilon_2 \}~\tau_2'$
\end{itemize}
\end{proof}
\label{lemma-context-type}
\end{lemma}



\section{Proof of theorem \ref{theorem-transitivity}}
If $\Gamma \vdash \tau_1 <: \tau_2$ and $\Gamma \vdash \tau_2 <: \tau_3$, then $\Gamma \vdash \tau_1 <: \tau_3$. \\
If $\Gamma \vdash \sigma_1 <: \sigma_2$ and $\Gamma \vdash \sigma_2 <: \sigma_3$, then $\Gamma \vdash \sigma_1 <: \sigma_3$. 
\begin{proof}
We induct on the the number of S-Alg used to derive the two judgments in the premise of the first statement: $\Gamma \vdash \tau_1 <: \tau_2$ and $\Gamma \vdash \tau_2 <: \tau_3$, or the two judgments in the premise of the second statement: $\Gamma \vdash \sigma_1 <: \sigma_2$ and $\Gamma \vdash \sigma_2 <: \sigma_3$.
\begin{itemize}
\item[BC] The S-Alg is not used, so we have $\Gamma \vdash \sigma_1 <: \sigma_2$ and $\Gamma \vdash \sigma_2 <: \sigma_3$ by S-Refl2 or one of S-Effect.  By lemma \ref{lemma-trans-effect} transitivity of subeffecting, it is easy to see $\Gamma \vdash \sigma_1 <: \sigma_3$

\item[IS1] Assume we used S-Alg $n$ times to derive $\Gamma \vdash \{x \Rightarrow \sigma_i^{i\in1...m}\} <:  \{x \Rightarrow {\sigma'}_i^{i\in1...n}\} $ and \mbox{$\Gamma \vdash \{x \Rightarrow {\sigma'}_i^{i\in1...n}\} <:  \{x \Rightarrow {\sigma''}_i^{i\in1...k}\}$}. By inversion of S-Alg, there is an injection $p: \{1..n\} \mapsto  \{1..m\}$ such that $\forall i \in 1..n,\ \Gamma, x: \{x \Rightarrow {\sigma}_i^{i\in 1..m} \} \vdash \sigma_{p(i)} <: \sigma'_i$. There is another injection $q: \{1..k\} \mapsto \{1..n\}$ such that $\forall i \in 1..k,\ \Gamma, x: \{x \Rightarrow {\sigma'}_i^{i\in 1..n} \} \vdash \sigma'_{q(i)} <: \sigma''_i$. So for each $i \in 1..k$ we have two judgments 
\begin{align*}
\Gamma, x: \{x \Rightarrow {\sigma}_i^{i\in 1..m} \} &\vdash \sigma_{p(q(i))} <: \sigma'_{q(i)}\\
\Gamma, x: \{x \Rightarrow {\sigma'}_i^{i\in 1..n} \} &\vdash \sigma'_{q(i)} <: \sigma''_i
\end{align*}
By lemma \ref{lemma-context-type}, we can write the second judgment as $\Gamma, x: \{x \Rightarrow {\sigma}_i^{i\in 1..m} \} \vdash \sigma'_{q(i)} <: \sigma''_{i}$. By IH, for all $i \in 1..k$, $ \Gamma, x: \{x \Rightarrow {\sigma''}_i^{i\in 1..k} \} \vdash \sigma_{p(q(i))} <: \sigma''_i$. Since the  function $p \circ q$ is a bijection from $\{1..k\} \mapsto \{1..n\}$, we can use the rule S-Alg again to derive $\Gamma \vdash \{x \Rightarrow \sigma_i^{i\in1...m}\} <:  \{x \Rightarrow {\sigma''}_i^{i\in1...k}\} $ 
\item[IS2] Assume we used S-Alg $n$ times to derive 
$\Gamma \vdash \keyw{def} m(x : \tau_1) : \{ \varepsilon_1 \}~\tau_1' <: \keyw{def} m(x : \tau_2) : \{ \varepsilon_2 \}~\tau_2'$
and
$\Gamma \vdash \keyw{def} m(x : \tau_2) : \{ \varepsilon_2 \}~\tau_2' <: \keyw{def} m(x : \tau_3) : \{ \varepsilon_3 \}~\tau_3'$. By inverse on S-Def, we have $\Gamma \vdash \tau_2 <: \tau_1$, $\Gamma \vdash \tau_3 <: \tau_2$, $\Gamma \vdash \tau_1' <: \tau_2'$, and $\Gamma \vdash \tau_2' <: \tau_3'$. By IH, we have $\Gamma \vdash \tau_1' <: \tau_3'$ and $\Gamma \vdash \tau_3 <: \tau_1$. We have $\Gamma \vdash \varepsilon_1 <: \varepsilon_3$ by transitivity of subeffects. Hence we can use S-Def again to derive $\Gamma \vdash \keyw{def} m(x : \tau_1) : \{ \varepsilon_1 \}~\tau_1' <: \keyw{def} m(x : \tau_3) : \{ \varepsilon_3 \}~\tau_3'$. 
\item[IS3] By transitivity of subeffecting, other cases for $\Gamma \vdash \sigma_1 <: \sigma_3$ are trivial. 

\end{itemize}
\end{proof}


\chapter{Proofs of the Type Soundness Theorems for Bounded Abstract Effects}
\label{app-effects-type-soundness}

\section{Lemmas}

%\begin{lemma}[Permutation]
%If \mbox{$\Gamma~|~\varnothing \vdash e : \{ \varepsilon \}~\tau$} and $\Delta$ is a permutation of $\Gamma$, then\linebreak
%\mbox{$\Delta~|~\varnothing \vdash e : \{ \varepsilon \}~\tau$}, and the latter derivation has the same depth as the former.
%\end{lemma}

\begin{proof}
Straightforward induction on typing derivations.
\end{proof}


\begin{lemma}[Weakening]
If $\Gamma~|~\varnothing \vdash e : \{ \varepsilon \}~\tau$ and $x \not\in dom(\Gamma)$, then $\Gamma,~x : \tau'~|~\varnothing \vdash e : \{ \varepsilon \}~\tau$, and the latter derivation has the same depth as the former.
\end{lemma}

\begin{proof}
Straightforward induction on typing derivations.
\end{proof}


\sloppy
\begin{lemma}[Reverse of \textsc{Subeffecting-Lowerbound}] \label{lemma-reverse1}
If $\Gamma \vdash \varepsilon_1 <: \varepsilon_2 \cup \{x.g\}$
, $\Gamma \vdash x : \{y \Rightarrow \sigma\}$, and
$\keyw{effect}\ g \leqslant \varepsilon \in \sigma$
then 
$\Gamma \vdash \varepsilon_1 <: \varepsilon_2 \cup [x/y]\varepsilon$
\end{lemma}
\begin{proof}
We prove this by induction on $size(\varepsilon_1 \cup \varepsilon_2 \cup \{x.g\})$, which is defined in Fig. \ref{f-size}
\begin{enumerate}
    \item[BC] If $size(\varepsilon_1 \cup \varepsilon_2 \cup \{x.g\}) = 0$. Then $x.g$ can not have a definition. This case is vacuously true.
    \item[IS] We case on the rule used to derive  $\Gamma \vdash \varepsilon_1  <: \varepsilon_2 \cup \{x.g\}$:
    \begin{enumerate}
        \item  $\Gamma \vdash \varepsilon_1  <: \varepsilon_2 \cup \{x.g\}$ is derived by Subeffect-Subset: If $x.g \not\in \varepsilon_1$, then we can use Subeffect-Subset to show 
        $\Gamma \vdash \varepsilon_1  <: \varepsilon_2 \cup [x/y]\varepsilon$
        If $x.g \in \varepsilon_1$. Then $\varepsilon_1 = \varepsilon_1' \cup \{x.g\}$, where $\varepsilon_1' \subseteq \varepsilon_2$. So we can use Subeffect-Def-1 to show
        $\Gamma \vdash \varepsilon_1' \cup \{x.g\}  <: \varepsilon_2 \cup [x/y]\varepsilon$
        \item $\Gamma \vdash \varepsilon_1  <: \varepsilon_2 \cup \{x.g\}$ is derived by Subeffect-Upperbound:\\
        Then we have 
        $\varepsilon_1 = \varepsilon_1' \cup \{z.h\}$,
        $\Gamma \vdash z : \{y' \Rightarrow\sigma\}$, 
        $\keyw{effect}\ h = \{ \varepsilon'\} \in \sigma$,
        and 
        \mbox{$\Gamma \vdash \varepsilon_1' \cup [z/y']\varepsilon' <: \varepsilon_2 \cup \{x.g\}$}
        By IH, we have 
        $\Gamma \vdash \varepsilon_1' \cup [z/y']\varepsilon' <: \varepsilon_2 \cup [x/y]\varepsilon$
        Using Subeffect-Upperbound, we have 
        $\Gamma \vdash \varepsilon_1' \cup \{z.h\} <: \varepsilon_2 \cup [x/y]\varepsilon$
        \item $\Gamma \vdash \varepsilon_1  <: \varepsilon_2 \cup \{x.g\}$ is derived by Subeffect-Def-1:\\
        If Subeffect-Def-1 uses the effect ${x.g}$, then we immediately have
        $\Gamma \vdash \varepsilon_1 <: \varepsilon_2 \cup [x/y]\varepsilon$
        Otherwise, if Subeffect-Def-1 doesn't use $x.g$, then we have 
        \mbox{$\varepsilon_2 = \varepsilon_2' \cup \{z.h\}$},
        \mbox{$\Gamma \vdash z : \{y' \Rightarrow\sigma\}$},
        \mbox{$\keyw{effect}\ h = \{ \varepsilon'\} \in \sigma$},
        and 
        \mbox{$\Gamma \vdash \varepsilon_1  <: \varepsilon_2' \cup [z/y']\varepsilon' \cup \{x.y\}$}.
        By IH, we have
        \mbox{$\Gamma \vdash \varepsilon_1  <: \varepsilon_2' \cup [z/y']\varepsilon' \cup [x/y]\varepsilon$}.
        Using Subeffect-Def-1, we have
        \mbox{$\Gamma \vdash \varepsilon_1<: \varepsilon_2 \cup [x/y]\varepsilon$}
        \item \mbox{$\Gamma \vdash \varepsilon_1  <: \varepsilon_2 \cup \{x.g\}$} is derived by Subeffect-Def-2:\\ This case is similar to (b)
    \end{enumerate}
\end{enumerate}
\end{proof}



\begin{lemma}[Reverse of \textsc{Subeffecting-Def-2}] \label{lemma-reverse2}
If $\Gamma \vdash \varepsilon_1 \cup \{x.g\} <: \varepsilon_2$
, $\Gamma \vdash x : \{y \Rightarrow \sigma\}$, and
$\keyw{effect}\ g = \{\varepsilon\} \in \sigma$
then 
$\Gamma \vdash \varepsilon_1 \cup [x/y]\varepsilon <: \varepsilon_2$
\end{lemma}

\begin{proof}
We prove this by induction on $size(\varepsilon_1 \cup \varepsilon_2 \cup \{x.g\})$, which is defined in Fig. \ref{f-size}
\begin{enumerate}
    \item[BC] If $size(\varepsilon_1 \cup \varepsilon_2 \cup \{x.g\}) = 0$. Then $x.g$ can not have a definition. This case is vacuously true.
    \item[IS] We case on the rule used to derive  $\Gamma \vdash \varepsilon_1 \cup \{x.g\} <: \varepsilon_2$:
    \begin{enumerate}
        \item  $\Gamma \vdash \varepsilon_1 \cup \{x.g\} <: \varepsilon_2$ is derived by Subeffect-Subset:\\
        Then $x.g \in \varepsilon_2$. So we can use Subeffect-Def-1 to derive $\Gamma \vdash \varepsilon_1 \cup [x/y]\varepsilon <: \varepsilon_2$
        \item $\Gamma \vdash \varepsilon_1 \cup \{x.g\} <: \varepsilon_2$ is derived by Subeffect-Upperbound:\\
        If the Subeffect-Upperbound rule uses the effect $x.g$, then we by the premise of Subeffect-Upperbound, we have
        $\Gamma \vdash \varepsilon_1 \cup [x/y]\varepsilon <: \varepsilon_2$
        If the Subeffect-Upperbound rule does not use the effect $x.g$, then we have 
        $\varepsilon_1 = \varepsilon_1' \cup \{z.h\}$,
        $\Gamma \vdash z : \{y' \Rightarrow\sigma\}$,
        $\keyw{effect}\ h \leqslant \varepsilon' \in \sigma$,
        and 
        $\Gamma \vdash \varepsilon_1' \cup [z/y']\varepsilon' \cup \{x.g\} <: \varepsilon_2$
        By IH, we have
        $\Gamma \vdash \varepsilon_1' \cup [z/y']\varepsilon' \cup [x/y]\varepsilon <: \varepsilon_2$.
        Using Subeffect-Upperbound, we derive
        $\Gamma \vdash \varepsilon_1 \cup [x/y]\varepsilon <: \varepsilon_2$.
        \item $\Gamma \vdash \varepsilon_1 \cup \{x.g\} <: \varepsilon_2$ is derived by Subeffect-Def-1:\\
        Then we have 
        $\varepsilon_2 = \varepsilon_2' \cup \{z.h\}$,
        $\Gamma \vdash z : \{y' \Rightarrow\sigma\}$,
        $\keyw{effect}\ h = \{ \varepsilon'\} \in \sigma$,
        and 
        \mbox{$\Gamma \vdash \varepsilon_1 \cup \{x.g\} <: \varepsilon_2' \cup [z/y']\varepsilon'$}.
        By IH, we have
        \mbox{$\Gamma \vdash \varepsilon_1 \cup [x/y]\varepsilon <: \varepsilon_2' \cup [z/y']\varepsilon'$}.
        Using Subeffect-Def-1, we have
        $\Gamma \vdash \varepsilon_1 \cup [x/y]\varepsilon <: \varepsilon_2 \cup \{z.h\}$.
        \item $\Gamma \vdash \varepsilon_1 \cup \{x.g\} <: \varepsilon_2$ is derived by Subeffect-Def-2:\\ This case is similar to (b)
    \end{enumerate}
\end{enumerate}
\end{proof}





\begin{lemma}[Transitivity in subeffecting]
If $\Gamma \vdash \varepsilon_1 <: \varepsilon_2$ and $\Gamma \vdash \varepsilon_2 <: \varepsilon_3$, then $\Gamma \vdash \varepsilon_1 <: \varepsilon_3$.
\label{lemma-trans-effect}
\end{lemma}
\begin{proof}
We prove this using structural induction on $size(\Gamma, \varepsilon_1 \cup \varepsilon_2 \cup \varepsilon_3)$, which is defined in Fig. \ref{f-size}

\begin{enumerate}
\item[BC] Let $size(\Gamma, \varepsilon_1 \cup \varepsilon_2 \cup \varepsilon_3) = 0$. The judgments $\Gamma \vdash \varepsilon_1 <: \varepsilon_2$ and $\Gamma \vdash \varepsilon_2 <: \varepsilon_3$ are derived from Subeffect-Subset. So we have transitivity immediately.
\item[IS] Let $N \geq 0$, assume $\forall \varepsilon_1, \varepsilon_2, \varepsilon_3$ with $size(\Gamma, \varepsilon_1 \cup \varepsilon_2 \cup \varepsilon_3) \leq N$, if $\varepsilon_1 <: \varepsilon_2$ and $\varepsilon_2 <: \varepsilon_3$, then $\varepsilon_1 <: \varepsilon_3$. Let $\Gamma \vdash \varepsilon_1 <: \varepsilon_2$ and $\Gamma \vdash \varepsilon_2 <: \varepsilon_3$ and $size(\Gamma, \varepsilon_1 \cup \varepsilon_2 \cup \varepsilon_3) = N+1$. Want to show $\varepsilon_1 <: \varepsilon_3$. We case on the rules used to derive $\Gamma \vdash \varepsilon_1 <: \varepsilon_2$ and $\Gamma \vdash \varepsilon_2 <: \varepsilon_3$
\begin{enumerate}
    \item $\Gamma \vdash \varepsilon_1 <: \varepsilon_2$ by Subeffect-Subset
    \begin{enumerate}
        \item $\Gamma \vdash \varepsilon_2 <: \varepsilon_3$ by Subeffect-Subset. \\
        Transitivity in this case is trivial.
        \item $\Gamma \vdash \varepsilon_2 <: \varepsilon_3$ by Subeffect-Upperbound. \\
        Let $\varepsilon_2 = \varepsilon_2' \cup \{x.g\}$. By Subeffect-Upperbound, we have $\Gamma \vdash x : \{y \Rightarrow \sigma\}$ $\keyw{effect}\ g \leqslant \varepsilon \in \sigma$ and $\varepsilon_2'\cup [x/y]\varepsilon <: \varepsilon_3$ There are two cases:
        \begin{enumerate}
            \item  If $\{x.g\} \not\in \varepsilon_1$, then $\varepsilon_1 \subseteq \varepsilon_2'$. Therefore $\Gamma \vdash \varepsilon_1 <: \varepsilon_2'\cup [x/y]\varepsilon$. By induction hypothesis, we have $\Gamma \vdash \varepsilon_1 <: \varepsilon_3$. 
            \item  If $\{x.g\} \in \varepsilon_1$, then $\varepsilon_1 = \varepsilon_1'\cup \{x.g\}$, and $\varepsilon_1' \subseteq \varepsilon_2'$. So we have $\Gamma \vdash \varepsilon_1'\cup [x/y]\varepsilon <: \varepsilon_2'\cup [x/y]\varepsilon$ by Subeffect-Subset. By IH, we have $\varepsilon_1'\cup[x/y]\varepsilon <: \varepsilon_3$. Then we use Subeffect-Upperbound to derive $\varepsilon_1' \cup \{x.g\} <: \varepsilon_3$
        \end{enumerate}       
        \item 
         $\Gamma \vdash \varepsilon_2 <: \varepsilon_3$ by Subeffect-Def-1.\\
        Let $\varepsilon_3 = \varepsilon_3' \cup \{x.g\}$. We have
        $\Gamma \vdash x : \{y \Rightarrow \sigma\}$,
        $\keyw{effect}\ g = \{\varepsilon\}$,
        and
        $\Gamma \vdash \varepsilon_2 <: \varepsilon_3' \cup [x/y]\varepsilon$.
        By IH, we have $\Gamma \vdash \varepsilon_1 <: \varepsilon_3' \cup [x/y]\varepsilon$
        Then we can use Subeffect-Def-1 again to derive $\Gamma \vdash \varepsilon_1 <: \varepsilon_3$
        

        \item $\Gamma \vdash \varepsilon_2 <: \varepsilon_3$ by Subeffect-Def-2.\\
        The proof is identical to ii.
    \end{enumerate}
    \item 
    $\Gamma \vdash \varepsilon_1 <: \varepsilon_2$ by Subeffect-Upperbound. \\
    So we have $\varepsilon_1 = \varepsilon_1' \cup \{x.g\}$.
    $\Gamma \vdash x : \{y \Rightarrow \sigma\}$,
    $\keyw{effect}\ g = \{\varepsilon\}$,
    and
    $\Gamma \vdash \varepsilon_1' \cup [x/y]\varepsilon <: \varepsilon_2$.
    Using IH, we have 
     $\Gamma \vdash \varepsilon_1' \cup [x/y]\varepsilon <: \varepsilon_3$.
     Using Suveffect-Upperbound again, we have 
     $\Gamma \vdash \varepsilon_1 <: \varepsilon_3$.

    \item $\Gamma \vdash \varepsilon_1 <: \varepsilon_2$ by Subeffect-Def-1. \\
    Therefore we let $\varepsilon_2 = \varepsilon_2' \cup \{x.g\}$, $\Gamma\vdash x:\{y \Rightarrow \sigma\}$, and $effect\ g = \{\varepsilon\} \in \sigma$. By premise of Subeffect-Def-1, we have $\Gamma \vdash \varepsilon_1 <: [x/y]\varepsilon \cup \varepsilon_2'$. Since $\Gamma \vdash \varepsilon_2 <: \varepsilon_3$, we have \mbox{$\Gamma \vdash \varepsilon_2' \cup \{x.g\} <: \varepsilon_3$}. 
    \begin{enumerate}
        \item $\Gamma \vdash \varepsilon_2' \cup \{x.g\} <: \varepsilon_3$ by Subeffect-Subset\\
        Then we have $\varepsilon_3 = \varepsilon_3' \cup \{x.g\}$, and $\varepsilon_2' \subseteq \varepsilon_3'$. Therefore we have $\varepsilon_2' \cup [x/y]\varepsilon \subseteq \varepsilon_3' \cup [x/y] \varepsilon$. Therefore, $\Gamma \vdash\varepsilon_2' \cup [x/y]\varepsilon  <: \varepsilon_3' \cup [x/y] \varepsilon $. By IH, we have $\Gamma \vdash \varepsilon_1 <: \varepsilon_3' \cup [x/y] \varepsilon$. By Subeffect-Def-1 ,we have $\Gamma\vdash \varepsilon_1 <: \varepsilon_3' \cup \{x.g\} = \varepsilon_3$
        \item $\Gamma \vdash \varepsilon_2 <: \varepsilon_3$ by Subeffect-Upperbound\\
      Since the effect $\{x.g\}$ is used by Subeffect-Def-1, it is not used by the rule Subeffect-Upperbound. Let $\varepsilon_2 = \varepsilon_2'' \cup \{x.g\} \cup \{z.h\}$. By Subeffect-Def-1, we have \mbox{$\Gamma \vdash \varepsilon_1 <: \varepsilon_2'' \cup [x/y]\varepsilon \cup \{z.h\}$}. By Subeffect-Upperbound, we have 
      $\Gamma \vdash z:\{y' \Rightarrow \sigma'\}$,
      $\keyw{effect}\ h \leqslant \varepsilon' \in \sigma'$,
      and $\Gamma \vdash \varepsilon_2'' \cup \{x.g\} \cup [z/y']\varepsilon' <: \varepsilon_3$.
      By Lemma \ref{lemma-reverse1} and $\Gamma \vdash \varepsilon_1 <: \varepsilon_2'' \cup [x/y]\varepsilon \cup \{z.h\}$ , we have 
      $\Gamma \vdash \varepsilon_1 <: \varepsilon_2''\cup [x.y]\varepsilon \cup [z/y']\varepsilon'$.
      By Lemma \ref{lemma-reverse2} and $\Gamma \vdash \varepsilon_2'' \cup \{x.g\} \cup [z/y']\varepsilon' <: \varepsilon_3$, we have 
      $\Gamma \vdash \varepsilon_2'' \cup [x/y]\varepsilon \cup [z/y']\varepsilon' <: \varepsilon_3$.
      Therefore, we use IH to derive $\Gamma \vdash \varepsilon_1 <: \varepsilon_3$.
      
        
        \item $\Gamma \vdash \varepsilon_2 <: \varepsilon_3$ by Subeffect-Def-1\\
        Therefore, let $\varepsilon_3 = \varepsilon_3' \cup \{z.h\}$, $\Gamma \vdash z:\{y \Rightarrow \sigma'\}$, and $effect\ h = \{\varepsilon'\} \in \sigma'$. And we have $\Gamma \vdash \varepsilon_2 <: \varepsilon_3' \cup \{z.h\}$. By premise of Subeffect-Def-1, we have $\Gamma \vdash \varepsilon_2 <: [z/y]\varepsilon'\cup\varepsilon_3'$. By IH, we have $\Gamma \vdash \varepsilon_1 <: [z/y]\varepsilon'\cup\varepsilon_3'$. Using Subeffect-Def-1, we derive that $\Gamma \vdash \varepsilon_1 <: \varepsilon_3$. 
        \item $\Gamma \vdash \varepsilon_2 <: \varepsilon_3$ by Subeffect-Def-2\\
        This case is identical to c (ii) 
    \end{enumerate}
    \item $\Gamma \vdash \varepsilon_1 <: \varepsilon_2$ by Subeffect-Def-2\\
    This case is identical to (b)
    \item $\Gamma \vdash \varepsilon_1 <: \varepsilon_2$ by Subeffect-Lowerbound\\
    This case is identical to (c)
\end{enumerate}

\end{enumerate}
\end{proof}



\begin{lemma}[Substitution in types]
If \mbox{$\Gamma,~z : \tau \vdash \tau_1 <:  \tau_2$} and \mbox{$\Gamma~|~\Sigma \vdash l : \{ \}~[l/z]\tau$}, then\linebreak
\mbox{$\Gamma \vdash [l/z]\tau_1 <: [l/z]\tau_2$}. Furthermore, if \mbox{$\Gamma,~z : \tau \vdash \sigma_1 <: \sigma_2$} and \mbox{$\Gamma~|~\Sigma \vdash l : \{ \}~[l/z]\tau$}, then\linebreak
\mbox{$\Gamma \vdash [l/z]\sigma_1 <: [l/z]\sigma_2$}.
\label{lemma-substitution-types}
\end{lemma}

\begin{proof} The proof is by simultaneous induction on a derivation of \mbox{$\Gamma,~z : \tau \vdash \tau_1 <:  \tau_2$} and \linebreak
    \mbox{$\Gamma,~z : \tau \vdash \sigma_1 <: \sigma_2$}. For a given derivation, we proceed by cases on the final typing rule used in the derivation:\\

\noindent\underline{\textit{Case \textsc{S-Refl1}:}} \mbox{$\tau_1 = \tau_2$}, and the desired result is immediate.\\

\noindent\underline{\textit{Case \textsc{S-Trans}:}} By inversion on \textsc{S-Trans}, we get \mbox{$\Gamma,~z : \tau \vdash \tau_1 <: \tau_2$} and \mbox{$\Gamma,~z : \tau \vdash \tau_2 <: \tau_3$}. By the induction hypothesis, \mbox{$\Gamma \vdash [l/z]\tau_1 <: [l/z]\tau_2$} and \mbox{$\Gamma \vdash [l/z]\tau_2 <: [l/z]\tau_3$}. Then, by\linebreak
\mbox{\textsc{S-Trans}}, \mbox{$\Gamma \vdash [l/z]\tau_1 <: [l/z]\tau_3$}.\\

\noindent\underline{\textit{Case \textsc{S-Perm}:}} \mbox{$\tau_1 = \{ x \Rightarrow \sigma_i^{i \in 1..n} \}$} and \mbox{$\tau_2 = \{ x \Rightarrow \sigma_i'^{i \in 1..n} \}$}. Substitution preserves the permutation relations, and thus, \mbox{$[l/z]\{ x \Rightarrow \sigma_i^{i \in 1..n} \}$} is a permutation of \mbox{$[l/z]\{ x \Rightarrow \sigma_i'^{i \in 1..n} \}$}. Then, by \textsc{S-Perm}, \mbox{$\Gamma \vdash [l/z]\{ x \Rightarrow \sigma_i^{i \in 1..n} \} <: [l/z]\{ x \Rightarrow \sigma_i'^{i \in 1..n} \}$}.\\

\noindent\underline{\textit{Case \textsc{S-Width}:}} \mbox{$\tau_1 = \{ x \Rightarrow \sigma_i^{i \in 1..n + k} \}$} and \mbox{$\tau_2 = \{ x \Rightarrow \sigma_i^{i \in 1..n} \}$}, and the desired result is immediate.\\

\noindent\underline{\textit{Case \textsc{S-Depth}:}} \mbox{$\tau_1 = \{ x \Rightarrow \sigma_i^{i \in 1..n} \}$} and \mbox{$\tau_2 = \{ x \Rightarrow {\sigma'}_i^{i \in 1..n} \}$}.  By inversion on \textsc{S-Depth}, \linebreak we get \mbox{$\forall i,~\Gamma,~x : \{ x \Rightarrow {\sigma}_i^{i \in 1..n} \},~z : \tau \vdash \sigma_i <: \sigma_i'$}. By the induction hypothesis, \linebreak
\mbox{$\forall i,~\Gamma,~x : \{ x \Rightarrow {\sigma}_i^{i \in 1..n} \} \vdash [l/z]\sigma_i <: [l/z]\sigma_i'$}. Then, by \textsc{S-Depth}, \linebreak
\mbox{$\Gamma \vdash [l/z]\{ x \Rightarrow \sigma_i^{i \in 1..n} \} <: [l/z]\{ x \Rightarrow \sigma_i'^{i \in 1..n} \}$}.\\


\noindent\underline{\textit{Case \textsc{S-Refl2}:}} \mbox{$\sigma_1 = \sigma_2$}, and the desired result is immediate.\\

\sloppy
\noindent\underline{\textit{Case \textsc{S-Def}:}} \mbox{$\sigma_1 = \keyw{def} m(x : \tau_1) : \{ \varepsilon_1 \}~\tau_2$} and \mbox{$\sigma_2 = \keyw{def} m(x : \tau_1') : \{ \varepsilon_2 \}~\tau_2'$}. By inversion on \textsc{S-Def}, we get \mbox{$\Gamma,~z : \tau \vdash \tau_1' <: \tau_1$}, \mbox{$\Gamma,~z : \tau \vdash \tau_2 <: \tau_2'$}, \mbox{$\Gamma, z : \tau \vdash \varepsilon_1 <: \varepsilon_2$}. By the induction hypothesis, \mbox{$\Gamma \vdash [l/z]\tau_1' <: [l/z]\tau_1$} and \mbox{$\Gamma \vdash [l/z]\tau_2 <: [l/z]\tau_2'$}. By lemma \ref{lemma-sub-effects}, \mbox{$\Gamma \vdash [l/z]\varepsilon_1 <: [l/z]\varepsilon_2$}. Then, by \textsc{S-Def}, \mbox{$\Gamma \vdash [l/z](\keyw{def} m(x : \tau_1) : \{ \varepsilon_1 \}~\tau_2) <: [l/z](\keyw{def} m(x : \tau_1') : \{ \varepsilon_2 \}~\tau_2')$}.\\

\noindent\underline{\textit{Case \textsc{S-Effect}:}} \mbox{$\sigma_1 = \keyw{effect} g = \{ \varepsilon \}$} and \mbox{$\sigma_2 = \keyw{effect} g$}, and the desired result is immediate.\\

\noindent Thus, substituting terms in types preserves the subtyping relationship.
\end{proof}


\begin{lemma}[Substitution in expressions and effects]
\label{lemma-sub-effects}
If \mbox{$\Gamma,~z : \tau'~|~\Sigma \vdash e :  \{ \varepsilon \}~\tau$} and \mbox{$\Gamma~|~\Sigma \vdash l : \{ \}~[l/z]\tau'$}, then \mbox{$\Gamma~|~\Sigma \vdash [l/z]e :  \{ [l/z]\varepsilon \}~[l/z]\tau$}. \\[3ex] And if $\Gamma, z : \tau' \mid \Sigma \vdash \varepsilon_1 <: \varepsilon_2$ and $\Gamma \mid \Sigma \vdash l : \{\} [l/z]\tau$, then $\Gamma \mid \Sigma \vdash [l/z] \varepsilon_1 <: [l/z] \varepsilon_2$.\\[3ex]
And if \mbox{$\Gamma,~z : \tau'~|~\Sigma \vdash d : \sigma$} and
\mbox{$\Gamma~|~\Sigma \vdash l : \{ \}~[l/z]\tau'$}, then $\Gamma~|~\Sigma \vdash [l/z]d : [l/z]\sigma$. \\[3ex]
Furthermore, if $\Gamma, z : \tau' \mid \Sigma \vdash \varepsilon\ {wf}$,  then 
$\Gamma \mid \Sigma \vdash [l/z]\varepsilon\ {wf}$
\end{lemma}

\begin{proof} The proof is by simultaneous induction on a derivation of $\Gamma,~z : \tau'~|~\Sigma \vdash e : \{ \varepsilon \}~\tau$, $\Gamma,~z : \tau'~|~\Sigma \vdash d : \sigma$, $\Gamma, z :\tau' \mid \Sigma \vdash \varepsilon_1 <: \varepsilon_1$, and $\Gamma, z : \tau' \mid \Sigma \vdash \varepsilon wf$. For a given derivation, we proceed by cases on the final typing rule used in the derivation:\\

\noindent\underline{\textit{Case \textsc{T-Var}:}} $e = x$, and by inversion on \textsc{T-Var}, we get \mbox{$x : \tau \in (\Gamma,~z : \tau')$}. There are two subcases to consider, depending on whether $x$ is $z$ or another variable. If $x = z$, then $[l/z]x = l$ and $\tau = \tau'$. The required result is then $\Gamma~|~\Sigma \vdash l : \{ \}~[l/z]\tau'$, which is among the assumptions of the lemma. Otherwise, $[l/z]x = x$, and the desired result is immediate.\\

\noindent\underline{\textit{Case \textsc{T-New}:}} \mbox{$e = \keywadj{new}(x \Rightarrow \overline{d})$}, and by inversion on \textsc{T-New}, we get\linebreak
\mbox{$\forall i,~d_i \in \overline{d},~\sigma_i \in \overline{\sigma},~\Gamma,~x : \{ x \Rightarrow \overline{\sigma} \},~z : \tau'~|~\Sigma \vdash d_i : \sigma_i$}. By the induction hypothesis,
\mbox{$\forall i,~d_i \in \overline{d},~\sigma_i \in \overline{\sigma},~\Gamma,~x : \{ x \Rightarrow \overline{\sigma} \}~|~\Sigma \vdash [l/z]d_i : [l/z]\sigma_i$}. Then, by \textsc{T-New},
\mbox{$\Gamma~|~\Sigma \vdash \keywadj{new}(x \Rightarrow [l/z]\overline{d}) : \{ \}~\{ x \Rightarrow [l/z]\overline{\sigma} \}$}, i.e.,
\mbox{$\Gamma~|~\Sigma \vdash [l/z](\keywadj{new}(x \Rightarrow \overline{d})) : \{ \}~[l/z]\{ x \Rightarrow \overline{\sigma} \}$}.\\

\noindent\underline{\textit{Case \textsc{T-Method}:}} \mbox{$e = e_1.m(e_2)$}, and by inversion on \textsc{T-Method}, we get
\mbox{$\Gamma,~z : \tau'~|~\Sigma \vdash e_1 : \{ \varepsilon_1 \}~\{ x \Rightarrow \overline{\sigma} \}$}; \mbox{$\keyw{def}~ m(y : \tau_2) : \{ \varepsilon_3 \}~\tau_1 \in \overline{\sigma}$};
\mbox{$\Gamma,~z : \tau'~|~\Sigma \vdash [e_1/x][e_2/y]\varepsilon_3~\mathit{wf}$}; and $\Gamma,~z : \tau'~|~\Sigma \vdash e_2 : \{ \varepsilon_2 \}~[e_1/x]\tau_2$. By the induction hypothesis, \mbox{$\Gamma~|~\Sigma \vdash [l/z]e_1 : \{ [l/z]\varepsilon_1 \}~[l/z]\{ x \Rightarrow \overline{\sigma} \}$},
\mbox{$\keyw{def}~ m(y : [l/z]\tau_2) : \{ [l/z]\varepsilon_3 \}~[l/z]\tau_1 \in [l/z]\overline{\sigma}$}, \mbox{$\Gamma~|~\Sigma \vdash [l/z]([e_1/x][e_2/y]\varepsilon_3)~\mathit{wf}$}, and
\mbox{$\Gamma~|~\Sigma \vdash [l/z]e_2 : \{ [l/z]\varepsilon_2 \}~[l/z][e_1/x]\tau_2$}. Then, by \mbox{\textsc{T-Method}},
\mbox{$\Gamma~|~\Sigma \vdash [l/z]e_1.m([l/z]e_2) : \{ [l/z]\varepsilon_1 \cup [l/z]\varepsilon_2 \cup [l/z]([e_1/x][e_2/y]\varepsilon_3) \}~[l/z]([e_1/x][e_2/y]\tau_1)$},
i.e., \mbox{$\Gamma~|~\Sigma \vdash [l/z](e_1.m(e_2)) : \{ [l/z](\varepsilon_1 \cup \varepsilon_2 \cup [e_1/x][e_2/y]\varepsilon_3) \}~[l/z]([e_1/x][e_2/y]\tau_1)$}.\\

\noindent\underline{\textit{Case \textsc{T-Field}:}} $e = e_1.f$, and by inversion on \textsc{T-Field}, we get $\Gamma,~z : \tau'~|~\Sigma \vdash e_1 : \{ \varepsilon \}~\{ x \Rightarrow \overline{\sigma} \}$ and \mbox{$\keyw{var}~ f : \tau \in \overline{\sigma}$}. By the induction hypothesis, $\Gamma~|~\Sigma \vdash [l/z]e_1 : \{ [l/z]\varepsilon \}~[l/z]\{ x \Rightarrow \overline{\sigma} \}$ and \mbox{$\keyw{var}~ f : [l/z]\tau \in [l/z]\overline{\sigma}$}. Then, by \textsc{T-Field}, $\Gamma~|~\Sigma \vdash ([l/z]e_1).f : \{ [l/z]\varepsilon \}~[l/z]\tau$, i.e., $\Gamma~|~\Sigma \vdash [l/z](e_1.f) : \{ [l/z]\varepsilon \}~[l/z]\tau$.\\

\noindent\underline{\textit{Case \textsc{T-Assign}:}} \mbox{$e = (e_1.f = e_2)$}, and by inversion on \textsc{T-Assign}, we get
\mbox{$\Gamma,~z : \tau'~|~\Sigma \vdash e_1 : \{ \varepsilon_1 \}~\{ x \Rightarrow \overline{\sigma} \}$}; $\keyw{var}~ f : \tau \in \overline{\sigma}$; and \mbox{$\Gamma,~z : \tau'~|~\Sigma \vdash e_2 : \{ \varepsilon_2 \}~\tau$}. By the induction hypothesis,  \mbox{$\Gamma~|~\Sigma \vdash [l/z]e_1 : \{ [l/z]\varepsilon_1 \}~[l/z]\{ x \Rightarrow \overline{\sigma} \}$}; \mbox{$\keyw{var}~ f : [l/z]\tau \in [l/z]\overline{\sigma}$};  and \mbox{$\Gamma~|~\Sigma \vdash [l/z]e_2 : \{ [l/z]\varepsilon_2 \}~[l/z]\tau$}. Then, by \textsc{T-Assign}, 
\mbox{$\Gamma~|~\Sigma \vdash [l/z]e_1.f = [l/z]e_2 : \{ [l/z]\varepsilon_1 \cup [l/z]\varepsilon_2\}~[l/z]\tau$}, i.e.,
\mbox{$\Gamma~|~\Sigma \vdash [l/z](e_1.f=e_2) : \{ [l/z](\varepsilon_1 \cup \varepsilon_2) \}~[l/z]\tau$}.\\

\noindent\underline{\textit{Case \textsc{T-Loc}:}} $e = l$, $[l/z]l = l$, and the desired result is immediate.\\

\noindent\underline{\textit{Case \textsc{T-Sub}:}} $e = e_1$, and by inversion on T-Sub, we get $\Gamma, z:\tau' \mid \Sigma \vdash e_1:\{\varepsilon_1\} \tau_1$, $\Gamma, z:\tau' \mid \Sigma \vdash \tau_1 <: \tau_2$ and $\Gamma, z:\tau' \mid \Sigma \vdash \varepsilon_1 <: \varepsilon_2$.
    By induction hypothesis, we have
    \mbox{$\Gamma \mid \Sigma \vdash [l/z]e_1:\{[l/z]\varepsilon_1\} [l/z]\tau_1$}, \mbox{$\Gamma \mid \Sigma \vdash [l/z]\tau_1 <: [l/z]\tau_2$}, and \mbox{$\Gamma \mid \Sigma \vdash [l/z]\varepsilon_1 <: [l/z]\varepsilon_2$}.
    Then, by T-sub,
    \mbox{$\Gamma \mid \Sigma \vdash [l/z]e_1 : \{[l/z]\varepsilon_2\}[l/z]\tau_2$}

\noindent\underline{\textit{Case \textsc{DT-Def}:}}  By inversion, we have 
  $ \Gamma, z:\tau ,~x : \tau_1 \mid \Sigma \vdash e : \{ \varepsilon' \}~\tau_2$,
  $\Gamma,z:\tau ,~x : \tau_1 \mid \Sigma \vdash \varepsilon~\mathit{wf} $,
  $\Gamma, z:\tau \mid \Sigma \vdash \varepsilon' <: \varepsilon $,
  By IH, we have
\mbox{$ \Gamma ,~x : [l/z]\tau_1 \mid \Sigma \vdash [l/z]e : \{ [l/z]\varepsilon' \}~[l/z]\tau_2$},
  \mbox{$\Gamma, ~x : [l/z]\tau_1 \mid \Sigma \vdash [l/z]\varepsilon~\mathit{wf} $},
  \mbox{$\Gamma \mid \Sigma \vdash [l/z]\varepsilon' <: [l/z]\varepsilon $}.
  By DT-Def, we have
  \mbox{$\Gamma \mid \Sigma \vdash \keyw{def} m(x : [l/z]\tau_1) : \{ [l/z]\varepsilon \}~[l/z]\tau_2 = [l/z]e~:~\keyw{def} m(x : [l/z]\tau_1) : \{ [l/z]\varepsilon \}~[l/z]\tau_2$}

\noindent\underline{\textit{Case \textsc{DT-Var}:}} $d = \keyw{var} f : \tau = n$, and by definition of $n$, there are two subcases:

\underline{\textit{Subcase $n$ is $x$:}} In this case, \mbox{$d = \keyw{var} f : \tau = x$}, and by inversion on \textsc{DT-Var}, we get\linebreak
\mbox{$\Gamma,~z : \tau'~|~\Sigma \vdash x : \{ \}~\tau$}. There are two subcases to consider, depending on whether $x$ is $z$ or another variable. If $x = z$, then by the induction hypothesis, \mbox{$\Gamma~|~\Sigma \vdash [l/z]x : \{ \}~[l/z]\tau$}, which yields \mbox{$\Gamma~|~\Sigma \vdash l : \{ \}~[l/z]\tau$} and \mbox{$\tau = \tau'$}, and thus,
\mbox{$\Gamma~|~\Sigma \vdash \keyw{var} f : [l/z]\tau = l~:~\keyw{var} f : [l/z]\tau$}, i.e.,\linebreak
\mbox{$\Gamma~|~\Sigma \vdash [l/z](\keyw{var} f : \tau = l)~:~[l/z](\keyw{var} f : \tau)$}, as required. If \mbox{$x \not = z$}, then
\mbox{$\Gamma~|~\Sigma \vdash [l/z]x : \{ \}~[l/z]\tau$} yields \mbox{$\Gamma~|~\Sigma \vdash x : \{ \}~[l/z]\tau$}, and thus,
\mbox{$\Gamma~|~\Sigma \vdash \keyw{var} f : [l/z]\tau = x~:~\keyw{var} f : [l/z]\tau$}, i.e.,\linebreak
\mbox{$\Gamma~|~\Sigma \vdash [l/z](\keyw{var} f : \tau = x)~:~[l/z](\keyw{var} f : \tau)$}, as required.

\underline{\textit{Subcase $n$ is $l$:}} In this case, \mbox{$d = \keyw{var} f : \tau = l$}, i.e., the field is resolved to a location $l$. This is not affected by the substitution, and the desired result is immediate.\\



\noindent\underline{\textit{Case \textsc{DT-Effect}:}}
By IH, we have
 $\Gamma \mid \Sigma \vdash [l/z]\varepsilon\ wf$.
 We use DT-Effect to derive
 \mbox{$\Gamma \mid \Sigma \vdash \keyw{effect} \ g = \{[l/z]\varepsilon\} :\keyw{effect}\ g = \{[l/z]\varepsilon\}$}


\noindent\underline{\textit{Case \textsc{Subeffect-Subset}:}}
   By inversion, we have $\varepsilon_1 \subseteq \varepsilon_2$. So $[l/z] \varepsilon_1 \subseteq [l/z] \varepsilon_2$. By Subeffect-Subset, we have 
    $\Gamma \mid \Sigma \vdash [l/z]\varepsilon_1 <: [l/z]\varepsilon_2$.
    
    
\noindent\underline{\textit{Case \textsc{Subeffect-Upperbound}:}}
    By inversion, we have 
    \mbox{$\varepsilon_1 = \varepsilon_1' \cup \{x.g\}$},
    \mbox{$\Gamma, z:\tau \mid \Sigma \vdash x: \{y \Rightarrow \sigma \}$},
    \mbox{$effect\ g \leqslant \{\varepsilon\} \in \sigma $}
    and 
    \mbox{$\Gamma, z:\tau \mid \Sigma \vdash \varepsilon_1' \cup [x/y]\varepsilon <: \varepsilon_2$}.
    By IH, we have 
    \mbox{$\Gamma \mid\Sigma \vdash [l/z]\varepsilon_1' \cup [l/z][x/y]\varepsilon <: [l/z]\varepsilon_2$}.
    Since $y$ is a free variable, we select $y$ such that $x \neq y$ and $y \neq z$. We case on if $z = x$:
    \begin{enumerate}
    \item
    If $z \neq x$, then we can swap the order of the substitutions on $\varepsilon$
    \mbox{$\Gamma \mid\Sigma \vdash [l/z]\varepsilon_1' \cup [x/y][l/z]\varepsilon <: [l/z]\varepsilon_2$}.
    Using substitution lemma for typing on \mbox{$\Gamma, z:\tau \mid \Sigma \vdash x: \{y \Rightarrow \sigma \}$}, we have 
    \mbox{$\Gamma \mid \Sigma \vdash x : \{y \Rightarrow [l/z]\sigma \}$},
    \mbox{$\keyw{effect}\ g \leqslant [l/z]\varepsilon \in [l/z]\sigma$}.
    Using Subeffect-Upperbound, we have 
    \mbox{$\Gamma \mid \Sigma \vdash [l/z]\varepsilon_1' \cup \{x.g\} <: [l/z]\varepsilon_2$},
    Which is equivalent to 
    \mbox{$\Gamma \mid \Sigma \vdash [l/z] \varepsilon_1 <: [l/z] \varepsilon_2$}.
     \item If $z = x$
     Then we have
     \mbox{$\Gamma \mid \Sigma \vdash [l/z]\varepsilon_1'\cup [l/x,y]\varepsilon <:[l/z]\varepsilon_2$},
     Which is equivalent to
     \mbox{$\Gamma \mid \Sigma \vdash [l/z]\varepsilon_1'\cup [l/y][l/z]\varepsilon <:[l/z]\varepsilon_2$}
     We case on the derivation of \mbox{$\Gamma, z:\tau \mid \Sigma \vdash z : \{y \Rightarrow \sigma\}$}. 
     \begin{enumerate}
         \item (T-Var)\\[3ex]
         \infer
    {\Gamma, z:\tau \mid \Sigma \vdash z : \tau}
    {z :\tau \in \Gamma, z : \tau}\\[3ex]
    So $\tau = \{y \Rightarrow \sigma\}$. By our assumption, we have 
    $\Gamma \mid \Sigma \vdash l : \{y \Rightarrow [l/z]\sigma\}$.
    Since \mbox{$\keyw{effect}\ g\leqslant \varepsilon \in \sigma$}, we have
    $\keyw{effect}\ g \leqslant [l/z]\varepsilon \in [l/z]\sigma$.
    Therefore, we can use Subeffect-Upperbound on $\{l.g\}$ to derive 
    $\Gamma \mid \Sigma \vdash [l/z]\varepsilon_1' \cup \{l.g\} <: [l/z]\varepsilon_2$,
    Which is equivalent to 
    $\Gamma \mid \Sigma \vdash [l/z]\varepsilon_1 <: [l/z]\varepsilon_2$
    \item (T-Sub)\\[3ex]
    \infer
    {\Gamma, z : \tau \mid \Sigma \vdash z : \{y \Rightarrow \sigma\}}
    {\Gamma, z : \tau\mid \Sigma \vdash z : \tau_1 \quad \Gamma, z:\tau\mid\Sigma \vdash \tau_1 <: \{y \Rightarrow \sigma\}}\\[3ex]
    Notice that we introduced a new type $\tau_1$ that $z$ can be ascribed to. The judgment \mbox{$\Gamma, z : \tau\mid \Sigma \vdash z : \tau_1$} can be derived by T-Sub, which introduce a new type $\tau_2$ such that $\Gamma, z:\tau \mid \Sigma \vdash \tau_2 <: \tau_1$, or T-Var, which shows $\tau_1 = \tau$. Therefore if we follow the derivation tree, we get a chain relation
    $\Gamma, z:\tau\mid\Sigma \vdash \tau_1 <: \{y \Rightarrow \sigma\}$,
    $\Gamma, z:\tau\mid\Sigma \vdash \tau_2 <: \tau_1$,
    $\dots$,
    $\Gamma, z:\tau\mid\Sigma \vdash \tau <: \tau_n$.
    We can apply IH on these judgments, so we have a chain
    $\Gamma\mid\Sigma \vdash [l/z]\tau_1 <: \{y \Rightarrow [l/z]\sigma\}$,
    $\Gamma\mid\Sigma \vdash [l/z]\tau_2 <: [l/z]\tau_1$
    $\dots$,
    $\Gamma\mid\Sigma \vdash [l/z]\tau <: [l/z]\tau_n$.
    By transitivity of subtyping, we have
    $\Gamma \mid \Sigma \vdash [l/z]\tau <: \{y \Rightarrow [l/z]\sigma\}$
    So we have 
    $\Gamma \mid \Sigma \vdash l : \{y \Rightarrow [l/z]\sigma\}$
    The rest of the proof is similar to case (a).
     \end{enumerate}
     \end{enumerate}

\noindent\underline{\textit{Case \textsc{Subeffect-Def-1}:}}
    By inversion, we have 
    $\varepsilon_2 = \varepsilon_2' \cup \{x.g\}$,
    $\Gamma, z:\tau \mid \Sigma \vdash x: \{y \Rightarrow \sigma \}$,
    \mbox{$\keyw{effect}\ g = \{\varepsilon\} \in \sigma $},
    and 
    $\Gamma, z:\tau \mid \Sigma \vdash \varepsilon_1 <: \varepsilon_2' \cup [x/y]\varepsilon$.
    By IH, we have 
    \mbox{$\Gamma \mid \Sigma \vdash [l/z]\varepsilon_1 <: [l/z]\varepsilon_2' \cup [l/z][x/y]\varepsilon$}.
    Since y is a free variable, we can select y such that $y \neq x$ and $y\neq z$. We case on if $x = z$:
    \begin{enumerate}
    \item If $z \neq x$, then
    $\Gamma \mid \Sigma \vdash [l/z]\varepsilon_1 <: [l/z]\varepsilon_2' \cup [x/y][l/z]\varepsilon$
    By substitution lemma for typing, we have 
    $\Gamma \mid \Sigma \vdash x : \{y \Rightarrow [l/z]\sigma \}$,
    $\keyw{effect}\ g = [l/z]\varepsilon \in [l/z]\sigma$.
    Using Subeffect-Def-1, we have
    $\Gamma \mid \Sigma \vdash [l/z]\varepsilon_1 <: [l/z]\varepsilon_2' \cup \{x.g\}$,
    which is equivalent to 
    $\Gamma \mid \Sigma \vdash [l/z]\varepsilon_1 <: [l/z]\varepsilon_2$
    \item If $z = x$
     Then we have
     $\Gamma \mid \Sigma \vdash [l/z]\varepsilon_1 <:[l/z]\varepsilon_2' \cup [l/x, y]\varepsilon$,
     which is equivalent to
     $\Gamma \mid \Sigma \vdash [l/z]\varepsilon_1  <:[l/z]\varepsilon_2' \cup [l/y][l/z]\varepsilon$
     
      We case on the derivation of $\Gamma, z:\tau \mid \Sigma \vdash z : \{y \Rightarrow \sigma\}$. \begin{enumerate}
         \item (T-Var)\\[3ex]
         \infer
    {\Gamma, z:\tau \mid \Sigma \vdash z : \tau}
    {z :\tau \in \Gamma, z : \tau}\\[3ex]
    So $\tau = \{y \Rightarrow \sigma\}$. By our assumption, we have 
    $\Gamma \mid \Sigma \vdash l : \{y \Rightarrow [l/z]\sigma\}$.
    Since \mbox{$\keyw{effect}\ g = \{\varepsilon\} \in \sigma$}, we have
    \mbox{$\keyw{effect}\ g = \{[l/z]\varepsilon\} \in [l/z]\sigma$}.
    Therefore, we can use Subeffect-Def-1 on $\{l.g\}$ to derive 
    $\Gamma \mid \Sigma \vdash [l/z]\varepsilon_1 <: [l/z]\varepsilon_2' \cup \{l.g\}$,
    Which is equivalent to 
    $\Gamma \mid \Sigma \vdash [l/z]\varepsilon_1 <: [l/z]\varepsilon_2$
    \item (T-Sub)\\[3ex]
    \infer
    {\Gamma, z : \tau \mid \Sigma \vdash z : \{y \Rightarrow \sigma\}}
    {\Gamma, z : \tau\mid \Sigma \vdash z : \tau_1 \quad \Gamma, z:\tau\mid\Sigma \vdash \tau_1 <: \{y \Rightarrow \sigma\}}\\[3ex]
    Notice that we introduced a new type $\tau_1$ that $z$ can be ascribed to. The judgment $\Gamma, z : \tau\mid \Sigma \vdash z : \tau_1$ can be derived by T-Sub, which introduce a new type $\tau_2$ such that $\Gamma, z:\tau \mid \Sigma \vdash \tau_2 <: \tau_1$, or T-Var, which shows $\tau_1 = \tau$. Therefore if we follow the derivation tree, we get a chain relation
    $\Gamma, z:\tau\mid\Sigma \vdash \tau_1 <: \{y \Rightarrow \sigma\}$,
    $\Gamma, z:\tau\mid\Sigma \vdash \tau_2 <: \tau_1$,
    $\dots$,
    $\Gamma, z:\tau\mid\Sigma \vdash \tau <: \tau_n$.
    We can apply IH on these judgments, so we have a chain
    $\Gamma\mid\Sigma \vdash [l/z]\tau_1 <: \{y \Rightarrow [l/z]\sigma\}$,
    $\Gamma\mid\Sigma \vdash [l/z]\tau_2 <: [l/z]\tau_1$,
    $\dots$,
    $\Gamma\mid\Sigma \vdash [l/z]\tau <: [l/z]\tau_n$.
    By transitivity of subtyping, we have
    $\Gamma \mid \Sigma \vdash [l/z]\tau <: \{y \Rightarrow [l/z]\sigma\}$.
    So we have 
    $\Gamma \mid \Sigma \vdash l : \{y \Rightarrow [l/z]\sigma\}$.
    The rest of the proof is similar to case (a).
     \end{enumerate}
    \end{enumerate}

\noindent\underline{\textit{Case \textsc{Subeffect-Def-2}:}}
This case is identical to \underline{\textit{Case \textsc{Subeffect-Upperbound}}}


\noindent\underline{\textit{Case \textsc{Subeffect-Lowerbound}:}}
This case is identical to \underline{\textit{Case \textsc{Subeffect-Def-1}}}


\noindent\underline{\textit{Case \textsc{WF-Effect}:}}Let $n_i.g_j \in \varepsilon$ be arbitrary. By inversion, we have
$\Gamma, z : \tau \mid \Sigma \vdash n_i : \{\} \{y_i \Rightarrow \overline{\sigma_i}\}$.
and the effect declaration of $g_j$ is in $\overline{\sigma_i}$. 
By IH, we have 
$\Gamma \mid \Sigma \vdash [l/z]n_i : \{\} \{y_i \Rightarrow [l/z]\overline{\sigma_i}\}$
and the effect declaration of $g_j$ is in $\overline{\sigma_i}$. So we have $[l/z]\varepsilon\ wf$ by WF-Effect. \\[3ex]


\noindent Thus, substituting terms in a well-typed expression preserves the typing.
\end{proof}


\section{Proof of Theorem \ref{theorem-preservation} (Preservation)}
\label{app-preservation}


If \mbox{$\Gamma~|~\Sigma \vdash e : \{ \varepsilon \}~\tau$}, \mbox{$\mu : \Sigma$}, and \mbox{$\langle e~|~\mu \rangle \longrightarrow \langle e'~|~\mu' \rangle$}, then \mbox{$\exists \Sigma' \supseteq \Sigma$}, \mbox{$\mu' : \Sigma'$}, $\exists \varepsilon'$, such that $\Gamma \vdash \varepsilon' <: \varepsilon$, and \mbox{$\Gamma~|~\Sigma' \vdash e' : \{ \varepsilon' \}~\tau$}.


\begin{proof} The proof is by induction on a derivation of \mbox{$\Gamma~|~\Sigma \vdash e : \{ \varepsilon \}~\tau$}. At each step of the induction, we assume that the desired property holds for all subderivations and proceed by case analysis on the final rule in the derivation. Since we assumed \mbox{$\langle e~|~\mu \rangle \longrightarrow \langle e'~|~\mu' \rangle$} and there are no evaluation rules corresponding to variables or locations, the cases when $e$ is a variable \mbox{(\textsc{T-Var})} or a location (\textsc{T-Loc}) cannot arise. For the other cases, we argue as follows:
\\

\noindent\underline{\textit{Case \textsc{T-New}:}}
\mbox{$e = \keywadj{new}(x \Rightarrow \overline{d})$}, and by inversion on \textsc{T-New}, we get\linebreak
\mbox{$\forall i,~d_i \in \overline{d},~\sigma_i \in \overline{\sigma},~\Gamma,~x : \{ x \Rightarrow \overline{\sigma} \}~|~\Sigma \vdash d_i : \sigma_i$}. The store changes from $\mu$ to\linebreak
\mbox{$\mu' = \mu,~l \mapsto \{ x \Rightarrow \overline{d} \}$}, i.e., the new store is the old store augmented with a new mapping for the location $l$, which was not in the old store ($l \not \in \mathit{dom}(\mu)$). From the premise of the theorem, we know that $\mu : \Sigma$, and by the induction hypothesis, all expressions of $\Gamma$ are properly allocated in $\Sigma$. Then, by \textsc{T-Store}, we have $\mu,~l \mapsto \{ x \Rightarrow \overline{d} \}~:~\Sigma,~l : \{ x \Rightarrow \overline{\sigma} \}$, which implies that $\Sigma' = \Sigma,~l : \{ x \Rightarrow \overline{\sigma} \}$. Finally, by \textsc{T-Loc}, $\Gamma~|~\Sigma \vdash l : \{\}~\{ x \Rightarrow \overline{\sigma} \}$, and $\varepsilon' = \varnothing = \varepsilon$. Thus, the right-hand side is well typed.
\\

\noindent\underline{\textit{Case \textsc{T-Method}:}}
$e = e_1.m(e_2)$, and by the definition of the evaluation relation, there are two subcases:

\underline{\textit{Subcase \textsc{E-Congruence}:}} In this case, either $\langle e_1~|~\mu \rangle \longrightarrow \langle e_1'~|~\mu' \rangle$ or $e_1$ is a value and \mbox{$\langle e_2~|~\mu \rangle \longrightarrow \langle e_2'~|~\mu' \rangle$}. Then, the result follows from the induction hypothesis and \mbox{\textsc{T-Method}}.

\underline{\textit{Subcase \textsc{E-Method}:}} In this case, both $e_1$ and $e_2$ are values, namely, locations $l_1$ and $l_2$ respectively. Then, by inversion on \textsc{T-Method}, we get that \mbox{$\Gamma~|~\Sigma \vdash e_1 : \{ \varepsilon_1 \}~\{ x \Rightarrow \overline{\sigma} \}$},\linebreak
\mbox{$\keyw{def}~ m(y : \tau_2) : \{ \varepsilon_3 \}~\tau_1 \in \overline{\sigma}$}, $\Gamma~|~\Sigma \vdash [e_1/x][e_2/y]\varepsilon_3~\mathit{wf}$, $\Gamma~|~\Sigma \vdash e_2 : \{ \varepsilon_2 \}~[e_1/x]\tau_2$, and\linebreak
$\varepsilon = \varepsilon_1 \cup \varepsilon_2 \cup [e_1/x][e_2/y]\varepsilon_3$. The store $\mu$ does not change, and since \mbox{\textsc{T-Store}} has been applied throughout, the store is well typed, and thus,\linebreak
\mbox{$\Gamma~|~\Sigma \vdash \keyw{def} m(x : \tau_1) : \{ \varepsilon \}~\tau_2 = e~:~\keyw{def} m(x : \tau_1) : \{ \varepsilon \}~\tau_2$}. Then, by inversion on \mbox{\textsc{DT-Def}}, we know that \mbox{$\Gamma,~x : \tau_1~|~\Sigma \vdash e : \{ \varepsilon' \}~\tau_2$} and\linebreak
$\Gamma, x : \tau_1 \mid \Sigma \vdash \varepsilon' <: \varepsilon$. Finally, by the subsumption lemma, substituting locations for variables in $e$ preserves its type, and therefore, the right-hand side is well typed.
\\

\noindent\underline{\textit{Case \textsc{T-Field}:}}
$e = e_1.f$, and by the definition of the evaluation relation, there are two subcases:

\underline{\textit{Subcase \textsc{E-Congruence}:}} In this case, $\langle e_1~|~\mu \rangle \longrightarrow \langle e_1'~|~\mu' \rangle$, and the result follows from the induction hypothesis and \textsc{T-Field}.

\underline{\textit{Subcase \textsc{E-Field}:}} In this case, $e_1$ is a value, i.e., a location $l$. Then, by inversion on \mbox{\textsc{T-Field}}, we have $\Gamma~|~\Sigma \vdash l : \{ \varepsilon \}~\{ x \Rightarrow \overline{\sigma} \}$, where $\varepsilon = \varnothing$, and $\keyw{var}~ f : \tau \in \overline{\sigma}$. The store $\mu$ does not change, and since \textsc{T-Store} has been applied throughout, the store is well typed, and thus, \mbox{$\Gamma~|~\Sigma \vdash \keyw{var}~ f : \tau = l_1~:~\keyw{var}~ f : \tau$}. Then, by inversion on \textsc{DT-Varl}, we know that\linebreak
\mbox{$\Gamma~|~\Sigma \vdash l_1 : \{ \}~\tau$} and $\varepsilon' = \varnothing = \varepsilon$, and the right-hand side is well typed.
\\

\noindent\underline{\textit{Case \textsc{T-Assign}:}}
$e = (e_1.f = e_2)$, and by the definition of the evaluation relation, there are two subcases:

\underline{\textit{Subcase \textsc{E-Congruence}:}} In this case, either $\langle e_1~|~\mu \rangle \longrightarrow \langle e_1'~|~\mu' \rangle$ or $e_1$ is a value and \mbox{$\langle e_2~|~\mu \rangle \longrightarrow \langle e_2'~|~\mu' \rangle$}. Then, the result follows from the induction hypothesis and \textsc{T-Assign}.

\underline{\textit{Subcase \textsc{E-Assign}:}} In this case, both $e_1$ and $e_2$ are values, namely locations $l_1$ and $l_2$ respectively. Then, by inversion on \textsc{T-Assign}, we get that \mbox{$\Gamma~|~\Sigma \vdash l_1 : \{ \varepsilon_1 \}~\{ x \Rightarrow \overline{\sigma} \}$}, \mbox{$\keyw{var}~ f : \tau \in \overline{\sigma}$}, \mbox{$\Gamma~|~\Sigma \vdash l_2 : \{ \varepsilon_2 \}~\tau$}, and \mbox{$\varepsilon = \varepsilon_1 = \varepsilon_2 = \varnothing$}. The store changes as follows:\linebreak
\mbox{$\mu' = [l_1 \mapsto \{ x \Rightarrow \overline{d}' \}/l_1 \mapsto \{ x \Rightarrow \overline{d} \}]\mu$}, where $\overline{d}' = [\keyw{var} f:\tau = l_2/\keyw{var} f : \tau = l]\overline{d}$. However, since \textsc{T-Store} has been applied throughout and the substituted location has the type expected by \textsc{T-Store}, the new store is well typed (as well as the old store), and thus,\linebreak
\mbox{$\Gamma~|~\Sigma \vdash \keyw{var}~ f : \tau = l_2~:~\keyw{var}~ f : \tau$}. Then, by inversion on \textsc{DT-Varl}, we know that\linebreak
\mbox{$\Gamma~|~\Sigma \vdash l_2 : \{ \}~\tau$} and $\varepsilon' = \varnothing$, and the right-hand side is well typed.
\\

\noindent\underline{\textit{Case \textsc{T-Sub}:}}
The result follows directly from the induction hypothesis.
\\

\noindent Thus, the program written in this language is always well typed.
\end{proof}



\section{Proof of Theorem \ref{theorem-progress} (Progress)}
\label{app-progress}


If $\varnothing~|~\Sigma \vdash e : \{ \varepsilon \}~\tau$ (i.e., $e$ is a closed, well-typed expression), then either
\begin{enumerate}
\item $e$ is a value (i.e., a location) or
\item $\forall \mu$ such that $\mu : \Sigma$,
   $\exists e', \mu'$ such that $\langle e~|~\mu \rangle \longrightarrow \langle e'~|~\mu' \rangle$.
\end{enumerate}

\begin{proof} The proof is by induction on the derivation of $\Gamma~|~\Sigma \vdash e : \{ \varepsilon \}~\tau$, with a case analysis on the last typing rule used. The case when $e$ is a variable (\textsc{T-Var}) cannot occur, and the case when $e$ is a location (\textsc{T-Loc}) is immediate, since in that case $e$ is a value. For the other cases, we argue as follows:
\\

\noindent\underline{\textit{Case \textsc{T-New}:}}
$e = \keywadj{new}(x \Rightarrow \overline{d})$, and by \textsc{E-New}, $e$ can make a step of evaluation if the $\keywadj{new}$ expression is closed and there is a location available that is not in the current store $\mu$. From the premise of the theorem, we know that the expression is closed, and there are infinitely many available new locations, and therefore, $e$ indeed can take a step and become a value (i.e., a location $l$). Then, the new store $\mu'$ is $\mu, l \mapsto \{ x \Rightarrow \overline{d} \}$, and all the declarations in $\overline{d}$ are mapped in the new store.
\\

\noindent\underline{\textit{Case \textsc{T-Method}:}}
\mbox{$e = e_1.m(e_2)$}, and by the induction hypothesis applied to\linebreak
\mbox{$\Gamma~|~\Sigma \vdash e_1 : \{ \varepsilon_1 \}~\{ x \Rightarrow \overline{\sigma} \}$}, either $e_1$ is a value or else it can make a step of evaluation, and, similarly, by the induction hypothesis applied to $\Gamma~|~\Sigma \vdash e_2 : \{ \varepsilon_2 \}~[e_1/x]\tau_2$, either $e_2$ is a value or else it can make a step of evaluation. Then, there are two subcases:

\underline{\textit{Subcase $\langle e_1~|~\mu \rangle \longrightarrow \langle e_1'~|~\mu' \rangle$ or $e_1$ is a value and $\langle e_2~|~\mu \rangle \longrightarrow \langle e_2'~|~\mu' \rangle$:}} If $e_1$ can take a step or if $e_1$ is a value and $e_2$ can take a step, then rule \textsc{E-Congruence} applies to $e$, and $e$ can take a step.

\underline{\textit{Subcase $e_1$ and $e_2$ are values:}} If both $e_1$ and $e_2$ are values, i.e., they are locations $l_1$ and $l_2$ respectively, then by inversion on \textsc{T-Method}, we have \mbox{$\Gamma~|~\Sigma \vdash l_1 : \{ \varepsilon_1 \}~\{ x \Rightarrow \overline{\sigma} \}$} and\linebreak
\mbox{$\keyw{def}~ m(y : \tau_2) : \{ \varepsilon_3 \}~\tau_1 \in \overline{\sigma}$}. By inversion on \textsc{T-Loc}, we know that the store contains an appropriate mapping for the location $l_1$, and since \textsc{T-Store} has been applied throughout, the store is well typed and $l_1 \mapsto \{ x \Rightarrow \overline{d} \} \in \mu$ with $\keyw{def} m(y : \tau_1) : \{ \varepsilon_3 \}~\tau_2 = e \in \overline{d}$. Therefore, the rule \textsc{E-Method} applies to $e$, $e$ can take a step, and $\mu' = \mu$.
\\

\noindent\underline{\textit{Case \textsc{T-Field}:}}
$e = e_1.f$, and by the induction hypothesis, either $e_1$ can make a step of evaluation or it is a value. Then, there are two subcases:

\underline{\textit{Subcase $\langle e_1~|~\mu \rangle \longrightarrow \langle e_1'~|~\mu' \rangle$:}} If $e_1$ can take a step, then rule \textsc{E-Congruence} applies to $e$, and $e$ can take a step.

\underline{\textit{Subcase $e_1$ is a value:}} If $e_1$ is a value, i.e., a location $l$, then by inversion on \textsc{T-Field}, we have\linebreak
\mbox{$\Gamma~|~\Sigma \vdash l : \{ \varepsilon \}~\{ x \Rightarrow \overline{\sigma} \}$} and $\keyw{var}~ f : \tau \in \overline{\sigma}$. By inversion on \textsc{T-Loc}, we know that the store contains an appropriate mapping for the location $l$, and since \textsc{T-Store} has been applied throughout, the store is well typed and $l \mapsto \{ x \Rightarrow \overline{d} \} \in \mu$ with $\keyw{var} f : \tau = l_1 \in \overline{d}$. Therefore, the rule \textsc{E-Field} applies to $e$, $e$ can take a step, and $\mu' = \mu$.
\\

\noindent\underline{\textit{Case \textsc{T-Assign}:}}
$e = (e_1.f = e_2)$, and by the induction hypothesis, either $e_1$ is a value or else it can make a step of evaluation, and likewise $e_2$. Then, there are two subcases:

\underline{\textit{Subcase $\langle e_1~|~\mu \rangle \longrightarrow \langle e_1'~|~\mu' \rangle$ or $e_1$ is a value and $\langle e_2~|~\mu \rangle \longrightarrow \langle e_2'~|~\mu' \rangle$:}} If $e_1$ can take a step or if $e_1$ is a value and $e_2$ can take a step, then rule \textsc{E-Congruence} applies to $e$, and $e$ can take a step.

\underline{\textit{Subcase $e_1$ and $e_2$ are values:}} If both $e_1$ and $e_2$ are values, i.e., they are locations $l_1$ and $l_2$ respectively, then by inversion on \textsc{T-Assign}, we have $\Gamma~|~\Sigma \vdash l_1 : \{ \varepsilon_1 \}~\{ x \Rightarrow \overline{\sigma} \}$, $\keyw{var}~ f : \tau \in \overline{\sigma}$, and $\Gamma~|~\Sigma \vdash l_2 : \{ \varepsilon_2 \}~\tau$. By inversion on \textsc{T-Loc}, we know that the store contains an appropriate mapping for the locations $l_1$ and $l_2$, and since \textsc{T-Store} has been applied throughout, the store is well typed and $l_1 \mapsto \{ x \Rightarrow \overline{d} \} \in \mu$ with $\keyw{var} f : \tau = l \in \overline{d}$. A new well-typed store can be created as follows: $\mu' = [l_1 \mapsto \{ x \Rightarrow \overline{d}' \}/l_1 \mapsto \{ x \Rightarrow \overline{d} \}]\mu$, where $\overline{d}' = [\keyw{var} f : \tau = l_2/\keyw{var} f : \tau = l]\overline{d}$. Then, the rule \textsc{E-Assign} applies to $e$, and $e$ can take a step.
\\

\noindent\underline{\textit{Case \textsc{T-Sub}:}}
The result follows directly from the induction hypothesis.
\\

\noindent Thus, the program written in this language never gets stuck.
\end{proof}





\backmatter

%\renewcommand{\baselinestretch}{1.0}\normalsize

% By default \bibsection is \chapter*, but we really want this to show
% up in the table of contents and pdf bookmarks.
\renewcommand{\bibsection}{\chapter{\bibname}}
%\newcommand{\bibpreamble}{This text goes between the ``Bibliography''
%  header and the actual list of references}
\bibliographystyle{plainnat}
\bibliography{related} %your bib file

\end{document}
