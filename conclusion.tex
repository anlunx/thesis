% !TEX root = thesis.tex

\chapter{Related Work and Conclusion}

\section{Related Work}

\subsection{Restrictive Effect Systems}
A restrictive effect system considers effects that are built into the language, such as reading and writing states or exceptions, and provide a way to restrict the usage of such effects. Our effect system introduced in chapter \ref{chapter-bound} is a restrictive effect system. Restrictive effect systems were first proposed by \citet{lucassen87} to track reads and writes to memory. Then \citet{lucassen88} extended this effect system to support polymorphism.  Restrictive Effects have since been used for a wide variety of purposes, including exceptions in Java~\cite{kiniry06} and asynchronous event handling~\cite{bracevac18}. 

\subsection{ Bounded Effect Polymorphism.}  A limited form of bounded effect polymorphism were explored by~\citet{10.5555/645393.651891}, who bound effect parameters by the resources they may act on; however, the bound cannot be another aribrary effect, as in our system.  \citet{DBLP:conf/ecoop/LongLR15} use a form of bounded effect polymorphism internally but do not expose it to users of their system.

\subsection{Subeffecting.} 
Some effect systems, such as Koka~\cite{leijen14}, provide a built-in set of effects with fixed sub-effecting relationships between them.
\citet{rytz12} supports more flexibility via an extensible framework for effects.  Users can plug in their own domain of effects, specifying an effect lattice representing sueffecting relationships.  Each plugin is monolithic.  In contrast, our effect members allow new effects to be incrementally added and related to existing effects using declared subeffect bounds.

\subsection{Path-dependent Effects}
The Effekt library by \citet{brachthauser20} explores algebraic effects as a library of the Scala programming language. Since their effect system is built on the path-dependent type system of Scala, it bears some similarities to our system. However, their work is largely orthogonal to our contributions due to following reasons: 

	 Our goal is to check the effects of general purpose code. In contrast, Brachthäuser's approach requires all effect-checked code to be written in a monad. This is required to support control effects (i.e. prescriptive effects), but it is not an incidental difference: it is also an integral part of their static effect checking system, because monads are the way that they couple Scala's type members (which provide abstraction and polymorphism) to effects. Their paper therefore, does not solve the problem of soundly checking effects for non-monadic code. Most code in Scala and Wyvern--let alone more conventional languages such as Java--is non-monadic, for good reasons: monads are restrictive and, for some kinds of programming, quite awkward. Programmers may be willing to use monads narrowly to get the benefits of control effects that Brachthäuser et al. support, but outside the Haskell community, it does not seem likely they would be willing to use them at a much broader scale for the purpose of descriptive effects.
	 
   Our approach provides abstraction and polymorphism for descriptive effects. As discussed above, Brachthäuser et al.'s leverage of Scala's type members provides abstraction and polymorphism only for prescriptive effects, and only in the context of monadic code. Even setting aside the issue of monads, above, it is unclear how their approach can provide abstraction for descriptive effects. The reason is that their abstraction works backwards from the kind we need. For example, we want to be able to implement a logger in terms of file I/O--and hide the fact that it is implemented that way. The natural way to start would be to model file I/O in their system as a set of effect operations that ``handle" I/O operations at the top level (their system does not provide support for this, so it would have to be added). A logger library could then provide a set of "log" effects and a handler for them, implemented in terms of the top-level I/O operations. But it would not be possible to hide the fact that the logger library was implemented in terms of the I/O operations, because the handler for the log effects would have to be annotated with I/O operation effects. Furthermore, all log operations would have to be nested in the scope of the log handler, annoyingly inverting control flow relative to the expected approach. And this would have to be done for every library that abstract from the built-in I/O effects, a highly anti-modular approach.
	



\subsection{Abstract Algebraic Effects}
\label{related-abstract}
Our discuss in chapter \ref{chapter-algebraic} tackles the issue of abstraction of algebraic effects. This issue was originally raised by \citet{leijen18}, but was not discussed in depth. \citet{biernacki19} introduced a core calculus called $\lambda^{HEL}$ with abstract algebraic effects. However, there are multiple distinctions that that distinguishes between our core calculus and $\lambda^{HEL}$. First, we adopted the agent-based syntax that syntactically distinguishes each module by assigning them to different agents. This design allows us to reason about the code using the information provided by agents, and provide syntactic proof for abstraction properties. Moreover, our calculus can be simply extended with existential types, which serves as an abstraction to represent module systems for a high-level language. The benefit of this design is that the programmer would not need to write embeddings explicitly, as embeddings are generated as a semantic object when a value of existential type is evaluated. In comparison, it is unclear from the paper \cite{biernacki19} how a top-level language with a module system could be translated to the coercion-based calculus $\lambda^{HEL}$.

\citet{Zhang19} describe a design of algebraic effects that preserves abstraction in the setting of parametric functions. If a function does not statically know about an algebraic effect, that effect tunnels through that function.  This is different from our form of abstraction, in which the definition of an effect is hidden from clients.


\section{Conclusion}

Effect systems have been actively studied for nearly four decades, but they are not widely used in the software development process because little attention is paid to improve the usability of effect systems when developing large and complex software. On the other hand, type abstraction is an invaluable tool to software designers, which enables programmers to reason about interfaces between different component of programs. Therefore, we explored different ways to achieve effect type abstraction within existing frameworks of computational effects.

This thesis presented the Bounded Abstract Effects, an effect system that allows effects to be defined as members of objects. The types for object in our language serves as the interface that can hide the implementation of an effect. Effects are declared as members of an object type, and effect members can be declared abstractly with upper and lower bounds. Our system enables effect abstraction between different program components, and allows building a hierarchy of abstract effects via effect bounds.

We have also presented a core calculus for abstract algebraic effects, which ensures the correctness of abstraction by using embeddings to keep track of the type information during evaluation. We have provided a syntactic proof of the correctness of the abstraction barrier and have shown that a portion of our Bounded Abstract Effects design can be implemented within the setting of algebraic effects and handlers.  Moreover, we have added the existential types for effects to our system as a foundation for abstract effect types. 






