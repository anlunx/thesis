% !TEX root = thesis.tex

\chapter{Algebraic Effects with Module Systems}
\section{Motivation}
In section \ref{sec-state}, we defined a resource type \m{Var} that provides two public functions, \m{get} and \m{set}, that are annotated with effects, but the concrete implementation that causes effects are not revealed to the client. This construct ensures that any program that accesses to a value of type \m{Var} is annotated with the correct effect. Therefore we can see that, in a restrictive effect system, it is useful to be able to hide the implementation of a computational effect.  

In this section we will show that this form of implementation hiding is also important for building a modular software with algebraic effects and handlers through an example of mutable state. Mutable state as an algebraic effects is often represented by a \m{state} effect with two operations \m{get} and \m{set}. In this example we assume that our mutable state can store or access a int.
\begin{lstlisting}
operation get : Unit -> Int
operation set : Int -> Unit
\end{lstlisting}

The handler of a state effect is usually defined in the following way:
\begin{lstlisting}[mathescape=true]
handler hstate = {
  return x -> $\lambda$_:Int. return x
  get(_; k) -> $\lambda$s:Int. (k s) s
  set(s; k) -> $\lambda$_:Int. (k ()) s
}
\end{lstlisting}
This handler would transform the handled computation into a lambda expression that receives a state as an argument. In the return clause, the argument is ignored. In the clause handling \m{get},  the state argument is passed into the continuation $k$. Since the continuation $k\ s$ is already transformed into a function that expects a state, we pass $s$ to the computation $k\ s$. The clause for the \m{set} operation is similar except that we ignore the state argument, but pass the argument obtained from the \m{set} operation. 

Now we consider the module of a single variable introduced in section \ref{sec-state}, where read and write to the variable cannot bypass the effect checking because the implementation details are hidden by the interface \m{Var}. If we would write a similar module in our core calculus, it is natural to extend our calculus with record type and universal types, and implement the module as following:
\begin{lstlisting}[mathescape=true]
val var = 
  < read $\hookrightarrow$ $\lambda$x:Unit. get((); y.return y),
    write $\hookrightarrow$ $\lambda$x:Int. set(x; y.return ()),
    handle $\hookrightarrow$  $\lambda$c: Unit $\rightarrow$ {state} Int. (with hstate handle c ()) 0 >
\end{lstlisting}

The module provides functions \m{read}, \m{write}, and a handler function \m{handle} that determines the semantics for operations in functions \m{read} and \m{write}. However, this encoding of the module \m{var} does not enforce that the operations \m{get} and \m{set} are always handled by the handle function. In fact, any client that calls function \m{read} and \m{write} can write its own handler to handle the effects. So in order to tackle with this issue, we use the existential type introduced in section \ref{sec-exist} to define the module:
\begin{lstlisting}[mathescape=true]
val var = 
  pack 
  <{get, set}, 
    < read $\hookrightarrow$ $\lambda$x:Unit. get((); y.return y),
      write $\hookrightarrow$ $\lambda$x:Int. set(x; y.return ()),
      handle $\hookrightarrow$  $\lambda$c: Unit $\rightarrow$ {state} Int. (with hstate handle c ()) 0 >
  > as $\exists$state. (read : Unit -> {state} Int) * (write : Int -> {state} Unit) * (handle :  Unit -> {state} Int -> {} Int)
\end{lstlisting}
This encoding of module defines an abstract effect \m{state} on top of the effects \m{get} and \m{set}, and export the module as an existential type that hides the definition of effect \m{state}, therefore enforcing that the client of this module can only handle effect \m{state} by calling the \m{handle} function. 

The embedding design presented in section \ref{sec-core} helps to ensure that the abstraction does not break. Imagine a client of module \m{var} that calls \m{read} and handles it by calling \m{handle}. 

\begin{lstlisting}[mathescape=true]
open var as (state, v) in (v.handle v.read)
\end{lstlisting}

In this example, the \m{state} effect in function \m{read} is correctly handled by the handler \m{handle}, and the result of the computation is 0. Note that the abstraction barrier is still preserved if some handler attempts to handle the abstracted effect. For example, let handler \m{hget} be a handler that handles effect \m{get}. And the client code uses \m{hget} to handle the effect in method \m{read}. 

\begin{lstlisting}[mathescape=true]
handler hget = {
  return x -> x
  get (x; k) -> 1
}
open var as (state, v) in 
  (v.handle ($\lambda$ x: Unit. with hget handle v.read ()))
\end{lstlisting}

Assume that the open expression is an i-expression. When opening the module \m{var}, we assign the opened expression \m{v} to a new agent j that  knows the definition of the \m{state} effect. Because the \m{hget} handler is evaluating in the agent i, it would not be able to handle the effect \m{state} of the function \m{v.read}. Instead, the \m{state} effect will be handled by the handler provided by the module \m{var}, therefore ensuring that the abstraction information is not leaked.



\section{Encoding Abstract Effects Using Algebraic Effects}
Our discussion of abstraction of algebraic effects has been focusing exclusively on purely functional programming, however, as shown in \citet{melicher20}, the expressiveness provided by abstract effects enables programmers to develop secure programs when side-effects are in play. In this section, we show that the effect system in this chapter provides a foundation for expressing abstract effect in chapter \ref{chapter-bound}. 

 \citet{melicher20} and chapter \ref{chapter-bound} presented an effect member mechanism to support effect abstraction: the ability to define higher-level effects in terms of lower-level effects, to hide that definition from clients of an abstraction, and to reveal partial information about an abstract effect through effect bounds. In this chapter we no longer use the object-oriented formalization but use the agent-based lambda calculus introduced in chapter \ref{chapter-algebraic} and extend it with the existential types as a foundation to support effect abstraction. Now we will look at different aspects of the original abstract effect system and discuss how we can incorporate them in the new setting with algebraic effects.
 
 
\subsection{Running Example}
We begin by encoding the running example presented in \citet{melicher20} which demonstrate the key feature of the abstract effects. The original example shows a type and a module implementing the logging facility in the text-editor application, and is shown in figure \ref{fig-logging}. 

\begin{figure}
\label{fig-logging}
\begin{lstlisting}
resource type Logger
  effect ReadLog
  effect UpdateLog
  effect readLog(): {this.ReadLog} String
  effect updateLog(newEntry: String): {this.UpdateLog} Unit

module def logger(f: File): Logger
  effect ReadLog = {f.Read}
  effect UpdateLog = {f.Append}
  def readLog(): {ReadLog} String = f.read()
  def updateLog(newEntry: String): {UpdateLog} Unit = f.append(newEntry)
  
resource type File
  effect Read
  effect Write
  effect Append
  ...
  def read(): {this.Read} String
  def write(s: String): {this.Write} Unit
  def append(s: String): {this.Append} Unit
  ...
\end{lstlisting}

\caption{The logging facility in the text-editor application}
\end{figure}

Consider the code in Fig.~\ref{fig-logging} that shows a type and a module implementing the logging facility of the text-editor application.  In the given implementation of the \li{Logger} type, the \li{logger} module accesses the log file.\footnote{The keyword \li{resource} in the type definition indicates that the implementations of this type may have state and may access system resources; this is orthogonal to effect checking.} All modules of type \li{Logger} must have two methods: the \li{readLog} method that returns the content of the log file and the \li{updateLog} method that appends new entries to the log file. In addition, the \li{Logger} type declares two \textit{abstract} effects, \li{ReadLog} and \li{UpdateLog}, that are produced by the corresponding methods. These effects are abstract because they are not given a definition in the \li{Logger} type, and so it is up to the module implementing the \li{Logger} type to define what they mean. The effect names are user-defined, allowing the choice of meaningful names. 

The \li{logger} module implements the \li{Logger} type. To access the file system, an object of type \li{File} (shown in Fig.~\ref{f-file}) is passed into \li{logger} as a parameter. The \li{logger} module's effect declarations are those of the \li{Logger} type, except now they are \textit{concrete}, i.e., they have specific definitions. The \li{ReadLog} effect of the \li{logger} module is defined to be the \li{Read} effect of the \li{File} object, and accordingly, the \li{readLog} method, which produces the \li{ReadLog} effect, calls \li{f}'s \li{read} method. Similarly, the \li{UpdateLog} effect of the \li{logger} module is defined to be \li{f.Append}, and accordingly, the \li{updateLog} method, which produces the \li{UpdateLog} effect, calls \li{f}'s \li{append} method. In general, effects in a module or object definition must always be concrete, whereas effects in a type definition may be either abstract or concrete.

Using the existential type, the type \li{Logger} can be translated to the following type, note that we only translate one abstract effect \li{ReadLog} and one method \li{readLog} to make the code more readable. We assume that the type String is built into the language.
\begin{lstlisting}[mathescape=true]
type Logger = $\exists$ReadLog. $\langle$ readLog : Unit $\rightarrow$ {ReadLog} String $\rangle$
\end{lstlisting}

Similarly, the \li{File} type can be implemented as follows. Again we leave only one abstract effect and one member function to maintain simplicity of the example.
\begin{lstlisting}[mathescape=true]
type File = $\exists$Read. $\langle$ read : Unit $\rightarrow$ {Read} String $\rangle$
\end{lstlisting}

Then we are finally able to encode the functor \li{logger}, which receives a value of type \li{File} and returns a \li{Logger}. 

\begin{lstlisting}[mathescape=true]
logger = $\lambda$f:File. open f as (fRead, fBody) in 
  pack (fRead, $\langle$readLog $\hookrightarrow$ $\lambda$x: Unit. fBody.read ()$\rangle$) as
    $\exists$ReadLog. $\langle$readLog : Unit $\rightarrow$ {ReadLog} String$\rangle$
\end{lstlisting}

\subsection{Effect Abstraction}

In section  \ref{sec:effect-abstraction},  we defined effect abstraction as the ability to define higher-level effects in terms of lower-level effects and potentially to hide that definition from clients of an abstraction. In the \li{logger} module above, we lifted the low-level file resource into a higher-level logging facility, and defined higher-level effects \li{ReadLog} as an abstraction of the lower-level effect \li{Read}. So application code can reason in terms of effects of higher-level resources when appropriate.



\subsection{Effect Aggregation}
In section \ref{sec:effect-aggregation} we argued that effect aggregation can make the effect system less verbose. The original example declares an effect \li{UpdateLog} as the sum of two effects \li{f.Read} and \li{f.Write}. 
\begin{lstlisting}[xleftmargin=-5pt, numbers=none]
module def logger(f: File): Logger
effect UpdateLog = {f.Read, f.Write}
def updateLog(newEntry: String): {this.UpdateLog} Unit
...
\end{lstlisting}
Although our new calculus is inherently more verbose than the calculus based on path-dependent effects, it can still encode effect aggregation. First we let the \li{File} type contains two abstract effects by existential quantification:
\begin{lstlisting}[mathescape=true]
type File = $\exists$ Read. $\exists$ Write. $\dots$
\end{lstlisting}
Then we can define logger functor. We first open the module \li{f}, then define an existential package where the abstract effect is defined as a sum of the two effects from the module \li{f}:
\begin{lstlisting}[mathescape=true]
logger = $\lambda$ f : File. open f as (fRead, x) in open x as (fWrite, y) in 
  pack ({fRead, fWrite}, $\langle$updateLog$\hookrightarrow \dots \rangle$) as 
    ($\exists$UpdateLog. $\langle$updateLog: String$\rightarrow${UpdateLog} Unit$\rangle$)
\end{lstlisting}





\section{Formalization}
\label{sec-exist}
In the previous sections, we have introduced the core calculus of abstract algebraic effects via embeddings. However it is  impractical for requiring programmers to explicitly annotate each program components with embeddings. So we present a top level language where the annotations are implicitly added during evaluation of the program.  

According to \citet{mitchell88}, there is a correspondence between abstraction data types and existential types. Existential types are often used as a foundation for expressing type abstraction in module systems. The  calculus introduced in this section contains a form of existential type that provide abstraction mechanisms for algebraic effects. The this form of expression of existential type would generate new agents during evaluation of program and automatically separate program components with different knowledge, so programmers would not need to explicitly work with agents and embeddings.



\subsection{Syntax}
\begin{figure}
\label{fig-syntax}
\begin{align*}
&(agents) &i, j &::= \{1 \dots n\}\\
&(lists) & l &::= i \mid il\\
&(expression\ types) &\tau &::= 1 \mid \tau \rightarrow \sigma \mid \exists f.\ \tau \\
&(computation\ types) &\sigma &::= \{\varepsilon\}\tau\\
&(effect\ types) &\varepsilon &::= \cdot \mid f, \varepsilon \mid op, \varepsilon\\
&(i\ values) &{v_i} &::= ()_i \mid \lambda x_i:\tau.\ c_i \mid \m{pack} (\varepsilon, v) \m{as} \exists f.\ \tau\\
&(i\ expression) &e_i &::= x_i \mid v_i \mid [e_j]^\tau_l \mid \m{pack} (\varepsilon, e) \m{as} \exists f.\ \tau  \mid \m{open} e \m{as} (f, x) \m{in} e'\\
&(i\ computation) &c_i &::= \texttt{return}\ e_i \mid op(e_i, y.c_i) \mid \texttt{do}\ x \leftarrow c_i\ \texttt{in}\ c_i' \mid e_i\ e_i' \mid \texttt{with}\ h_i\ \texttt{handle}\ c_i\\
 &\ &\mid\ &[c_j]^\sigma_l \mid [op]^\varepsilon_l (e_i, y_i.c_i)\\
&(i\ handler) &h_i &::= \texttt{handler}\ \{\texttt{return}\ x_i \mapsto c^r_i, op^1(x_i^1, k^1) \mapsto c_i^1 \dots  op^n(x_i^n, k^n) \mapsto c_i^n\}  
\end{align*}
\caption{Syntax for Existential Effects}
\end{figure}

Most of the syntax remains the same for our new language. The existential type $\exists f.\ \tau$ is added as an expression type. The intuition of the type is that the value of this type is a value of type $\{\varepsilon/f\}\tau$ for some effect type $\varepsilon$. 

There  are two new forms of expressions: The introduction form of the existential type, $\m{pack} (\varepsilon, e) \m{as} \exists f.\ tau$ and the elimination form, $\m{open} e \m{as} (f, x) \m{in} e'$. The $\m{pack}$ expression creates existential package that contains an effect type $\varepsilon$ and an expression $e$.  The $\m{open}$ expression opens up an existential package and substitute the expression in the package for the variable $x$.

We only introduce one form of value: the existential package $\m{pack} (\varepsilon, v) \m{as} \exists f.\ \tau$ 


\subsection{Dynamic Semantics}
\begin{figure}[t]
\footnotesize{
\noindent$\fbox{$\langle \{\Delta\}, e \rangle \mapsto \langle \{\Delta\}, e \rangle$}$
\[
\begin{array}{c}

\infer[\textsc{(E-pack)}]
  {\langle \{\Delta\}, \m{pack} (\varepsilon, e) \m{as} \exists f.\ \tau  \rangle \mapsto \langle \{\Delta'\} , \m{pack} (\varepsilon, e') \m{as} \exists f.\ \tau  \rangle}
  {\langle \{\Delta\}, e \rangle \mapsto \langle \{\Delta'\}, e' \rangle} \\[3ex]
  
\infer[\textsc{(E-open1)}]
   {\langle \{\Delta\},  \m{open} (\m{pack} (\varepsilon, e) \m{as} \exists f.\ \tau) \m{as} (f, x) \m{in} e'  \rangle \mapsto \langle \{\Delta'\}, \{[e]_j^\tau / x\}e' \rangle}
  {f \m{fresh} & j \m{fresh} & \Delta'_j = \Delta_i[f \mapsto \varepsilon] & \forall \Delta_k \in \{\Delta\}, \Delta_k' = \Delta_k[f \mapsto f]} \\[3ex]
  
\infer[\textsc{(E-open2)}]
   {\langle \{\Delta\},  \m{open} e \m{as} (f, x) \m{in} e''  \rangle \mapsto \langle \{\Delta'\}, \m{open} e' \m{as} (f, x) \in e''\rangle}
  {\langle \{\Delta\}, e \rangle \mapsto \langle \{\Delta'\}, e' \rangle} \\[3ex]
  
\infer[\textsc{(E-EmbedPack)}]
  {\langle \{\Delta\}, [\m{pack} (\varepsilon, v) \m{as} \exists f.\ \tau]^{\exists f.\ \tau'}_j \rangle \mapsto \langle \{\Delta\}, \m{pack} (\varepsilon, [v]^{\{\varepsilon/f\}\tau'}_j) \m{as} \exists f.\ \tau' \rangle}
  {}
  
\end{array}
\]
\label{static-exist}
\caption{Additional Dynamic Semantics for Existential Type}
}
\end{figure}

The semantics for existential type $\exists f.\ \tau$  hides the definition of the effect label $f$ from the client of the value of this type. We leverage the our previous design of multi-agent calculus to achieve information hiding. However, the previous design assumes that the type information for each agents are predetermined and does not change during evaluation. Since existential types generate new abstraction boundaries, we need a different semantics that allow agents to be created during evaluation. Therefore, we modify the reduction rule to evaluate a pair that contains both the expression to evaluate and a context of type information. The idea to keep track of type information while evaluating terms was used by \citet{grossman00} to encode parametric polymorphism in their system.

We introduce the notation $\{\Delta\}$ to express a list of type maps for all agents in the context $\{\Delta_1, \dots \Delta_n\}$. The type information of each agents can change, and new agents can be generated, the evaluation judgment becomes $\langle \{\Delta\}, e \rangle \mapsto \langle \{\Delta\}, e \rangle$. 

(E-Pack) shows the congruence rule for reduction of a pack expression. (E-Open) opens an existential package: This rule requires $f$ to be a fresh label, which achievable by doing alpha conversion in $e'$. $j$ is a fresh agent. A new type map for agent $j$ extends the type map for $i$ by mapping $f$ to $\varepsilon$. Every existing type map in $\{\Delta\}$ is extended by mapping $f$ to itself. Finally, $[e]^\tau_j$ is substituted for $x$ in $e'$.

(E-EmbedPack) shows how the embedding interacts with existential packages. The existential package is lifted out of the embedding, and the value $v$ becomes an embedded j-value with type annotation $\{\varepsilon/f\}\tau'$.

The reduction rules for remaining terms are not changed by the introduction of $\{\Delta\}$ and are therefore not shown. 

\subsection{Static Semantics}
\begin{figure}[t]
\footnotesize{
\noindent$\fbox{$\{\Delta\} \mid \Gamma  \vdash e_i : \tau$}$
\[
\begin{array}{c}

\infer[\textsc{(T-pack)}]
  {\{\Delta\}\mid \Gamma \vdash \m{pack} (\varepsilon, e) \m{as} \exists f.\ \tau : \exists f.\ \tau}
  {\{\Delta\}\mid \Gamma \vdash e : \{\varepsilon / f\} \tau} \\[3ex]
  
\infer[\textsc{(T-open)}]
  {\{\Delta\}\mid \Gamma \vdash \m{open} e \m{as} (f, x) \m{in} e' : \tau'}
  {\{\Delta\}\mid \Gamma \vdash e : \exists f.\ \tau & \forall \Delta_i \in \{\Delta\}, \Delta'_i = \Delta_i[f\mapsto f] & \{\Delta'\}\mid \Gamma, x:\tau \vdash e' : \tau'}

\end{array}
\]
\label{static-exist}
\caption{Additional Static Semantics for Existential Effects}
}
\end{figure}

The typing rules also requires the set of type maps, so the judgment have the form $\{\Delta\} \mid \Gamma \vdash e : \tau$
(T-Pack) assigns the type $\exists f.\ \tau$ to the existential package if the expression $e$ has type $\{\varepsilon/f\}\tau$. The rule (T-Open) assigns the type $\tau'$ to the open expression if $e$ has the existential type $\exists f.\ \tau$ and $e'$ has type $\tau'$ given that the context is extended with variable $x$, and the set of type maps is extended with the effect label $f$. 

Similar to the previous system, type relations ensures the correctness of the type annotation on embeddings. Since we introduced the existential type, we need to add an additional rule to the type relations:

$$\infer{\exists f.\ \tau \leq_l \exists f.\ \tau'}{\tau \leq_l \tau'}$$


\subsection{Type Safety}
\begin{lemma} (Preservation for expression)\\
If $\{\Delta\} \mid \Gamma \vdash e : \tau$ and $\langle \{\Delta\}, e\rangle \mapsto \langle \{\Delta'\}, e'\rangle$, then $\{\Delta'\} \mid \Gamma \vdash e' : \tau$
\begin{proof}
By rule induction on the dynamic semantics of expressions. 
\begin{enumerate}
\item (E-Congruence): By inversion on typing judgement and applying IH.
\item (E-Unit): By directly applying (T-Unit)
\item (E-Lambda): 
$$\infer[\textsc{(E-Lambda)}]
  {[\lambda x_j : \tau'.\ c_j]^{\tau \rightarrow \sigma}_{j} \longrightarrow \lambda x_i : \tau.\ [\{[x_i]^{\tau'}_{i}/x_j\}c_j]^\sigma_{jl}}
  {}$$
By inversion on (T-EmbedExp), we have $\{\Delta\} \mid \Gamma \vdash  [\lambda x_j : \tau'.\ c_j]^{\tau \rightarrow \sigma}_{j}: \tau \rightarrow \sigma$, and $\{\Delta\} \mid \Gamma \vdash \lambda x_j : \tau'.\ c_j : \tau' \rightarrow \sigma'$, where $\{\Delta\} \mid \Gamma \vdash \tau' \rightarrow \sigma' \leq_j \tau \rightarrow \sigma$. By inversion on (T-Lam), we have $\{\Delta\} \mid \Gamma, x_j : \tau' \vdash c_j : \sigma$. Then by substitution lemma, we have $\{\Delta\} \mid \Gamma, x_i : \tau  \vdash \{[x_i]^{\tau'}_i/x_j\}c_j : \sigma'$. By (T-Embed), 
$\{\Delta\} \mid \Gamma, x_i : \tau  \vdash [\{[x_i]^{\tau'}_i/x_j\}c_j : \sigma']^{\sigma}_j : \sigma$. Then the result follows by (T-Lam).

\item (E-Pack): By inversion and IH.
\item (E-Open): By inversion on (T-Pack), we have $\{\Delta\} \mid \Gamma \vdash e : \{\varepsilon /f\} \tau$. Then by (T-EmbedExp), we have $\{\Delta'\} \mid \Gamma \vdash [e]^\tau_j : \tau$. By inversion on (T-Open), $\{\Delta\} \mid \Gamma, x : \tau \vdash e' : \tau'$. Since $j$ is fresh, $e'$ doesn't contain any $j$ term, so the type information from agent $j$ doesn't affect the typing of $e'$. Therefore, $\{\Delta'\} \mid \Gamma, x : \tau \vdash e' : \tau'$. Finally, by substitution lemma, we have $\{\Delta'\} \mid \Gamma \vdash \{[e]^\tau_j / x \}e' : \tau'$.

\item (E-EmbedPack): By inversion on (T-EmbedExp), we have $\{\Delta\} \mid \Gamma \vdash \m{pack} (\varepsilon, v) \m{as} \exists f.\ \tau : \exists f.\ \tau$, and $\{\Delta\} \mid \Gamma \vdash \exists f.\ \tau \leq_j \exists f.\ \tau'$. By type relation, we have $\tau \leq_j \tau'$. Then by inversion on (T-Pack), we have $\{\Delta\} \mid \Gamma \vdash v :\{\varepsilon / f\}\tau$. Then by T-EmbedExp, we have $\{\Delta\}\mid\Gamma \vdash [v]^{\{\varepsilon / f\}\tau'}_j : \{\varepsilon / f\}\tau'$. Then by (T-Pack), we get $\{\Delta\}\mid\Gamma \vdash \m{pack} (\varepsilon, [v]^{\{\varepsilon / f\}\tau'}_j) \m{as} \exists f.\ \tau' :\exists f.\ \tau'$


\end{enumerate}
\end{proof}

\end{lemma}

\begin{lemma} (Progress)\\
If $\{\Delta\} \mid \Gamma \vdash e : \tau$ then either $e$ is a value or there exists $e'$ and $\{\Delta'\}$ such that $\langle \{\Delta\}, e \rangle \mapsto \langle \{\Delta'\}, e' \rangle$
\begin{proof}
By induction on the typing rule:
\begin{enumerate}[align=left]
\item[(T-Pack)] Let $e = \m{pack} (\varepsilon, e') \m{as} \exists f. \tau$ There are two cases. If $e'$ is value, then we are done. If $e'$ is not a value, by inversion and IH, we have $\langle \{\Delta\}, e' \rangle \mapsto \langle \{\Delta'\}, e'' \rangle$. Then we can apply E-Pack to evaluate $e$.
\item[(T-Open)] Again, let $e = \m{open} e' \m{as}(f, x) \in e''$. By IH and inversion, if $e'$ is not a value, then we can evaluate $e$ by (E-Open2). If $e'$ is a value, by inversion, $e' = \m{pack} (\varepsilon, e_1) \m{as} \exists f.\tau$. Then we can evaluate $e$ by applying (E-Open1).
\end{enumerate}

\end{proof}
\end{lemma}

\section{Discussion and Future Work}

\subsection{Parametric Polymorphism}

We introduced polymorphic effects in chapter \ref{chapter-bound} as a part of our restrictive effect system. However, we did not include  polymorphism in our agent-based core calculus. Parametric polymorphism on effects would significantly increase the expressiveness of the language. For example, in the example of the \m{var} module presented in the previous section, the type of the argument to the \m{handle} function is \m{Unit -> \{state\} Int}. So it restricts the handled computation to only have the \m{state} effect, and is therefore unrealistic as we may want to handle computations with other effects as well. 
\begin{lstlisting}[mathescape=true]
val var = 
  pack 
  <{get, set}, 
    < read $\hookrightarrow$ $\lambda$x:Unit. get((); y.return y),
      write $\hookrightarrow$ $\lambda$x:Int. set(x; y.return ()),
      handle $\hookrightarrow$  $\lambda$c: Unit $\rightarrow$ {state} Int. (with hstate handle c ()) 0 >
  > as $\exists$state. (read : Unit -> {state} Int) * (write : Int -> {state} Unit) * (handle :  Unit -> {state} Int -> {} Int)
\end{lstlisting} 

Therefore we would desire to add universal quantifications for effect variables to the system. However, unlike existential quantification, which is a straightforward extension to our calculus, the universal effect introduces a problem that breaks the abstraction barrier. The following example illustrates the problem brought by parametric polymorphism on effects:

\begin{lstlisting}
type B
  effect E {
  	def op() : Unit
  }

module b : B
  ...

type A
  effect E 
  def handle[F](c: Unit -> {F, this.E} Unit): {F} Unit
  def m(): {this.E} Unit
  
module a: A
  effect E = {b.E}
  def handle[F](c: Unit -> {F, this.E} Unit): {F} Unit
     handle
       c()
     with
       b.op() -> resume ()
  def m(): {this.E} Unit
     b.op()
  \end{lstlisting}

The module \m{a} defines a polymorphic handle function that handles a computation with effect \m{a.E} and a polymorphic effect \m{F}. Since effect \m{a.E} is defined abstract in the type \m{A}, the client of this module should not observe that fact that effect \m{a.E} is equivalent to \m{b.E}. However, the client1 passes effect \m{b.E} as the polymorphic effect into the handle function, and because the implementation of  handle function handles the effect \m{b.E}, client1 would surprised by the fact that the effect \m{b.E} is handled. In comparison, the client2 code passes an unrelated effect \m{c.E} to the handling function, and observes that the operation of effect \m{c.E} is not handled.

\begin{lstlisting}
//Client1: Prints nothing
handle 
  a.handle[b.E] (
     () => b.op(); a.m()
  )
with
    b.op() -> print "desired output"

//Client2: Prints "desired output"
handle 
  a.handle[c.E] (
     () => c.op(); a.m()
  )
with
    c.op() -> print "desired output"
    
\end{lstlisting}


\subsection{Effect Bounds}
As we have shown in chapter \ref{chapter-bound}, effect bounds are a useful tool for making the effect system expressive. Currently our calculus does not support bounded quantification because its formalization is different from the path-dependent formalization we previously developed to express effect bounds. 

One possible direction to achieve this is through the use of bounded existential types. Because we use existential types to express information hiding on effect types, it is natural to adopt the technique of bounded existential types to achieve the idea of effect bounds. One difficulty of such adoption may be the way to unify subeffecting with our multi-agent approach.












\begin{comment}
\section{Wyvern with Algebraic Effects hand Handlers}
\todo[inline]{Finish this section}

\subsection{Syntax}
\begin{figure}[htb]
\footnotesize{
\[
\begin{array}{lll}
\begin{array}{lllr}
e & ::= & x \\
& | & l\\
& | & \keyw{new}(x \Rightarrow \overline{d}) \\
& | & [e]_j^\tau\\
& | & e \rhd \tau\\

c & ::= & \m{return} e\\
& | & e.m(e)\\
& | & op(e; y.c)\\
& | & \m{with} h \m{handle} c\\
& | & [c]^\sigma\\
& | &[op]^\varepsilon(e; y.c)


\end{array}
\begin{array}{lllr}
d & ::= & m(x:\tau) = c \\
& |   & \keyw{effect} E =  \varepsilon \\
\varepsilon & ::= & \cdot \mid e.E, \varepsilon \mid op, \varepsilon\\
\tau & ::= & \{ x \Rightarrow \overline{\rho} \} \\
\Gamma & :: = & \varnothing~|~\Gamma,~x : \tau \\
\rho & ::= & m : \tau \rightarrow \sigma \\
       & |   & \keyw{effect} E \\
       & |   & \keyw{effect} E = \varepsilon  \\
\sigma & ::= & \{\varepsilon\}\tau \\
\end{array}
\end{array}
\]
}
\caption{Wyvern's object-oriented core syntax with algebraic effects and handlers}
\label{fig-wyv}
\end{figure}


In this section we extend the formalization of Wyvern programming language with abstract algebraic effects and handlers. In figure \ref{fig-wyv}, we show an updated version of the Wyverns's object-oriented, and it is extended with several new constructs.  The expression $x$ and location $l$ is the same from the original Wyvern language. The initialization $\keyw{new}(x \Rightarrow \overline{d})$ initializes an object with a list of declarations $\overline{d}$. 

The embedding of expression $[e]^\tau_j$ is added to separate code with different knowledge of effect abstraction, as shown in chapter \ref{chapter-algebraic}. We also introduce the type coercion $e \rhd \tau$ to make the subtyping of our calculus explicit. 

There are two forms of declarations. The method declaration $m(x:\tau) = c$, and the effect declaration $\keyw{effect} E = \varepsilon$. The object type has the form $\{x \Rightarrow \rho\}$, where $\rho$ is the declaration type. A declaration type can be an method type $m : \tau \rightarrow \sigma$, an abstract effect $\keyw{effect} E$, or a concrete effect $\keyw{effect} E = \varepsilon$. 

\subsection{Operational Semantics}
\begin{figure}[t]
\footnotesize{
\noindent$\fbox{$\langle \{\Delta\} \mid \mu \mid e \rangle \mapsto \langle \{\Delta\}, \mu, e \rangle$}$
\[
\begin{array}{c}

\infer[\textsc{(E-New)}]
	{\langle \{\Delta\}\mid \mu \mid \keyw{new}(x \Rightarrow \overline{d}) \mapsto \langle \{\Delta'\}\mid \mu, l \mapsto \{ x \Rightarrow \overline{d} \} \mid l\rangle }
	{l\ fresh & \forall \keyw{effect}\ E = \varepsilon \in \overline{d} 
	& \forall \Delta_k \in \{\Delta\} & \Delta_k' = \Delta_k[l.E \mapsto [l/x]\varepsilon] } \\[3ex]
	
\infer[\textsc{(E-EmbedExp1)}]
	{\langle \{\Delta\} \mid \mu \mid [e]^\tau_j \rangle \mapsto    \langle \{\Delta'\} \mid \mu' \mid [e']^\tau_j \rangle}
	{\langle \{\Delta\} \mid \mu \mid e \rangle \mapsto    \langle \{\Delta'\} \mid \mu' \mid e' \rangle }\\[3ex]
  
\infer[\textsc{(E-EmbedExp2)}]
	{\langle \{\Delta\} \mid \mu \mid [l]^\tau_j \rangle \mapsto    \langle \{\Delta'\} \mid \mu, l' \mapsto \{x \Rightarrow \overline{d'}\}  \mid l' \rangle}
	{\begin{gathered}
	  \tau = \{x \Rightarrow \overline{\rho} \} \qquad l \mapsto \{x \Rightarrow \overline{d} \}\in \mu \\
	 (\forall \rho_i \in \overline{\rho}  \qquad
	  \rho_i = m : \tau' \rightarrow \sigma  \qquad
	  m(y:\tau'') = c \in \overline{d} \qquad
	  d'_i = (m(z : \tau') = [\{[z]^{\tau''}_i/y\}c]^\sigma_j )) \\
	  (\forall \keyw{effect} E = \varepsilon \in \overline{d} \qquad d_i' = \keyw{effect} E = \varepsilon)\\
	  (\forall \keyw{effect}\ E  = \varepsilon \in \overline{d} \qquad  \forall \Delta_k \in \{\Delta\} 
	  \qquad \Delta_k' = \Delta_k[l'.E \mapsto l.E])
	  \end{gathered} } \\[3ex]
\infer[\textsc{(E-Coercion1)}]
	{\langle \{\Delta\} \mid \mu \mid e \rhd \tau \rangle \mapsto \langle \{\Delta'\} \mid \mu' \mid e' \rhd \tau \rangle}
	{\langle \{\Delta\} \mid \mu \mid e  \rangle \mapsto \langle \{\Delta'\} \mid \mu' \mid e'  \rangle} \\[3ex]

\infer[\textsc{(E-Coercion2)}]
	{\langle \{\Delta\} \mid \mu \mid l \rhd \tau \rangle \mapsto \langle \{\Delta'\} \mid \mu' \mid [l]^\tau_j \rangle}
	{j\ fresh & (\forall \keyw{effect} E \in \tau & ( \forall \Delta_k \in \{\Delta\} & \Delta_k(l.E) = l.E) & \Delta_j(l.E) = \varepsilon) }
\end{array}
\]
\label{dyn-wyv}
\caption{Dynamic semantics for Wyvern extended with algebraic effects}
}
\end{figure}



\subsection{Typing Rules}
\begin{figure}[t]
\footnotesize{
\noindent$\fbox{$\{\Delta\} \mid \Gamma  \vdash e_i : \tau$}$
\[
\begin{array}{c}

\infer[\textsc{(T-var)}]
  {\{\Delta\}\mid \mu  \mid \Gamma \vdash x : \Gamma(x)}
  {} \\[3ex]
  
\infer[\textsc{(T-loc)}]
  {\{\Delta\}\mid \mu, l \mapsto \{x \Rightarrow \overline{d}\} \mid \Gamma \vdash l : \{x \Rightarrow \overline{\rho}\}}
  {\forall i &  \{\Delta\}\mid \mu \mid \Gamma, x : \{x \Rightarrow \overline{\rho}\}  \vdash d_i : \rho_i} \\[3ex]

\infer[\textsc{(T-new)}] 
	{\{\Delta\}\mid \mu \mid \Gamma \vdash  \keyw{new}(x \Rightarrow \overline{d}) : \{ x \Rightarrow \overline{\rho}\}}
	{\{\Delta\}\mid \mu \mid \Gamma, x : \{x \Rightarrow \overline{\rho}\} \vdash d_i : \rho_i}\\[3ex]

\infer[\textsc{(T-EmbedExp)}] 
	{\{\Delta\}\mid \mu \mid \Gamma \vdash [e]^\tau_j : \tau}
	{\{\Delta\}\mid \mu \mid \Gamma \vdash e : \tau' & \tau' \leq_{ji} \tau }\\[3ex]


\infer[\textsc{(T-Coercion)}] 
	{\{\Delta\}\mid \mu \mid \Gamma \vdash e \rhd \tau : \tau}
	{ \{\Delta\}\mid \mu \mid \Gamma \vdash e : \tau' & \tau' <: \tau}\\[3ex]
\end{array}
\]
\label{static-wyv}
\caption{Typing rules for Wyvern extended with algebraic effects}
}
\end{figure}

\begin{figure}[t]
\footnotesize{
\noindent$\fbox{$\tau <: \tau$}$
\[
\begin{array}{c}
  
\infer[\textsc{(S-Depth)}]
  {\{ \overline{\rho} \} <:  \{\overline{\rho'}\}}
  {\forall i, \rho_i <: \rho'_i} \\[3ex]
  
\end{array}
\]

\noindent$\fbox{$\rho <: \rho$}$
\[
\begin{array}{c}
  
\infer[\textsc{(S-Refl)}]
  {\rho <: \rho}
  {} \\[3ex]
  
\infer[\textsc{(S-Effect)}]
  {\keyw{effect} E = \varepsilon <: \keyw{effect} E}
  {} \\[3ex]
  
\end{array}
\]


\label{subtyping-wyv}
\caption{Subtyping rules for Wyvern extended with algebraic effects}
}
\end{figure}




\subsection{Type Safety}
\begin{lemma} (Preservation for expressions)
If $\{\Delta\}\mid \mu \mid \Gamma \vdash e : \tau$ and $\langle \{\Delta\} \mid \mu \mid e \rangle \mapsto  langle \{\Delta'\} \mid \mu' \mid e' \rangle$, then $\{\Delta'\}\mid \mu' \mid \Gamma \vdash e' : \tau$

\begin{proof} (Sketch)
By induction on the reduction rule:
\begin{enumerate}
\item (E-New) The object initialization is transformed to a location $l$. According to the typing rule. The type of the location should be identical to the type of the object.
\item (E-EmbedExp1) Follows immediately by applying IH
\item (E-EmbedExp2)  By inversion on (T-EmbedExp), we have $\{\Delta\} \mid \mu \mid \Gamma \vdash [l]^\tau_j : \tau$. 
\end{enumerate}
\end{proof}

\end{lemma}
\end{comment}



